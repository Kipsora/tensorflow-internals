%%%%%%%%------------------------------------------------------------------------
% This program is free software: you can redistribute it and/or modify
% it under the terms of the GNU General Public License as published by
% the Free Software Foundation, either version 3 of the License, or
% (at your option) any later version.
% 
% This program is distributed in the hope that it will be useful,
% but WITHOUT ANY WARRANTY; without even the implied warranty of
% MERCHANTABILITY or FITNESS FOR A PARTICULAR PURPOSE.  See the
% GNU General Public License for more details.
% 
% You should have received a copy of the GNU General Public License
% along with this program.  If not, see <http://www.gnu.org/licenses/>.

%%%%%%%%------------------------------------------------------------------------
%%%% 导言区 
%% 文档类型为article
\documentclass[a4paper, 10pt]{book}
%1m = 39.4 inch
%大18开 (18.5cm * 23cm)
%\usepackage[left=3.25cm, right=3.25cm, top=2.3cm,bottom=1.4cm]{geometry}
\usepackage{geometry}
\geometry{left=3.75cm,right=3.25cm,top=3cm,bottom=2.5cm}

%% en_preamble包含基本的宏包配置
\input{style/en_preamble}

%% 如果不写中文的话就不需要引用xecjk_preamble里面的配置
\input{style/xecjk_preamble}

\input{style/coding}

% \usepackage[
%   placement=center,
%   angle=45,
%   scale=20,
%   color=black!40,
%   %hshift=60,
%   %vshift=-5
% ]{background}

% \backgroundsetup{contents={样章}}
% \backgroundsetup{contents={\includegraphics[width=0.2\textwidth]{figures/cock.jpg}}}

\newcommand{\myclearpage}{\clearpage{\pagestyle{empty}\cleardoublepage}}
\newcommand{\mydedicate}{\clearpage{\pagestyle{empty}\cleardedicatepage}}
\newcommand{\alert}[1]{\textbf{#1}}

%%%% 导言区结束
%%%%%%%%------------------------------------------------------------------------

%%%%%%%%------------------------------------------------------------------------
%%%% 正文部分

\begin{document}

\frontmatter
\pagestyle{empty}

%%自定义封面
\def\titlename{TensorFlow内核剖析}
\def\subtitle{TensorFlow Internals}
\def\authors{刘光聪\ 著}
% \def\orgnization{\ascii{}}

\input{style/title}
\myclearpage

\mydedicate

\input{contents/foreword}
\myclearpage

\input{contents/preface}
\myclearpage

\tableofcontents
\myclearpage

\def\thelstlisting{\thechapter-\arabic{lstlisting}}
%% 中文习惯是设定首行缩进为2em。
%% 注意此设置一定要在document环境之中,这可能与\setlength作用范围相关
\setlength{\parindent}{2em}

%%%%%%%%%%%%%%%%%%%%%%
%%开始正文,页面计数从正文开始
\mainmatter
\setcounter{page}{1}
% \pagestyle{fancy}

% \input{contents/logistic-regression}

\part{基础知识}
\input{contents/introduction}
\begin{savequote}[45mm]
\ascii{Any fool can write code that a computer can understand. Good programmers write code that humans can understand.}
\qauthor{\ascii{- Martin Flower}}
\end{savequote}

\chapter{编程环境} 
\label{ch:prog-env}

\begin{content}

为了实现\tf{}的快速入门,本章将介绍\tf{}的编程环境,包括代码结构,工程构建,以便对\tf{}的系统架构建立基本的感性认识。

\end{content}

\section{代码结构}

\begin{content}

\subsection{克隆源码}

首先,从\ascii{Github}上克隆\tf{}的源代码。


\begin{leftbar}
\begin{python}
$ git clone git@github.com:tensorflow/tensorflow.git
\end{python}
\end{leftbar}

然后,切换到最新的稳定分支上。例如,\code{r1.4}分支。

\begin{leftbar}
\begin{python}
$ cd tensorflow
$ git checkout r1.4
\end{python}
\end{leftbar}

\subsection{源码结构}

运行如下命令,打印出\tf{}源码的组织结构。

\begin{leftbar}
\begin{python}[]
$ tree -d -L 1 ./tensorflow
\end{python}
\end{leftbar}

其中,本书将重点关注\code{core, python}组件,部分涉及\code{c, cc, stream\_executor}组件。

\begin{leftbar}
\begin{c++}[caption={TensorFlow源码结构}]
./tensorflow
├── c
├── cc
├── compiler
├── contrib
├── core
├── docs_src
├── examples
├── g3doc
├── go
├── java
├── python
├── stream_executor
├── tools
└── user_ops
\end{c++}
\end{leftbar}

截止当前最新发布的\ascii{1.4}版本,\tf{}代码库拥有大约\ascii{100}万代码。其中,包括\ascii{53}万行\ascii{C/C++}代码,\ascii{37}万行\ascii{Python}代码,而且代码规模在不断膨胀之中。其中,\ascii{Python}提供的\ascii{API}是最完善的;相比之下,其他编程语言的\ascii{API}尚未成熟,甚至处于起步阶段。

\begin{leftbar}
\begin{python}[caption={TensorFlow代码统计}]
-------------------------------------------------------
Language             files    blank    comment    code
-------------------------------------------------------
C++                   2238    77610     68275    443099
Python                1881    92018    151807    369399
C/C++ Header          1175    27392     46215     86691
Markdown               218     8859         2     30925
CMake                   50     2183       986     16398
Go                      28     1779     13290     15003
Java                    72     1789      3111      7779
Bourne Shell           103     1487      3105      6074
Protocol Buffers        87     1313      3339      3452
Objective C++            9      227       181      1201
C                        8      157       130       941
make                     4      105       136       612
XML                     25      135       265       315
Groovy                   3       46        74       246
Maven                    5       21         4       245
DOS Batch                9       30         0       143
Dockerfile               7       41        69       133
Perl                     2       29        38       130
Bourne Again Shell       3       24        63       111
JSON                     3        0         0        23
Objective C              1       10        13        21
YAML                     1        3        24        15
-------------------------------------------------------
SUM:                  5932   215258    291127    982956
-------------------------------------------------------
\end{python}
\end{leftbar}

\subsection{Core}

内核的源码结构如下所示,主要包括平台,实用函数库,基础框架,\ascii{Protobuf}定义,本地运行时,分布式运行时,图操作,\ascii{OP}定义,以及\ascii{Kernel}实现等组成,这是本书重点剖析的组件之一,将重点挖掘基础框架中隐藏的领域模型,追踪整个运行时环境的生命周期和图操作的详细过程,并揭示常见\ascii{OP}的\ascii{Kernel}实现原理和运行机制。

\begin{leftbar}
\begin{c++}[caption={Core源码结构}]
./tensorflow/core
├── common_runtime
├── debug
├── distributed_runtime
├── example
├── framework
├── graph
├── grappler
├── kernels
├── lib
├── ops
├── platform
├── profiler
├── protobuf
├── public
├── user_ops
└── util
\end{c++}
\end{leftbar}

其中,\code{core}主要由\code{C++}实现,大约拥有\ascii{26}万行代码。

\begin{leftbar}
\begin{python}[caption={Core代码统计}]
-------------------------------------------------------
Language             files    blank   comment      code
-------------------------------------------------------
C++                   1368    44791     38968    259289
C/C++ Header           653    15040     24474     50506
Protocol Buffers        57      736      2371      1806
Markdown                11      327         0      1285
JSON                     2        0         0        18
-------------------------------------------------------
SUM:                  2091    60894     65813    312904
-------------------------------------------------------
\end{python}
\end{leftbar}

\subsection{Python}

\ascii{Python}定义和实现了\tf{}的编程模型,并对外开放\ascii{API}供程序员使用,其源码结构如下所示,这也是本书重点剖析的部分。

\begin{leftbar}
\begin{c++}[caption={Python源码结构}]
./tensorflow/python
├── client
├── debug
├── estimator
├── feature_column
├── framework
├── grappler
├── kernel_tests
├── layers
├── lib
├── ops
├── platform
├── profiler
├── saved_model
├── summary
├── tools
├── training
├── user_ops
└── util
\end{c++}
\end{leftbar}

其中,该组件由\code{Python}实现,大约有\ascii{18}万行代码。

\begin{leftbar}
\begin{python}[caption={Python代码统计}]
-------------------------------------------------------
Language            files     blank   comment      code
-------------------------------------------------------
Python                714     45769     69407    179565
C++                    20       496       506      3658
C/C++ Header           15       207       387       363
Markdown                4        48         0       200
Protocol Buffers        3        16        10        71
Bourne Shell            1        13        28        68
-------------------------------------------------------
SUM:                  757     46549     70338    183925
-------------------------------------------------------
\end{python}
\end{leftbar}

\subsection{Contrib}

\code{contrib}是第三方贡献的编程库,它也是\tf{}标准化之前的实验性编程接口,犹如\ascii{Boost}社区与\ascii{C++}标准之间的关系。当\code{contrib}的接口成熟后,便会被\tf{}标准化,并从\code{contrib}中搬迁至\code{core, python}中,并正式对外发布。

\begin{leftbar}
\begin{python}[caption={Contrib源码结构}]
./tensorflow/contrib
├── android
├── batching
├── bayesflow
├── benchmark_tools
├── boosted_trees
├── cloud
├── cluster_resolver
├── cmake
├── compiler
├── copy_graph
├── crf
├── cudnn_rnn
├── data
├── decision_trees
├── deprecated
├── distributions
├── eager
├── factorization
├── ffmpeg
├── framework
├── fused_conv
├── gdr
├── graph_editor
├── grid_rnn
├── hooks
├── hvx
├── image
├── imperative
├── input_pipeline
├── integrate
├── keras
├── kernel_methods
├── labeled_tensor
├── layers
├── learn
├── legacy_seq2seq
├── linalg
├── linear_optimizer
├── lookup
├── losses
├── makefile
├── memory_stats
├── meta_graph_transform
├── metrics
├── mpi
├── nccl
├── ndlstm
├── nearest_neighbor
├── nn
├── opt
├── pi_examples
├── predictor
├── quantization
├── reduce_slice_ops
├── remote_fused_graph
├── resampler
├── rnn
├── saved_model
├── seq2seq
├── session_bundle
├── signal
├── slim
├── solvers
├── sparsemax
├── specs
├── staging
├── stat_summarizer
├── stateless
├── tensor_forest
├── tensorboard
├── testing
├── text
├── tfprof
├── timeseries
├── tpu
├── training
├── util
├── verbs
└── xla_tf_graph
\end{python}
\end{leftbar}

由于\tf{}社区相当活跃,\code{contrib}的变更相当频繁,截止\ascii{1.4}版本,大约有\ascii{23}万行代码,主要由\ascii{Python}设计和实现的编程接口,部分运行时由\ascii{C++}实现。

\begin{leftbar}
\begin{python}[caption={Contrib代码统计}]
-------------------------------------------------------
Language            files     blank   comment      code
-------------------------------------------------------
Python               1007     41436     75096    170355
C++                   201      5500      5313     32944
CMake                  48      2172       955     16358
C/C++ Header           99      1875      2867      6583
Markdown               46      1108         0      3485
Bourne Shell           18       232       386      1272
C                       7       151       118       931
Protocol Buffers       20       224       454       680
make                    4       105       136       612
Java                    2        77       209       335
Groovy                  1        10        20        75
Bourne Again Shell      1         6        15        59
Dockerfile              1         2         1        14
XML                     2         3         0         9
-------------------------------------------------------
SUM:                 1457     52901     85570    233712
-------------------------------------------------------
\end{python}
\end{leftbar}

\subsection{StreamExecutor}

\ascii{StreamExecutor}是\ascii{Google}另一个开源组件库,它提供了主机端(\ascii{host-side})的编程模型和运行时环境,实现了\ascii{CUDA}和\ascii{OpenCL}的统一封装。使得在主机端的代码中,可以将\ascii{Kernel}函数无缝地部署在\code{CUDA}或\code{OpenCL}的计算设备上执行。

目前,\ascii{StreamExecutor}被大量应用于\ascii{Google}内部\ascii{GPGPU}应用程序的运行时。其中,\tf{}运行时也包含了一个\ascii{StreamExecutor}的快照版本,用于封装\ascii{CUDA}和\code{OpenCL}的运行时。本书将简单介绍\ascii{CUDA}的编程模型和线程模型,并详细介绍\ascii{StreamExecutor}的系统架构与工作原理,揭示\ascii{Kernel}函数的实现模式和习惯用法。

\begin{leftbar}
\begin{c++}[caption={StreamExecutor源码结构}]
./tensorflow/stream_executor
├── cuda
├── host
├── lib
└── platform
\end{c++}
\end{leftbar}

其中,\ascii{StreamExecutor}使用\code{C++}实现,大约有\ascii{2.5}万行代码。

\begin{leftbar}
\begin{python}[caption={StreamExecutor代码统计}]
-------------------------------------------------------
Language            files     blank   comment      code
-------------------------------------------------------
C++                    43      2440      1196     16577
C/C++ Header           81      2322      5080      8625
-------------------------------------------------------
SUM:                  124      4762      6276     25202
-------------------------------------------------------
\end{python}
\end{leftbar}

\subsection{Compiler}

众所周知,灵活性是\tf{}最重要的设计目标和核心优势,因此\tf{}的系统架构具有良好的可扩展性。\tf{}可用于定义任意图结构,并使用异构的计算设备有效地执行。但是,熊掌与鱼翅不可兼得,当低级\ascii{OP}组合为计算子图时,并期望在\ascii{GPU}上有效执行时,运行时将启动更多的\ascii{Kernel}的运算。

因此,\tf{}分解和组合\ascii{OP}的方法,在运行时并不能保证以最有效的方式运行。此时,\ascii{XLA}技术孕育而生,它使用\ascii{JIT}编译技术来分析运行时的计算图,它将多个\ascii{OP}融合在一起,并生成更高效的本地机器代码,提升计算图的执行效率。

\begin{leftbar}
\begin{python}[caption={Compiler源码结构}]
./tensorflow/compiler
├── aot
├── jit
├── plugin
├── tests
├── tf2xla
└── xla
\end{python}
\end{leftbar}

\ascii{XLA}技术目前处于初级的研发阶段,是\tf{}社区较为活跃研究方向,截止目前代码规模大约为\ascii{12.5}万行,主要使用\ascii{C++}实现。

\begin{leftbar}
\begin{python}[caption={Compiler代码统计}]
-------------------------------------------------------
Language            files     blank   comment      code
-------------------------------------------------------
C++                   455     19010     18334    102537
C/C++ Header          250      5939     10323     15510
Python                 37      1255      1416      6446
Protocol Buffers        5       312       501       781
Markdown                2         0         0         3
-------------------------------------------------------
SUM:                  749     26516     30574    125277
-------------------------------------------------------
\end{python}
\end{leftbar}

\end{content}

\section{工程构建}

\begin{content}

在开始之前,尝试\tf{}源码的构建过程,了解\tf{}的基本构建方式、工具,及其依赖的组件库、第三方工具包,对于理解\tf{}工作原理具有很大的帮助。但是,因篇幅受限,本章仅以\ascii{Mac OS}系统为例,讲述\tf{}的源码编译、安装、及其验证过程。其他操作系统,请查阅\tf{}发布的官方文档。

\subsection{环境准备}

\ascii{TensorFlow}的前端是一个支持多语言的编程接口。因此,编译\ascii{TensorFlow}源代码之前,需要事先安装相关的编译器、解释器、及其运行时环境。例如,使用\ascii{Python}作为编程接口,需要事先安装\ascii{Python}解释器。其次,在构建系统之前,也需要事先安装\ascii{GCC}或\ascii{Clang}等\ascii{C++}编译器,用于编译后端系统实现。因为\ascii{TensorFlow}使用\ascii{C++11}语法实现,所以要保证安装\ascii{C++}编译器要支持\ascii{C++11}。另外,\ascii{TensorFlow}使用\ascii{Bazel}的构建工具,可以将其视为更高抽象的\ascii{Make}工具。不幸的是,\ascii{Bazel}使用\ascii{Java8}实现,其依赖于\ascii{JDK}。因此在安装\ascii{Bazel}之前,还得需要事先安装\ascii{1.8}及以上版本的\ascii{JDK}。

\subsubsection{安装JDK}

推荐从\ascii{Oracle}官网上下载\ascii{1.8}版本的\ascii{JDK},然后创建相关的环境变量,并将其添加到\code{~/.bashrc}的配置文件中。

\begin{leftbar}
\begin{python}
$ echo 'export JAVA_HOME=$(/usr/libexec/java_home)' >> ~/.bashrc
$ echo 'export PATH="$JAVA_HOME/bin:$PATH"' >> ~/.bashrc
\end{python}
\end{leftbar}

\subsubsection{安装Bazel}

在\ascii{Mac OS}上,可以使用\ascii{brew}安装\ascii{Bazel}。

\begin{leftbar}
\begin{python}
$ brew install bazel
\end{python}
\end{leftbar}

如果系统未安装\ascii{brew},可以执行如下命令先安装\ascii{brew}。当然,安装\ascii{brew}需要事先安装\ascii{Ruby}解释器,在此不再冗述。

\begin{leftbar}
\begin{python}
$ ruby -e "$(curl -fsSL https://raw.githubusercontent.com/Homebrew/install/master/install)"
\end{python}
\end{leftbar}

\subsubsection{安装Swig}

\ascii{TensorFlow}使用\ascii{Swig}构建多语言的编程环境,自动生成相关编程语言的包装器。因此,在构建之前需要事先安装\ascii{Swig}的工具包。

\begin{leftbar}
\begin{python}
$ brew install swig
\end{python}
\end{leftbar}

\subsubsection{安装Python依赖包}

使用\ascii{pip}安装\ascii{TensorFlow}所依赖的\ascii{Python}包。

\begin{leftbar}
\begin{python}
$ sudo pip install six numpy wheel autograd
\end{python}
\end{leftbar}

如果系统未安装\ascii{pip},则可以使用\ascii{brew}预先安装\ascii{pip}。

\begin{leftbar}
\begin{python}
$ brew install pip
\end{python}
\end{leftbar}

\subsubsection{安装CUDA工具包}

当系统安装了\ascii{CUDA}计算兼容性大于或等于\ascii{3.0}的\ascii{GPU}卡时,则需要安装\ascii{CUDA}工具包,及其\ascii{cuDNN},实现\tf{}运行时的\ascii{GPU}加速。推荐从\ascii{NVIDIA}官网上下载\ascii{CUDA Toolkit 8}及以上版本,并安装到系统中,配置相关环境变量。

\begin{leftbar}
\begin{python}
$ echo 'export CUDA_HOME=/usr/local/cuda' >> ~/.bashrc
$ echo 'export LD_LIBRARY_PATH=$CUDA_HOME/lib:$LD_LIBRARY_PATH' >> ~/.bashrc
\end{python}
\end{leftbar}

然后,再下载\ascii{cuDNN 5.1}及以上版本,并将其解压至\code{CUDA\_HOME}相关系统目录中。

\begin{leftbar}
\begin{python}
$ sudo tar -xvf cudnn-8.0-macos-x64-v5.1.tgz -C /usr/local
\end{python}
\end{leftbar}

\subsection{配置}

至此,编译环境一切就绪,执行\code{./configure}配置\ascii{TensorFlow}的编译环境了。当系统支持\ascii{GPU},则需要配置相关的\ascii{CUDA/cuDNN}编译环境。

\begin{leftbar}
\begin{python}
$ ./configure
\end{python}
\end{leftbar}

\subsection{构建}

当配置成功后,使用\ascii{Bazel}启动\ascii{TensorFlow}的编译。在编译启动之前,会尝试从代码仓库中下载相关依赖库的源代码,包括\ascii{gRPC, Protobuf, Eigen}等,并自动完成编译。

\begin{leftbar}
\begin{python}
$ bazel build --config=opt //tensorflow/tools/pip_package:build_pip_package
\end{python}
\end{leftbar}

当支持\ascii{GPU}计算时,添加\code{--config=cuda}编译选项。

\begin{leftbar}
\begin{python}
$ bazel build -c opt --config=cuda //tensorflow/tools/pip_package:build_pip_package
\end{python}
\end{leftbar}

编译成功后,便可以构建\ascii{TensorFlow}的\ascii{Wheel}包。

\begin{leftbar}
\begin{python}
$ bazel-bin/tensorflow/tools/pip_package/build_pip_package /tmp/tensorflow_pkg
\end{python}
\end{leftbar}

\subsection{安装}

当\ascii{Wheel}包构建成功后,使用\ascii{pip}安装\ascii{TensorFlow}到系统中。

\begin{leftbar}
\begin{python}
$ sudo pip install /tmp/tensorflow_pkg/tensorflow-1.4.0-py2-none-any.whl
\end{python}
\end{leftbar}

\subsection{验证}

启动\ascii{Python}解释器,验证\ascii{TensorFlow}安装是否成功。

\begin{leftbar}
\begin{python}
$ python
>>> import tensorflow as tf
>>> hello = tf.constant('Hello, TensorFlow!')
>>> sess = tf.Session()
>>> print(sess.run(hello))
Hello, TensorFlow!
\end{python}
\end{leftbar}

\subsection{IDE}

在阅读代码之前,选择一个适宜的\ascii{IDE}可以改善代码阅读的质量和速度。推荐使用\ascii{Eclipse CDT}阅读\ascii{C++}代码,安装\code{PyDev}的插件阅读\ascii{Python}代码。同时,也推荐\ascii{JetBrains}出品的\ascii{Clion}阅读\ascii{C++},\ascii{PyCharm}阅读\ascii{Python}。但是,当阅读\ascii{C++}代码时,需要配置\ascii{TensorFlow, CUDA, Eigen3}头文件的搜索目录,并添加相关预定义的宏,以便\code{IDE}正确解析代码中的符号。本章以\ascii{Eclipse CDT}为例讲述相关配置方法。

\subsubsection{创建Eclipse工程}

创建一个\ascii{Eclipse C++}工程,如\refig{setup-eclipse}所示。确定唯一的项目名称,手动地指定\ascii{TensorFlow}源代码的根目录,并选择\ascii{Makefile}的空工程。然后,按照\ascii{Properties > C/C++ General > Paths and Symbols > Includes}配置头文件的搜索目录。

\begin{table}[!htbp]
\resizebox{0.95\textwidth}{!} {
\begin{tabular*}{1.2\textwidth}{@{}ll@{}}
\toprule
\ascii{配置项} & \ascii{目录} \\
\midrule
\ascii{TensorFlow} & \code{/usr/local/lib/python2.7/site-packages/tensorflow/include} \\
\ascii{CUDA} & \code{/usr/local/cuda/include} \\ 
\bottomrule
\end{tabular*}
}
\caption{头文件搜索目录}
\label{tbl:tf-includes}
\end{table}

\begin{figure}[!htbp]
\centering
\includegraphics[width=0.75\textwidth]{figures/setup-eclipse.png}
\caption{创建Eclipse C++工程}
 \label{fig:setup-eclipse}
\end{figure}

\subsubsection{配置Eigen}

不幸的是,\ascii{Eigen}对外公开的头文件缺少\code{.h}的后缀名,\ascii{CDT}无法解析相关的符号。请参阅\code{\href{http://eigen.tuxfamily.org/index.php?title=IDEs}{http://eigen.tuxfamily.org/index.php?title=IDEs}}相关说明,按照\ascii{Preferences > C/C++ > Coding Style > Organize Includes > Header Substitution}导入\code{eigen-header-substitution.xml}文件,如\refig{eclipse-eigen3}所示。

\begin{figure}[!htbp]
\centering
\includegraphics[width=0.75\textwidth]{figures/eclipse-eigen3.png}
\caption{替换\ascii{Eigen}的头文件}
 \label{fig:eclipse-eigen3}
\end{figure}

\end{content}

\section{代码生成}

\begin{content}

在构建\ascii{TensorFlow}系统时,\ascii{Bazel}或\ascii{CMake}会自动生成部分源代码。理解代码生成器的输出结果,可以加深理解系统的行为模式。

\end{content}

\section{技术栈}

\begin{content}

如\refig{tf-stack}所示,按照系统的层次结构展示了\tf{}的技术栈,构成了\tf{}生态系统的核心。

\begin{figure}[H]
\centering
\includegraphics[width=0.7\textwidth]{figures/tf-stack.png}
\caption{TensorFlow技术栈}
 \label{fig:tf-stack}
\end{figure}

\end{content}

\begin{savequote}[45mm]
\ascii{Any fool can write code that a computer can understand. Good programmers write code that humans can understand.}
\qauthor{\ascii{- Martin Flower}}
\end{savequote}

\chapter{破冰之旅} 
\label{ch:ice-breaker}

\begin{content}

在开始探究\tf{}内核之前,亲自动手实践模型的训练,熟悉模型训练的基本方法和调优技术,对于理解后续章节的内容将大有裨益。通过本文学习和实践,将了解到如何构建并训练出能够识别手写数字的神经网络\footnote{ 本章内容摘自\ascii{Martin G\"{o}rner}在\ascii{Codelabs}上发表的文章:\href{https://codelabs.developers.google.com/codelabs/cloud-tensorflow-mnist}{Tensorflow and deep learning, without a PhD},经由\ascii{Martin G\"{o}rner}同意,授权该文章在本书中发表。}。

本章将循序渐进地分别使用单层感知器模型,多层感知器模型,最后尝试使用卷积神经网络。并在训练过程中,介绍了算法调优的一些常用技术,包括选择更好的激活函数,应用学习速率衰减的技术,实施\ascii{Dropout}技术等。最终,将模型的准确率提升至\ascii{99%}以上。

在介绍每一种网络模型之前,将简单给出该模型的算法理论知识,帮助大家更好地理解程序的内容。但是,本书不是介绍机器学习算法的专业书籍,如需了解更多相关算法内容,请查阅相关文献和论文。

\end{content}

\section{问题提出}

\begin{content}

本章使用\ascii{MNIST}数据集完成手写数字的网络模型训练,它包含了\ascii{60000}个训练样本数据;其中,包括\ascii{10000}个测试样本数据。如\refig{mnist-x}所示,对于任意一个样本数据$x$,使用$28 \times 28$像素的数字矩阵表示。为了简化,将$28 \times 28$的矩阵实施扁平化处理,得到长度为\ascii{784}的一维向量。

\begin{figure}[H]
\centering
\includegraphics[width=0.9\textwidth]{figures/mnist-x.png}
\caption{MNIST样本数据表示}
 \label{fig:mnist-x}
\end{figure}

\subsection{样本数据集}

因此,在\ascii{MNIST}训练数据集中,\code{mnist.train.images}是一个\code{[60000, 784]}的二维矩阵。其中,矩阵中每一个元素,表示图片中某个像素的强度值,其值介于\ascii{0}和\ascii{1}之间。如\refig{mnist-train-xs}所示。

\begin{figure}[H]
\centering
\includegraphics[width=0.9\textwidth]{figures/mnist-train-xs.png}
\caption{MNIST训练数据集:输入数据集}
 \label{fig:mnist-train-xs}
\end{figure}

相对应地,\ascii{MNIST}数据集的标签是介于\ascii{0}到\ascii{9}的数字,\code{mnist.train.labels}是一个\code{[60000, 10]}的二维矩阵,其中每一行是一个\ascii{\quo{one-hot}}向量。如\refig{mnist-train-ys}所示。

\begin{figure}[H]
\centering
\includegraphics[width=0.9\textwidth]{figures/mnist-train-ys.png}
\caption{MNIST训练数据集:标签数据集}
 \label{fig:mnist-train-ys}
\end{figure}

\subsection{图示说明}

为了更好地可视化整个训练过程,使用\ascii{matplotlib}工具包绘制了\ascii{5}种类型的画板。如\refig{mnist-training-digits}所示,表示一个\ascii{mini-batch}的训练样本数据集。其中,\code{batch\_size = 100},白色背景表示数字被正确识别;而红色背景表示数字被误分类,手写数字的左侧标识了正确的标签值,而右侧标识了错误的预测值。

\ascii{MNIST}拥有\ascii{50000}个训练样本,如果\code{batch\_size}为\ascii{100},则需要迭代\ascii{500}次才能完整地遍历一次训练样本数据集,常称为一个\ascii{epoch}周期。

\begin{remark}
本章示例代码未使用\ascii{TensorBoard},而是用了\ascii{matplotlib},在训练时可以实时观察误差和精度的曲线变化。
\end{remark}

\begin{figure}[H]
\centering
\includegraphics[width=0.6\textwidth]{figures/mnist-training-digits.jpeg}
\caption{一次mini-batch的训练样本数据集,其中\code{batch\_size=100}}
 \label{fig:mnist-training-digits}
\end{figure}

如\refig{mnist-test-digits}所示,\ascii{MNIST}使用了规模为\ascii{10000}的测试样本数据集测试模型的当前精度。其中,左侧表示目前模型的大致精度;同样地,白色背景表示数字被正确识别;而红色背景表示数字被误分类,手写数字的左侧标识了正确的标签值,而右侧标识了错误的预测值。

\begin{figure}[H]
\centering
\includegraphics[width=0.6\textwidth]{figures/mnist-test-digits.jpeg}
\caption{当前的模型精度:基于测试样本数据集}
 \label{fig:mnist-test-digits}
\end{figure}

如\refig{mnist-cross-entropy-loss-fig}所示,使用交叉熵函数量化预测值与标签值之前的误差。其中,\ascii{x}轴表示迭代的次数,\ascii{y}轴表示损失值。另外,基于训练样本数据集,损失值的曲线抖动较大;而基于测试样本数据集,损失值的曲线抖动较小。

\begin{figure}[H]
\centering
\includegraphics[width=0.6\textwidth]{figures/mnist-cross-entropy-loss-fig.jpeg}
\caption{训练和测试的交叉熵损失}
 \label{fig:mnist-cross-entropy-loss-fig}
\end{figure}

如\refig{mnist-accuracy-fig}所示,可以实时计算得到模型在当前训练数据集和测试集上的精度。其中,\ascii{x}轴表示迭代的次数,\ascii{y}轴表示精度值。同理,基于训练样本数据集,精度曲线抖动较大;而基于测试样本数据集,精度曲线抖动较小。

\begin{figure}[H]
\centering
\includegraphics[width=0.6\textwidth]{figures/mnist-accuracy-fig.jpeg}
\caption{训练和测试的精度}
 \label{fig:mnist-accuracy-fig}
\end{figure}

如\refig{mnist-weight-fig}所示,对于模型的每一个训练参数(包括偏置),可以统计得到其对应的数值分布图。当模型不能收敛时,参数的数值分布图能够给出有帮助的提示信息。

\begin{figure}[H]
\centering
\includegraphics[width=0.6\textwidth]{figures/mnist-weight-fig.png}
\caption{权重分布图}
 \label{fig:mnist-weight-fig}
\end{figure}

\end{content}

\section{单层感知器}

\begin{content}

首先,尝试构建\ascii{10}个神经元的单层感知器。如\refig{mnist-slp}所示,对于诸如手写数字识别的多分类问题,理论上常使用\ascii{softmax}的激活函数。

\begin{figure}[H]
\centering
\includegraphics[width=0.8\textwidth]{figures/mnist-slp.png}
\caption{单层感知器}
 \label{fig:mnist-slp}
\end{figure}

\subsection{理论基础}

理论上,\ascii{softmax}回归是\ascii{logistic}回归的广义扩展。其中,\ascii{logistic}回归是为了解决二分类问题,即$y \in \{ 0,1\}$;而\ascii{softmax}回归是为了解决$ k $分类问题,即$y \in \{ 1,2,...,k\}$。

\subsubsection{符号定义}

为了形式化地描述\ascii{softmax}回归问题,此处定义了一些常用符号。

 \begin{itemize}
   \item \ascii{训练样本集}: $ S = \{ ({x^{(i)}},{y^{(i)}});i = 1,2,...,m\} $
   \item \ascii{第$i$个训练样本}: $ ({x^{(i)}},{y^{(i)}}) $
   \item \ascii{样本输入}: $ x = ({x_1},{x_2},...,{x_n})^{T}  \in {\mathbb{R}^n} $
   \item \ascii{样本标签(one-hot)}: $ y = ({y_1},{y_2},...,{y_k})^{T} \in {\mathbb{R}^k} $
   \item \ascii{权重}: $ W \in {\mathbb{R}^{n \times k}} $   
   \item \ascii{偏置}: $ b \in {\mathbb{R}^k} $   
   \item \ascii{softmax函数}: $ 
softmax {(z_i)} = \tfrac{{{e^{{z_i}}}}}{{\sum\limits_{j = 1}^k {{e^{{z_j}}}} }}  \quad i = 1,2,...,k
$
 \end{itemize}

\subsubsection{softmax函数}

如\refig{softmax}所示,模型先求取线性加权和$z$,然后求取$e^z$,最后再实施归一化操作。

\begin{figure}[H]
\centering
\includegraphics[width=0.8\textwidth]{figures/softmax.png}
\caption{softmax函数}
 \label{fig:softmax}
\end{figure}

\subsubsection{权重与偏置}

权重$W$为一个$n \times k$的二维矩阵。

\[
W = \left( {{W_1},{W_2},...,{W_k}} \right) = \left( {\begin{array}{*{20}{c}}
  {{w_{11}}}& \ldots &{{w_{1k}}} \\ 
   \vdots & \ddots & \vdots  \\ 
  {{w_{n1}}}& \cdots &{{w_{nk}}} 
\end{array}} \right) \in {\mathbb{R}^{n \times k}}
\]

其中,$W_j$是一个长度为$n$的向量。

\[
{W_j} = {\left( {{w_{1j}},{w_{2j}},...,{w_{nj}}} \right)^T} \in {\mathbb{R}^n}, j = 1,2,...,k \\
\]

而偏置$b$是一个长度为$k$的\ascii{\quo{one-hot}}向量。

\[
b = {({b_1},{b_2},...,{b_k})^T} \in {\mathbb{R}^k}
\]

\subsubsection{模型定义}

多分类问题的单层感知器模型,使用\ascii{softmax}激活函数,可以如此定义。

\[\begin{aligned}
  y =  & {h_{W,b}}(x) = softmax (z) = softmax ({W^T}x + b) \\ 
   =  & {\left( {{y_1},{y_2},...,{y_k}} \right)^T} \\ 
   =  & \frac{1}{{\sum\limits_{j = 1}^k {{e^{{z_j}}}} }}{\left( {{e^{{z_1}}},{e^{{z_2}}},...,{e^{{z_k}}}} \right)^T} \\ 
   =  & \frac{1}{{\sum\limits_{j = 1}^k {{e^{W_j^Tx + {b_j}}}} }}{\left( {{e^{W_1^Tx + {b_1}}},{e^{W_2^Tx + {b_2}}},...,{e^{W_k^Tx + {b_k}}}} \right)^T} \ 
\end{aligned} \]

其中,对于任意给定的样本$ (x, y) \in S $,$ z_i $表示$W_i^Tx+b_i$的线性加权和,而$y_i(i=1,2,...,k)$表示将其划归为类$i$的概率。

\[\begin{gathered}
  P\left( {y = i|x;W,b} \right) = \frac{{{e^{W_i^Tx + b_i}}}}{{\sum\limits_{j = 1}^k {{e^{W_j^Tx + b_j}}} }} \hfill \\
  i = 1,2,...,k \hfill \\ 
\end{gathered} \]


\subsubsection{交叉熵函数}

基于样本数据集$ S = \{ ({x^{(i)}},{y^{(i)}});i = 1,2,...,m\} $,交叉熵损失函数可以如此定义。

\[\begin{aligned}
  J(W,b) =  &  - \frac{1}{m}\sum\limits_{i = 1}^m {{y^{(i)}}\log \left( {{{\widehat y}^{(i)}}} \right)}  \\ 
   =  &  - \frac{1}{m}\sum\limits_{i = 1}^m {\sum\limits_{j = 1}^k {y_j^{(i)}\log \left( {\widehat y_j^{(i)}} \right)} }  \\
\end{aligned} \]

\ascii{softmax}多分类问题,就是求取最优解$(W^*,b^*)$,使得

\[W^*,b^* = \mathop {\arg \min }\limits_{W,b} J(W,b)\]

\subsubsection{计算梯度}

对于任意一个样本$ (x,y) \in S $,可以推导出$ J(W,b) $相对于$ W $与$ b $的梯度公式。

\[\begin{aligned}
  {\nabla _W}J\left( {W,b;x,y} \right) =  & \left( {\widehat y - y} \right)x \\ 
  {\nabla _b}J\left( {W,b;{x^{(i)}},{y^{(i)}}} \right) =  & \left( {\widehat y - y} \right) \\ 
\end{aligned} \]


\subsubsection{参数更新}

对于训练样本数据$ S $,根据$W, b$的梯度公式,完成本次迭代的参数更新。

\[\begin{aligned}
  W \leftarrow  & W - \alpha \frac{{\sum\limits_{i = 1}^m {{\nabla _W}J\left( {W,b;{x^{(i)}},{y^{(i)}}} \right)} }}{m} \\ 
  b \leftarrow  & b - \alpha \frac{{\sum\limits_{i = 1}^m {{\nabla _b}J\left( {W,b;{x^{(i)}},{y^{(i)}}} \right)} }}{m} \\ 
\end{aligned} \]

\subsection{定义模型}

接下来,使用\tf{}完成该模型的搭建和训练。需要注意的是,理论上的公式与\tf{}具体实现存在微妙的差异。理论上,公式中的$x$常表示一个样本,但\tf{}中的\code{x}常表示一个\ascii{mini-batch}的样本数据集。因此,使用\tf{}设计网络模型时,需要特别关注各个张量大小的变化是否符合预期。

\subsubsection{输入和标签}

首先,使用\code{tf.placeholder}分别定义训练样本的输入和标签。

\begin{leftbar}
\begin{python}
x = tf.placeholder(tf.float32, [None, 28, 28, 1])
t = tf.placeholder(tf.float32, [None, 10])
\end{python}
\end{leftbar}

\code{tf.placeholder}定义了一个占位的\ascii{OP}。\code{None}表示未确定的样本数目,此处表示\code{batch\_size}的大小;当\code{Session.run}时,将通过\code{feed\_dict}的字典提供一个\ascii{mini-batch}的样本数据集,从而自动推导出\code{tf.placeholder}的大小。

另外,每张图片使用$ 28 \times 28 \times 1 $的三维数据表示(灰度为\ascii{1})。为了简化问题,此处将输入的样本数据扁平化,将其变换为长度为\ascii{784}的一维向量。其中,\ascii{-1}表示\ascii{mini-batch}的样本数目,由运行时自动推演其大小。

\begin{leftbar}
\begin{python}
x = tf.reshape(x, [-1, 784])
\end{python}
\end{leftbar}

\subsubsection{定义变量}

然后,使用\code{tf.Variable}定义模型参数。定义训练参数时,必须指定参数的初始化值;训练参数将根据初始值,自动推演数据的类型,及其大小。

\begin{leftbar}
\begin{python}
w = tf.Variable(tf.zeros([784, 10]))
b = tf.Variable(tf.zeros([10]))
\end{python}
\end{leftbar}

此外,变量在使用之前,必须完成初始化。此处,\code{init\_op}将初始化所有全局的训练参数。

\begin{leftbar}
\begin{python}
init_op = tf.global_variables_initializer()
\end{python}
\end{leftbar}

\subsubsection{模型定义}

接下来,便可以很容易地得到多分类问题的单层感知器模型。

\begin{leftbar}
\begin{python}
y = tf.nn.softmax(tf.matmul(x, w) + b)
\end{python}
\end{leftbar}

如\refig{mnist-linear-sum}所示,首先计算\code{x}与\code{w}的矩阵乘法,让后将\code{b}广播(\ascii{broadcast})到矩阵的每一行相加,最终得到训练参数的线性加权和。

\begin{figure}[H]
\centering
\includegraphics[width=0.8\textwidth]{figures/mnist-linear-sum.png}
\caption{线性加权和}
 \label{fig:mnist-linear-sum}
\end{figure}

如\refig{mnist-softmax}所示,\ascii{softmax}将逐行实施运算,最终,\code{y}的大小为\code{[100, 10]}。

\begin{figure}[H]
\centering
\includegraphics[width=0.8\textwidth]{figures/mnist-softmax.png}
\caption{激活函数:softmax}
 \label{fig:mnist-softmax}
\end{figure}

\subsubsection{损失函数}

对于多分类问题,可以使用交叉熵的损失函数。

\begin{leftbar}
\begin{python}
cross_entropy = -tf.reduce_sum(t * tf.log(y))
\end{python}
\end{leftbar}

如\refig{mnist-cross-entropy}所示,\code{t}和\code{y}的大小都为\code{[100, 10]};特殊地,\code{t}的每一行都是一个\quo{\ascii{one-hot}}向量。

对\code{y}实施\code{tf.log}操作,也将得到一个大小为\code{[100, 10]}的矩阵。然后,\code{t}与\code{tf.log(y)}逐位相乘(并非矩阵相乘),也将得到大小为\code{[100, 10]}的矩阵。最终,\code{tf.reduce\_sum}将矩阵中所有元素相加,得到一个标量(\ascii{scalar})值。

\begin{figure}[H]
\centering
\includegraphics[width=0.8\textwidth]{figures/mnist-cross-entropy.png}
\caption{交叉熵损失函数}
 \label{fig:mnist-cross-entropy}
\end{figure}

\subsubsection{精度}

\code{tf.argmax(y,1)}将按第\ascii{1}个维度计算最大值的索引。既按照$ y_{100 \times 10} $的每一行,计算得到在每一行中最大值的的索引值。因此,\code{tf.argmax(y,1)}将得到大小为\code{[100, 1]}的矩阵,或大小为\ascii{100}的向量。同样地,\code{tf.argmax(t,1)}也是一个大小为\ascii{100}的向量。

然后,使用\code{tf.equal}将它们逐元素(\ascii{element-wise})进行相等性比较,得到大小为\ascii{100}的布尔向量。为了计算精度,先将布尔向量转别为数值向量,最终求取该数值向量的均值。

\begin{leftbar}
\begin{python}
is_correct = tf.equal(tf.argmax(y,1), tf.argmax(t,1))
accuracy = tf.reduce_mean(tf.cast(is_correct, tf.float32))
\end{python}
\end{leftbar}

\subsection{优化算法}

接下来,使用梯度下降算法实现交叉熵损失函数的最小化。其中,\code{learning\_rate}表示学习速率,描述参数更新的快慢和步伐大小,是一个典型的超参。

\begin{leftbar}
\begin{python}
optimizer = tf.train.GradientDescentOptimizer(learning_rate=0.003)
train_step = optimizer.minimize(cross_entropy)
\end{python}
\end{leftbar}

如\refig{mnist-gd}所示,可以将损失函数比作一座山,登山者试图寻找最佳的行动方案达到山谷。登山者站在某个山坡上环顾四周,决定沿梯度的反方向向下走一小步,直到达到局部最优。

当实施梯度下降更新算法时,初始点不同,获得的最小值也不同,因此梯度下降求得的只是局部最小值。另外,越接近最小值时,下降速度越慢。下降的步伐大小也非常重要,如果太小,则找到函数最小值的速度就很慢;如果太大,则可能会越过极值点。

\begin{figure}[H]
\centering
\includegraphics[width=0.8\textwidth]{figures/mnist-gd.jpeg}
\caption{梯度下降算法}
 \label{fig:mnist-gd}
\end{figure}

\subsection{训练模型}

在此之前,\tf{}仅构造计算图,并没有启动计算图的执行。接下来,客户端创建一个会话,建立与本地或远端计算设备集的通道,启动计算图的执行过程。

首先,完成训练参数的初始化。通过运行模型参数的初始化子图,并发地执行各个训练参数的初始化器,将初始值就地修改到相应的训练参数内。

\begin{leftbar}
\begin{python}
with tf.Session() as sess:
  sess.run(init_op)
\end{python}
\end{leftbar}

然后,开始迭代地执行\code{train\_step},完成模型的一次迭代训练。其中,每\ascii{100}次迭代,计算当前模型在训练数据集及测试数据集的精度和损失。

\begin{leftbar}
\begin{python}
with tf.Session() as sess:
  for step in range(1000):
    batch_xs, batch_ys = mnist.train.next_batch(100)        
    sess.run(train_step, feed_dict={x: batch_xs, t: batch_ys})
    
    if step % 100 == 0:
      acc, loss = sess.run([accuracy, cross_entropy], 
        feed_dict={x: batch_xs, t: batch_ys})
      acc, loss = sess.run([accuracy, cross_entropy], 
        feed_dict={x: mnist.test.images, t: mnist.test.labels}) 
\end{python}
\end{leftbar}

据统计,经过\ascii{1000}次迭代,可得到大约\percent{92}的精度。

\begin{figure}[H]
\centering
\includegraphics[width=0.8\textwidth]{figures/mnist-slp-accuracy.png}
\caption{可视化:单层感知器,运行1000次step}
 \label{fig:mnist-slp-accuracy}
\end{figure}

\end{content}

\section{多层感知器}

\begin{content}

为了进一步提高精度,接下来尝试搭建\ascii{5}层的多层感知器模型。

\begin{figure}[H]
\centering
\includegraphics[width=0.8\textwidth]{figures/mnist-5-layer.png}
\caption{5层感知器}
 \label{fig:mnist-5-layer}
\end{figure}

\subsection{理论基础}

\subsubsection{符号定义}

为了形式化地描述多层感知器模型,此处定义了一些常用符号。

\begin{itemize}
   \item \alert{$ {n_{\ell}} $}: 网络层数,其中第$0$层为输入层,第$n_{\ell}$层为输出层
   \item \alert{$ {s_{\ell}} $}: 第$\ell$层的节点数,$ \ell = 0, 1, ..., n_{\ell} $
   \item \alert{$ w_{ji}^{(\ell)} $}: 第$(\ell-1)$层节点$i$与第$\ell$层节点$j$之间的权重,$ \ell = 1, ..., n_{\ell} $
   \item \alert{$ b_i^{(\ell)} $}: 第$\ell$层节点$i$的偏置项,$ \ell = 1, ..., n_{\ell} $
   \item \alert{$ a_i^{(\ell)} $}: 第$\ell$层节点$i$的输出,$ \ell = 1, ..., n_{\ell}, x = a^{(0)}, y = a^{(n_{\ell})} $
   \item \alert{$ z_i^{(\ell)} $}: 第$\ell$层节点$i$的权重和,$ \ell = 1, ..., n_{\ell} $
   \item \alert{$ \delta _i^{(\ell)} $}: 第$\ell$层节点$i$的误差项,$ \ell = 1, ..., n_{\ell} $
   \item \alert{$ S = \{ ({x^{(t)}},{y^{(t)}});t = 1,2,...,m\} $}: 样本空间
 \end{itemize}

\subsubsection{前向传播}

$z^{(\ell )}$表示$\ell$层的线性加权和,它由第$\ell - 1$层的输出$a^{(\ell  - 1)}$与第$\ell$层的权重矩阵$w^{(\ell )}$相乘,再加上第$\ell$层的偏置向量所得。

推而广之,第$\ell$层的输出,由激活函数$f({z^{(\ell )}})$所得。其中,$a^{(0)} = x, y = {a^{({n_\ell })}}$。

\[\begin{gathered}
  {z^{(\ell )}} = {w^{(\ell )}}{a^{(\ell  - 1)}} + {b^{(\ell )}} \hfill \\
  {a^{(\ell )}} = f({z^{(\ell )}}) \hfill \\
  {a^{(0)}} = x \hfill \\
  y = {a^{({n_\ell })}} \hfill \\ 
\end{gathered} \]

\subsubsection{后向传播}

然后,反向计算各层的误差。其中,第$\ell$层的误差,由$\ell + 1$层的误差计算所得。特殊地,在输出层,预测值$a^{({n_\ell })}$与$y$之间的误差,可以直接计算得到。

\[{\delta ^{(\ell)}} = \left\{ \begin{gathered}
  {({w^{(\ell + 1)}})^T}{\delta ^{(\ell + 1)}} \circ f\,'({z^{(\ell)}});{\text{  }}\ell \ne {n_\ell} \hfill \\
  ({a^{(\ell)}} - y) \circ f\,'({z^{(\ell)}}); {\text{  }}\ell = {n_\ell} \hfill \\ 
\end{gathered}  \right.\]

损失函数$J(w,b)$相对于各层的权重矩阵与偏置向量的梯度便可以计算得到。

\[\begin{gathered}
  {\nabla _{{w^{(\ell )}}}}J(w,b;x,y) = {\delta ^{(\ell )}}{\left( {{a^{(\ell  - 1)}}} \right)^T} \hfill \\
  {\nabla _{{b^{(\ell )}}}}J(w,b;x,y) = {\delta ^{(\ell )}} \hfill \\
  \ell  = 1,2,...,{n_\ell } \hfill \\ 
\end{gathered} \]

一般地,在实际系统实现中,下游层传递梯度给上层,上层直接完成梯度的计算。

\subsubsection{参数更新}

对于给定样本数据集$ S = \{ ({x^{(t)}},{y^{(t)}});t = 1,2,...,m\} $,根据梯度反传公式,可以计算得到参数更新的变化量。

\[\begin{aligned}
  \Delta {w^{(\ell )}} \leftarrow \Delta {w^{(\ell )}} + {\nabla _{{w^{(\ell )}}}}J\left( {w,b;{x^{(t)}},{y^{(t)}}} \right) \\ 
  \Delta {b^{(\ell )}} \leftarrow \Delta {b^{(\ell )}} + {\nabla _{{b^{(\ell )}}}}J\left( {w,b;{x^{(t)}},{y^{(t)}}} \right) \\ 
  t = 1,2,...,m;\ell  = 1,2,...,{n_\ell } \\ 
\end{aligned} \]

最后,执行梯度下降算法,完成训练参数一个迭代的更新。

\[\begin{aligned}
  {w^{(\ell )}} \leftarrow  & {w^{(\ell )}} - \alpha \left( {\frac{{\Delta {w^{(\ell )}}}}{m}} \right) \\ 
  {b^{(\ell )}} \leftarrow  & {b^{(\ell )}} - \alpha \frac{{\Delta {b^{(\ell )}}}}{m} \\ 
  \ell  = & 1,2,...,{n_\ell }  \\
\end{aligned} \]

\subsection{定义模型}

相对于上一节中尝试的单层感知器,此处定义每一个隐式层的权重时,并没有使用常量定义变量的初始值,而使用满足某种数据分布特征的随机值。

\begin{leftbar}
\begin{python}
K = 200
L = 100
M = 60
N = 30

w1 = tf.Variable(tf.truncated_normal([28*28, K] ,stddev=0.1)) 
b1 = tf.Variable(tf.zeros([K]))

w2 = tf.Variable(tf.truncated_normal([K, L], stddev=0.1))
b2 = tf.Variable(tf.zeros([L]))

w3 = tf.Variable(tf.truncated_normal([L, M], stddev=0.1)) 
b3 = tf.Variable(tf.zeros([M]))

w4 = tf.Variable(tf.truncated_normal([M, N], stddev=0.1)) 
b4 = tf.Variable(tf.zeros([N]))

w5 = tf.Variable(tf.truncated_normal([N, 10], stddev=0.1)) 
b5 = tf.Variable(tf.zeros([10]))
\end{python}
\end{leftbar}

在定义每一个隐式层时,采用\ascii{sigmoid}的激活函数。而在最后的输出层,采用\ascii{softmax}的激活函数。

\begin{leftbar}
\begin{python}
y1 = tf.nn.sigmoid(tf.matmul(x,  w1) + b1)
y2 = tf.nn.sigmoid(tf.matmul(y1, w2) + b2)
y3 = tf.nn.sigmoid(tf.matmul(y2, w3) + b3)
y4 = tf.nn.sigmoid(tf.matmul(y3, w4) + b4)
y  = tf.nn.softmax(tf.matmul(y4, w5) + b5)
\end{python}
\end{leftbar}

经过迭代的模型训练,可以得到大约\percent{97}左右的精度。但是,网络随着层次的增加,模型变得越来越难以收敛。接下来,尝试一些常见的调优技术,改善网络的性能。

\subsection{优化技术}

\subsubsection{激活函数:ReLU}

在深度模型中,不适合使用\ascii{sigmoid}激活函数。它将把所有的值都挤到了\ascii{0}到\ascii{1}之间;随着网络层次的增加,引发梯度消失的问题。

\begin{figure}[H]
\centering
\includegraphics[width=0.8\textwidth]{figures/mnist-relu.png}
\caption{ReLU激活函数}
 \label{fig:mnist-relu}
\end{figure}

可以使用\ascii{ReLU(Rectified Linear Unit)}替代\ascii{sigmoid},不仅避免了\ascii{sigmoid}导致的一些问题,而且能够加快初始的收敛速度。

\begin{leftbar}
\begin{python}
y1 = tf.nn.relu(tf.matmul(x,  w1) + b1)
y2 = tf.nn.relu(tf.matmul(y1, w2) + b2)
y3 = tf.nn.relu(tf.matmul(y2, w3) + b3)
y4 = tf.nn.relu(tf.matmul(y3, w4) + b4)
y  = tf.nn.softmax(tf.matmul(y4, w5) + b5)
\end{python}
\end{leftbar}

另外,如果使用\ascii{ReLU}激活函数,偏置向量常常初始化为小的正值,使得神经元在一开始就会工作在\ascii{ReLU}的非零区域内。

\begin{leftbar}
\begin{python}
K = 200
L = 100
M = 60
N = 30

w1 = tf.Variable(tf.truncated_normal([28*28, K] ,stddev=0.1)) 
b1 = tf.Variable(tf.ones([L])/10)

w2 = tf.Variable(tf.truncated_normal([K, L], stddev=0.1))
b2 = tf.Variable(tf.ones([L])/10)

w3 = tf.Variable(tf.truncated_normal([L, M], stddev=0.1)) 
b3 = tf.Variable(tf.ones([L])/10)

w4 = tf.Variable(tf.truncated_normal([M, N], stddev=0.1)) 
b4 = tf.Variable(tf.ones([L])/10)

w5 = tf.Variable(tf.truncated_normal([N, 10], stddev=0.1)) 
b5 = tf.Variable(tf.ones([L])/10)
\end{python}
\end{leftbar}

如\refig{mnist-sigmoid-to-relu}所示,前\ascii{300}次迭代,使用\ascii{ReLU}相对于使用\ascii{sigmoid},其初始收敛速度提升显著。

\begin{figure}[H]
\centering
\includegraphics[width=0.8\textwidth]{figures/mnist-sigmoid-to-relu.png}
\caption{应用ReLU激活函数:初始收敛速度提升显著}
 \label{fig:mnist-sigmoid-to-relu}
\end{figure}

\subsubsection{不定值}

为了得到稳定的数值计算结果,避免出现精度突降为\ascii{0}的情况发生。追溯实现代码,可能引入\code{log(0)}计算得到\code{NaN}不定值的问题。可以使用\code{softmax\_cross\_entropy\_with\_logits}计算交叉熵损失,并将线性加权和作为其输入(常称为\ascii{logits})。

\begin{leftbar}
\begin{python}
logits = tf.matmul(y4, w5) + b5
y = tf.nn.softmax(logits)

cross_entropy = tf.nn.softmax_cross_entropy_with_logits(
  logits=logits, labels=t)
\end{python}
\end{leftbar}

\subsubsection{学习速率衰减}

随着网路层次的增加,及其应用相关优化技术后,模型的精度能够能得到\percent{98}左右,但很难得到一个稳定的精度。如\refig{mnist-lr-too-larger}所示,精度和损失抖动相当明显。

\begin{figure}[H]
\centering
\includegraphics[width=0.8\textwidth]{figures/mnist-lr-too-larger.png}
\caption{噪声抖动:学习速率过大}
 \label{fig:mnist-lr-too-larger}
\end{figure}

可以采用更好的优化算法,例如\code{AdamOptimizer}。随着迭代过程的次数,学习速率将指数级衰减,在模型训练后期可以得到一个更稳定的精度和损失曲线。

\begin{leftbar}
\begin{python}
lr = tf.placeholder(tf.float32)
train_step = tf.train.AdamOptimizer(lr).minimize(cross_entropy)
\end{python}
\end{leftbar}

在每个迭代训练过程中,根据当前\code{step}的值,实时计算当前迭代的学习速率\code{lr},然后通过\code{feed\_dict}传递给\code{Session.run}执行。其中,学习速率衰减方程如下代码所示,随着迭代次数的增加,学习速率指数衰减。

\begin{leftbar}
\begin{python}
def lr(step):
  max_lr, min_lr, decay_speed = 0.003, 0.0001, 2000.0
  return min_lr + (max_lr - min_lr) * math.exp(-step/decay_speed)

with tf.Session() as sess:
  for step in range(10000):
    batch_xs, batch_ys = mnist.train.next_batch(100)
    sess.run(train_step, 
      feed_dict={x: batch_xs, t: batch_ys, pkeep: 0.75, lr: lr(step)})
\end{python}
\end{leftbar}

如\refig{mnist-apply-learning-rate-decay}所示,应用学习速率衰减方法后,可以得到一个更稳定的精度和损失曲线。

\begin{figure}[H]
\centering
\includegraphics[width=0.8\textwidth]{figures/mnist-apply-learning-rate-decay.png}
\caption{应用Adam优化算法后,精度和损失趋于稳定}
 \label{fig:mnist-apply-learning-rate-decay}
\end{figure}

\subsubsection{应用Dropout}

但是,损失曲线在训练集与测试集上相分离,出现明显的过拟合现象。即模型在训练数据集上表现良好,但在测试数据集上出现反弹,模型缺乏足够的泛化能力。

\begin{figure}[H]
\centering
\includegraphics[width=0.8\textwidth]{figures/mnist-overfitting.png}
\caption{过拟合}
 \label{fig:mnist-overfitting}
\end{figure}

如\refig{mnist-dropout}所示,在训练时对隐藏层的输出实施\ascii{dropout}操作,以\code{1 - pkeep}的概率随机丢弃神经元的输出,并在反向传播梯度时不再更新相应的权重。而在推理时恢复所有神经元的输出,间接改善了网络的泛化能力。

\begin{figure}[H]
\centering
\includegraphics[width=0.8\textwidth]{figures/mnist-dropout.png}
\caption{Dropout方法}
 \label{fig:mnist-dropout}
\end{figure}

使用\tf{}实现\ascii{dropout}操作时,先定义一个超级参数\code{pkeep},表示隐藏层的神经元以概率\code{pkeep}随机保留,以概率\code{1 - pkeep}随机丢弃。

\begin{leftbar}
\begin{python}
pkeep = tf.placeholder(tf.float32)

y1 = tf.nn.relu(tf.matmul(x,  w1) + b1)
y1d = tf.nn.dropout(y1, pkeep)

y2 = tf.nn.relu(tf.matmul(y1d, w2) + b2)
y2d = tf.nn.dropout(y2, pkeep)

y3 = tf.nn.relu(tf.matmul(y2d, w3) + b3)
y3d = tf.nn.dropout(y3, pkeep)

y4 = tf.nn.relu(tf.matmul(y3d, w4) + b4)
y4d = tf.nn.dropout(y4, pkeep)

logits = tf.matmul(y4d, w5) + b5
y = tf.nn.softmax(Ylogits)
\end{python}
\end{leftbar}

在训练时,置超参\code{pkeep}的值小于\ascii{1};而在推理时,置超参\code{pkeep}的值为\ascii{1}。

\begin{leftbar}
\begin{python}
with tf.Session() as sess:
  for step in range(10000):
    batch_xs, batch_ys = mnist.train.next_batch(100)
    sess.run(train_step, 
      feed_dict={x: batch_xs, t: batch_ys, pkeep: 0.75, lr: lr(step)})

    if step % 100 == 0:
      acc, loss = sess.run([accuracy, cross_entropy], 
        feed_dict={x: batch_xs, t: batch_ys, pkeep: 1})
      acc, loss = sess.run([accuracy, cross_entropy], 
        feed_dict={x: mnist.test.images, t: mnist.test.labels, pkeep: 1})
\end{python}
\end{leftbar}

在每一隐藏层实施\ascii{dropout}操作之后,训练集与测试集的损失曲线再次相交。但是,精度和损失曲线相比又出现小幅的抖动,而且训练集与测试集的损失曲线相重合的程度不是很理想,过拟合问题依然很突出。

也就是说,过拟合问题存在其他更深刻的原因。例如,将$ 28 \times 28 $的图片实施扁平化操作,将其变换为一个长度为\ascii{784}的一维向量,这将完全丢失了像素的空间排列信息。

接下来,通过尝试构造卷积神经网络,从原图像中提取特征,从而保留了像素的空间排列信息,进而提升网络的性能。

\begin{figure}[H]
\centering
\includegraphics[width=0.8\textwidth]{figures/mnist-apply-dropout-result.png}
\caption{实施Dropout后,训练集与测试集的损失曲线再次重合}
 \label{fig:mnist-apply-dropout-result}
\end{figure}

\end{content}

\section{卷积网络}

\begin{content}

\subsection{特征与优势}

随着网络层次增加,全连接网络的梯度消失的问题将越发突出,收敛速度变得越来越慢。相对于全连接网络,卷积网络具有\ascii{3}个主要特征,它减少了网络参数的数量,提升网络的泛化能力。

\subsubsection{局部连接}

相对于全连接网络,卷积网络实现了局部连接,即每个神经元并不与上一层的神经元都存在连接。如\refig{mnist-conv-local-conn}左侧所示,假如存在一张$ 1000 \times 1000 $像素的图像,及其$ 10^6 $个隐藏层的神经元。在全连接网络里,将拥有$ 10^3 \times 10^3 \times 10^6 = 10^{12} $个训练参数。

事实上,每个神经元没有必要与上一层神经元都存在连接。如\refig{mnist-conv-local-conn}右侧所示,假如每个隐藏层的神经元仅与上一层$ 10 \times 10 $的局部图像存在连接,$ 10^6 $个隐藏层的神经元则需要$ 10^6 \times 10^2 = 10^8$个网络连接,相比减少了\ascii{4}个数量级。

\begin{figure}[H]
\centering
\includegraphics[width=0.9\textwidth]{figures/mnist-conv-local-conn.png}
\caption{局部连接}
 \label{fig:mnist-conv-local-conn}
\end{figure}

\subsubsection{权值共享}

为了进一步减少网络连接,卷积网络还实现了权重共享;即每一组连接共享相同的权重,而不是每个连接存在不同的权重。如\refig{mnist-conv-local-conn-2}右侧所示,每个隐藏层的神经元仅与$ 10 \times 10 $的局部图像存在连接,且共享$ 10 \times 10 $的权重矩阵,与隐藏层的神经元的数目无关。相对于如\refig{mnist-conv-local-conn-2}左侧的局部连接网络,需要$10^8$个参数,卷积层仅需$10^2$个参数。

如\refig{mnist-conv-local-conn-2}右侧所示,为了提取不同特征,例如不同边缘的图像特征,可以使用多个过滤器。例如,存在\ascii{100}个过滤器,则需要$10^4$个参数。

\begin{figure}[H]
\centering
\includegraphics[width=0.9\textwidth]{figures/mnist-conv-local-conn-2.png}
\caption{权值共享,多个过滤器}
 \label{fig:mnist-conv-local-conn-2}
\end{figure}

\subsubsection{下采样}

如\refig{mnist-subsample}所示,可选地实施下采样,进一步减少网络的参数,提升模型的鲁棒性。

\begin{figure}[H]
\centering
\includegraphics[width=0.9\textwidth]{figures/mnist-subsample.png}
\caption{下采样}
 \label{fig:mnist-subsample}
\end{figure}

\subsection{卷积运算}

卷积运算是一个计算密集型的\ascii{OP}。如\refig{mnist-conv2d-gif}所示,存在一个权重向量\code{w[3,3,3,2]},其中输入通道数为\ascii{3},输出通道数为\ascii{2},卷积核大小为$3 \times 3$。

显而易见,输入图像的通道数,等价于卷积核的深度;卷积核的数目,等于\ascii{Feature Map}的输出通道数。另外,为了采集图像的边缘特征,在原图像外围补了(\ascii{padding})一圈零值。每次卷积计算,移动的步长(\ascii{stride})为\ascii{2}。因此,最终输出的\ascii{Feature Map}大小为$3 \times 3$。

\begin{figure}[H]
\centering
\includegraphics[width=0.8\textwidth]{figures/mnist-conv2d-gif.png}
\caption{卷积运算}
 \label{fig:mnist-conv2d-gif}
\end{figure}

\subsubsection{例子}

假如存在一个$32 \times 32 \times 3$的图片,卷积核大小为$5 \times 5 \times 3$。其中,卷积核的深度等于图片的输入通道数。如\refig{mnist-conv-1dot}所示,卷积核与图片中大小为$5 \times 5 \times 3$的块实施点积运算,得到一个值。

\begin{figure}[H]
\centering
\includegraphics[width=0.8\textwidth]{figures/convolutional-layer-2.png}
\caption{卷积运算:卷积核与图片块的点积运算}
 \label{fig:mnist-conv-1dot}
\end{figure}

如\refig{mnist-conv-ndot}所示,卷积核遍历整个图片空间,最终得到一个大小为$28 \times 28 \times 1$的\ascii{Feature Map}。

\begin{figure}[H]
\centering
\includegraphics[width=0.8\textwidth]{figures/convolutional-layer-3.png}
\caption{卷积运算:卷积核遍历图片,步长为1}
 \label{fig:mnist-conv-ndot}
\end{figure}

如\refig{mnist-conv-multi-filters}所示,如果存在多个卷积核,则得到多个\ascii{Feature Map}。

\begin{figure}[H]
\centering
\includegraphics[width=0.8\textwidth]{figures/convolutional-layer-4.png}
\caption{卷积运算:多个卷积核}
 \label{fig:mnist-conv-multi-filters}
\end{figure}

\subsection{公式推导}

\subsubsection{前向传播}

$Z^{(\ell )}$表示$\ell$层的线性加权和,它由第$\ell - 1$层的输出$A^{(\ell  - 1)}$与第$\ell$层的权重矩阵$W^{(\ell )}$卷积,再加上第$\ell$层的偏置向量所得。

推而广之,第$\ell$层的输出,由激活函数$f({Z^{(\ell )}})$所得。其中,$A^{(0)} = x, y = {A^{({n_\ell })}}$。

\[\begin{gathered}
  {Z^{(\ell )}} = {A^{(\ell  - 1)}} * {W^{(\ell )}} + {b^{(\ell )}} \hfill \\
  {A^{(\ell )}} = f\left( {{Z^{(\ell )}}} \right) \hfill \\ 
\end{gathered} \]

\subsubsection{后向传播}

然后,反向计算各层的误差。其中,第$\ell$层的误差,由$\ell + 1$层的误差计算所得。相对于全连接网络,此处运用的是卷积运算,并非矩阵乘法运算。

\[
{\delta ^{(\ell )}} = {\delta ^{(\ell  + 1)}} * {W^{(\ell  + 1)}} \circ f\,'\left( {{z^{(\ell )}}} \right)
\]

损失函数$J(w,b)$相对于各层的权重矩阵与偏置向量的梯度便可以计算得到。

\[\begin{aligned}
  {\nabla _{{W^{(\ell )}}}}J(W,b) =  & {A^{(\ell  - 1)}} * {\delta ^{(\ell )}} \\ 
  {\nabla _{{b^{(\ell )}}}}J(W,b) =  & {\delta ^{(\ell )}} \\ 
\end{aligned} \]

\subsection{实现卷积网络}

实现卷积网络时,首先需要定义每层过滤器的权重矩阵,用于提取图像的特征。权重矩阵在图像里表现的像一个从原始图像矩阵中提取特定信息的过滤器。一个权重矩阵可能用来提取图像边缘信息,一个权重矩阵可能是用来提取一个特定颜色,另一个权重矩阵可能对不需要的噪声进行模糊化。

当存在多个卷积层时,初始层往往提取较多的一般特征;随着网络结构变得更深,权值矩阵提取的特征越来越复杂,并且越来越适用于眼前的具体问题。

一般地,过滤器常使用一个\ascii{4}维的张量表示,前两维表示过滤器的大小,第三维表示输入的通道数,第四维表示输出的通道数。如\refig{mnist-filter}所示。

\begin{figure}[H]
\centering
\includegraphics[width=0.4\textwidth]{figures/mnist-filter.png}
\caption{卷积层的过滤器}
 \label{fig:mnist-filter}
\end{figure}

如\refig{mnist-conv2d-1}所示,构造了\ascii{3}个卷积层和\ascii{2}个全连接层。其中,中间隐藏层使用\ascii{ReLU}的激活函数,最后的输出层采用\ascii{softmax}的激活函数。

\begin{figure}[H]
\centering
\includegraphics[width=0.9\textwidth]{figures/mnist-conv2d-1.png}
\caption{实现卷积神经网络}
 \label{fig:mnist-conv2d-1}
\end{figure}

使用\tf{}实现卷积网络,如下代码所示。

\begin{leftbar}
\begin{python}
K = 4 
L = 8
M = 12
N = 200

w1 = tf.Variable(tf.truncated_normal([5, 5, 1, K], stddev=0.1))
b1 = tf.Variable(tf.ones([K])/10)

w2 = tf.Variable(tf.truncated_normal([5, 5, K, L], stddev=0.1))
b2 = tf.Variable(tf.ones([L])/10)

w3 = tf.Variable(tf.truncated_normal([4, 4, L, M], stddev=0.1))
b3 = tf.Variable(tf.ones([M])/10)

w4 = tf.Variable(tf.truncated_normal([7 * 7 * M, N], stddev=0.1))
b4 = tf.Variable(tf.ones([N])/10)

w5 = tf.Variable(tf.truncated_normal([N, 10], stddev=0.1))
b5 = tf.Variable(tf.ones([10])/10)

y1 = tf.nn.relu(tf.nn.conv2d(
       x,  w1, strides=[1, 1, 1, 1], padding='SAME') + b1)
y2 = tf.nn.relu(tf.nn.conv2d(
       y1, w2, strides=[1, 2, 2, 1], padding='SAME') + b2)
y3 = tf.nn.relu(tf.nn.conv2d(
       y2, w3, strides=[1, 2, 2, 1], padding='SAME') + b3)

yy = tf.reshape(Y3, shape=[-1, 7 * 7 * M])
y4 = tf.nn.relu(tf.matmul(yy, w4) + b4)

logits = tf.matmul(y4, w5) + b5
y = tf.nn.softmax(logits)
\end{python}
\end{leftbar}

如\refig{mnist-conv2d-1-result}所示,经过$10^4$次训练,可以得到大约\percent{98.9}的准确率。

\begin{figure}[H]
\centering
\includegraphics[width=0.9\textwidth]{figures/mnist-conv2d-1-result.png}
\caption{实现卷积网络:可以得到\percent{98.9}的准确率}
 \label{fig:mnist-conv2d-1-result}
\end{figure}

\subsection{增强卷积网络}

如\refig{mnist-conv2d-2}所示,保留之前的网络层次结构,构造了\ascii{3}个卷积层和\ascii{2}个全连接层。其中,中间隐藏层使用\ascii{ReLU}的激活函数,最后的输出层采用\ascii{softmax}的激活函数。

但是,相对于之前的卷积网络,使用了更多的通道提取更多的特征。同时,在全连接的隐藏层中实施\ascii{dropout}操作,增强网络的泛化能力。

\begin{figure}[H]
\centering
\includegraphics[width=0.9\textwidth]{figures/mnist-conv2d-2.png}
\caption{改善卷积神经网络}
 \label{fig:mnist-conv2d-2}
\end{figure}

使用\tf{}实现更大的卷积网络,如下代码所示。

\begin{leftbar}
\begin{python}
K = 6
L = 12
M = 24
N = 200

w1 = tf.Variable(tf.truncated_normal([6, 6, 1, K], stddev=0.1))
b1 = tf.Variable(tf.ones([K])/10)

w2 = tf.Variable(tf.truncated_normal([5, 5, K, L], stddev=0.1))
b2 = tf.Variable(tf.ones([L])/10)

w3 = tf.Variable(tf.truncated_normal([4, 4, L, M], stddev=0.1))
b3 = tf.Variable(tf.ones([M])/10)

w4 = tf.Variable(tf.truncated_normal([7 * 7 * M, N], stddev=0.1))
b4 = tf.Variable(tf.ones([N])/10)

w5 = tf.Variable(tf.truncated_normal([N, 10], stddev=0.1))
b5 = tf.Variable(tf.ones([10])/10)

y1 = tf.nn.relu(tf.nn.conv2d(
       x,  w1, strides=[1, 1, 1, 1], padding='SAME') + b1)
y2 = tf.nn.relu(tf.nn.conv2d(
       y1, w2, strides=[1, 2, 2, 1], padding='SAME') + b2)
y3 = tf.nn.relu(tf.nn.conv2d(
       y2, w3, strides=[1, 2, 2, 1], padding='SAME') + b3)

yy = tf.reshape(Y3, shape=[-1, 7 * 7 * M])
y4 = tf.nn.relu(tf.matmul(yy, w4) + b4)
y4d = tf.nn.dropout(y4, pkeep)

logits = tf.matmul(y4d, w5) + b5
y = tf.nn.softmax(logits)
\end{python}
\end{leftbar}

如\refig{mnist-conv2d-2-result}所示,经过$10^4$次训练,可以得到大约\percent{99.3}的准确率。

\begin{figure}[H]
\centering
\includegraphics[width=0.9\textwidth]{figures/mnist-conv2d-2-result.png}
\caption{增强卷积网络:可以得到\percent{99.3}的准确率}
 \label{fig:mnist-conv2d-2-result}
\end{figure}

同时,相对于之前实现的卷积网络,过拟合问题得到了明显地改善,如\refig{mnist-conv2d-3-result}所示。

\begin{figure}[H]
\centering
\includegraphics[width=0.9\textwidth]{figures/mnist-conv2d-3-result.png}
\caption{增强卷积网络:过拟合问题明显改善}
 \label{fig:mnist-conv2d-3-result}
\end{figure}

\end{content}

\part{系统架构}
\begin{savequote}[45mm]
\ascii{Any fool can write code that a computer can understand. Good programmers write code that humans can understand.}
\qauthor{\ascii{- Martin Flower}}
\end{savequote}

\chapter{系统架构} 
\label{ch:architecture}

\begin{content}

本章将阐述\tf{}的系统架构,并一个简单的例子,讲述图结构的变换过程。最后,通过挖掘会话管理的工作机制,加深理解\tf{}运行时的工作机理。

\end{content}

\section{系统架构}
	
\begin{content}

如\refig{tf-architecture}所示,\tf{}的系统结构以\ascii{C API}为界,将整个系统分为\emph{前端}和\emph{后端}两个子系统\footnote{事实上,后端系统中也存在\ascii{Client}的代码,前端系统是\tf{}对外的编程接口。在后面的章节,将详细地讨论这个问题。}。

\begin{enum}
  \eitem{前端系统:提供编程模型,负责构造计算图;}
  \eitem{后端系统:提供运行时环境,负责执行计算图。} 
\end{enum}

\tf{}的系统设计遵循良好的分层架构,后端系统的设计和实现可以进一步分解为\ascii{4}层。

\begin{enum}
  \eitem{运行时:分别提供本地模式和分布式模式,并共享大部分设计和实现;}
  \eitem{计算层:由各个\ascii{OP}的\ascii{Kernel}实现组成;在运行时,\ascii{Kernel}实现执行\ascii{OP}的具体数学运算;} 
  \eitem{通信层:基于\ascii{gRPC}实现组件间的数据交换,并能够在支持\ascii{IB}网络的节点间实现\ascii{RDMA}通信;}
  \eitem{设备层:计算设备是\ascii{OP}执行的主要载体,\tf{}支持多种异构的计算设备类型。}
\end{enum}

从图操作的角度看待系统行为,\tf{}运行时就是完成计算图的构造、编排、及其运行。

\begin{enum}
  \eitem{表达图:构造计算图,但不执行图;}
  \eitem{编排图:将计算图的节点以最佳的执行方案部署在集群中各个计算设备上执行;} 
  \eitem{运行图:按照拓扑排序执行图中的节点,并启动每个\ascii{OP}的\ascii{Kernel}计算。}   
\end{enum}

\begin{figure}[H]
\centering
\includegraphics[width=1.0\textwidth]{figures/tf-architecture.png}
\caption{TensorFlow系统架构}
 \label{fig:tf-architecture}
\end{figure}

\subsection{Client}

\ascii{Client}是前端系统的主要组成部分,它是一个支持多语言的编程环境。\ascii{Client}基于\ascii{TensorFlow}的编程接口,构造计算图。目前,\ascii{TensorFlow}支持\ascii{Python}和\ascii{C++}的编程接口较为完善,尤其对\ascii{Python}的\ascii{API}支持最为全面。并且,对其他编程语言的\ascii{API}支持日益完善。

此时,\ascii{TensorFlow}并未执行任何的图计算,直至与后台计算引擎建立\ascii{Session},并以\ascii{Session}为桥梁,建立\ascii{Client}与\ascii{Master}之间的通道,并将\ascii{Protobuf}格式的\ascii{GraphDef}序列化后传递给\ascii{Master},启动计算图的执行过程。

\subsection{Master}

在分布式的运行时环境中,\ascii{Client}执行\code{Session.run}时,传递整个计算图给后端的\ascii{Master}。此时,计算图是完整的,常称为\emph{\ascii{Full Graph}}。随后,\ascii{Master}根据\code{Session.run}传递给它的\code{fetches, feeds}参数列表,反向遍历\ascii{Full Graph},并按照依赖关系,对其实施剪枝,最终计算得到最小的依赖子图,常称为\ascii{Client Graph}。

接着,\ascii{Master}负责将\ascii{Client Graph}按照任务的名称分裂(\code{SplitByTask})为多个\ascii{Graph Partition};其中,每个\ascii{Worker}对应一个\ascii{Graph Partition}。随后,\ascii{Master}将\ascii{Graph Partition}分别注册到相应的\ascii{Worker}上,以便在不同的\ascii{Worker}上并发执行这些\ascii{Graph Partition}。最后,\ascii{Master}将通知所有\ascii{Work}启动相应\ascii{Graph Partition}的执行过程。

其中,\ascii{Work}之间可能存在数据依赖关系,\ascii{Master}并不参与两者之间的数据交换,它们两两之间互相通信,独立地完成交换数据,直至完成所有计算。

\subsection{Worker}

对于每一个任务,\tf{}都将启动一个\ascii{Worker}实例。\ascii{Worker}主要负责如下\ascii{3}个方面的职责:

\begin{enum}
  \eitem{处理来自\ascii{Master}的请求;}
  \eitem{对注册的\ascii{Graph Partition}按照本地计算设备集实施二次分裂(\code{SplitByDevice}),并通知各个计算设备并发执行各个\ascii{Graph Partition};}
  \eitem{按照拓扑排序算法在某个计算设备上执行本地子图,并调度\ascii{OP}的\ascii{Kernel}实现;} 
  \eitem{协同任务之间的数据通信。}
\end{enum}

首先,\ascii{Worker}收到\ascii{Master}发送过来的图执行命令,此时的计算图相对于\ascii{Worker}是完整的,也称为\ascii{Full Graph},它对应于\ascii{Master}的一个\ascii{Graph Partition}。随后,\ascii{Worker}根据当前可用的硬件环境,包括\ascii{(GPU/CPU)}资源,按照\ascii{OP}设备的约束规范,再将图分裂\code{(SplitByDevice)}为多个\ascii{Graph Partition};其中,每个计算设备对应一个\ascii{Graph Partition}。接着,\ascii{Worker}启动所有的\ascii{Graph Partition}的执行。最后,对于每一个计算设备,\ascii{Worker}将按照计算图中节点之间的依赖关系执行拓扑排序算法,并依次调用\ascii{OP}的\ascii{Kernel}实现,完成\ascii{OP}的运算(一种典型的多态实现技术)。

其中,\ascii{Worker}还要负责将\ascii{OP}运算的结果发送到其他的\ascii{Worker}上去,或者接受来自其他\ascii{Worker}发送给它的运算结果,以便实现\ascii{Worker}之间的数据交互。\tf{}特化实现了源设备和目标设备间的\ascii{Send/Recv}。

\begin{enum}
  \eitem{本地\ascii{CPU}与\ascii{GPU}之间,使用\code{cudaMemcpyAsync}实现异步拷贝;}
  \eitem{本地\ascii{GPU}之间,使用端到端的\ascii{DMA}操作,避免主机端\ascii{CPU}的拷贝。} 
\end{enum}

对于任务间的通信,\tf{}支持多种通信协议。

\begin{enum}
  \eitem{\ascii{gRPC over TCP;}}
  \eitem{\ascii{RDMA over Converged Ethernet。}} 
\end{enum}

此外,\tf{}已经初步开始支持\ascii{cuNCCL}库,用于改善多\ascii{GPU}间的通信。

\subsection{Kernel}

\ascii{Kernel}是\ascii{OP}在某种硬件设备的特定实现,它负责执行\ascii{OP}的具体运算。目前,\ascii{TensorFlow}系统中包含\ascii{200}多个标准的\ascii{OP},包括数值计算,多维数组操作,控制流,状态管理等。

一般每一个\ascii{OP}根据设备类型都会存在一个优化了的\ascii{Kernel}实现。在运行时,运行时根据\ascii{OP}的设备约束规范,及其本地设备的类型,为\ascii{OP}选择特定的\ascii{Kernel}实现,完成该\ascii{OP}的计算。

其中,大多数\ascii{Kernel}基于\code{Eigen::Tensor}实现。\code{Eigen::Tensor}是一个使用\ascii{C++}模板技术,为多核\ascii{CPU/GPU}生成高效的并发代码。但是,\ascii{TensorFlow}也可以灵活地直接使用\ascii{cuDNN, cuNCCL, cuBLAS}实现更高效的\ascii{Kernel}。

此外,\ascii{TensorFlow}实现了矢量化技术,在高吞吐量、以数据为中心的应用需求中,及其移动设备中,实现更高效的推理。如果对于复合\ascii{OP}的子计算过程很难表示,或执行效率低下,\ascii{TensorFlow}甚至支持更高效的\ascii{Kernel}注册,其扩展性表现非常优越。

\end{content}

\section{图控制}

\begin{content}

通过一个最简单的例子,进一步抽丝剥茧,逐渐挖掘出\tf{}计算图的控制与运行机制。

\subsection{组建集群}

如\refig{tf-1ps-1worker}所示。假如存在一个简单的分布式环境:\ascii{1 PS + 1 Worker},并将其划分为两个任务:

\begin{enum}
  \eitem{\ascii{ps0}: 使用\code{/job:ps/task:0}标记,负责模型参数的存储和更新;}
  \eitem{\ascii{worker0}: \code{/job:worker/task:0}标记,负责模型的训练。} 
\end{enum}

\begin{figure}[!htbp]
\centering
\includegraphics[width=0.9\textwidth]{figures/tf-1ps-1worker.png}
\caption{TensorFlow集群:\ascii{1 PS + 1 Worker}}
 \label{fig:tf-1ps-1worker}
\end{figure}

\subsection{图构造}

如\refig{tf-graph-construction}所示。\ascii{Client}构建了一个简单计算图;首先,将$w$与$x$进行矩阵相乘,再与截距$b$按位相加,最后更新至$s$中。

\begin{figure}[!htbp]
\centering
\includegraphics[width=0.9\textwidth]{figures/tf-graph-construction.png}
\caption{图构造}
 \label{fig:tf-graph-construction}
\end{figure}

\subsection{图执行}

如\refig{tf-graph-execution}所示。首先,\ascii{Client}创建一个\code{Session}实例,建立与\ascii{Master}之间的通道;接着,\ascii{Client}通过调用\code{Session.run}将计算图传递给\ascii{Master}。

随后,\ascii{Master}便开始启动一次\ascii{Step}的图计算过程。在执行之前,\ascii{Master}会实施一系列优化技术,例如\emph{公共表达式消除},\emph{常量折叠}等。最后,\ascii{Master}负责任务之间的协同,执行优化后的计算图。

\begin{figure}[!htbp]
\centering
\includegraphics[width=0.9\textwidth]{figures/tf-graph-execution.png}
\caption{图执行}
 \label{fig:tf-graph-execution}
\end{figure}

\subsubsection{图分裂}

如\refig{tf-graph-split-by-task}所示,存在一种合理的图划分算法。\ascii{Master}将模型参数相关的\ascii{OP}划分为一组,并放置在\ascii{ps0}任务上;其他\ascii{OP}划分为另外一组,放置在\ascii{worker0}任务上执行。

\begin{figure}[!htbp]
\centering
\includegraphics[width=1.0\textwidth]{figures/tf-graph-split-by-task.png}
\caption{图分裂:按任务划分}
 \label{fig:tf-graph-split-by-task}
\end{figure}

\subsubsection{子图注册}

如\refig{tf-register-graph}所示。在图分裂过程中,如果计算图的边跨越节点或设备,\ascii{Master}将该边实施分裂,在两个节点或设备之间插入\ascii{Send}和\ascii{Recv}节点,实现数据的传递。

其中,\code{Send}和\code{Recv}节点也是\ascii{OP},只不过它们是两个特殊的\ascii{OP},由内部运行时管理和控制,对用户不可见;并且,它们仅用于数据的通信,并没有任何数据计算的逻辑。

最后,\ascii{Master}通过调用\code{RegisterGraph}接口,将子图注册给相应的\ascii{Worker}上,并由相应的\ascii{Worker}负责执行运算。

\begin{figure}[!htbp]
\centering
\includegraphics[width=1.0\textwidth]{figures/tf-register-graph.png}
\caption{子图注册:插入Send和Recv节点}
 \label{fig:tf-register-graph}
\end{figure}

\subsubsection{子图运算}

如\refig{tf-run-graph}所示。\ascii{Master}通过调用\code{RunGraph}接口,通知所有\ascii{Worker}执行子图运算。其中,\ascii{Worker}之间可以通过调用\code{RecvTensor}接口,完成数据的交换。

\begin{figure}[!htbp]
\centering
\includegraphics[width=1.0\textwidth]{figures/tf-run-graph.png}
\caption{子图执行}
 \label{fig:tf-run-graph}
\end{figure}

\end{content}

\section{会话管理}
	
\begin{content}

接下来,通过概述会话的整个生命周期过程,及其与图控制之间的关联关系,进一步揭开运行时的内部运行机制。

\subsection{创建会话}

首先,\ascii{Client}\emph{首次}执行\code{tf.Session.run}时,会将整个图序列化后,通过\ascii{gRPC}发送\code{CreateSessionRequest}消息,将图传递给\ascii{Master}。

随后,\ascii{Master}创建一个\code{MasterSession}实例,并用全局唯一的\code{handle}标识,最终通过\code{CreateSessionResponse}返回给\ascii{Client}。如\refig{tf-create-session-overview}所示。

\begin{figure}[!h]
\centering
\includegraphics[width=0.7\textwidth]{figures/tf-create-session-overview.png}
\caption{创建会话}
 \label{fig:tf-create-session-overview}
\end{figure}

\subsection{迭代运行}

随后,\ascii{Client}会启动迭代执行的过程,并称每次迭代为一次\ascii{Step}。此时,\ascii{Client}发送\code{RunStepRequest}消息给\ascii{Master},消息携带\code{handle}标识,用于\ascii{Master}索引相应的\code{MasterSession}实例。如\refig{tf-run-step-overview}所示。

\begin{figure}[!h]
\centering
\includegraphics[width=1.0\textwidth]{figures/tf-run-step-overview.png}
\caption{迭代执行}
 \label{fig:tf-run-step-overview}
\end{figure}

\subsubsection{注册子图}

\ascii{Master}收到\code{RunStepRequest}消息后,将执行图剪枝,分裂,优化等操作。最终按照任务\ascii{(Task)},将图划分为多个子图片段\ascii{(Graph Partition)}。随后,\ascii{Master}向各个\ascii{Worker}发送\code{RegisterGraphRequest}消息,将子图片段依次注册到各个\ascii{Worker}节点上。

当\ascii{Worker}收到\code{RegisterGraphRequest}消息后,再次实施分裂操作,最终按照设备\ascii{(Device)},将图划分为多个子图片段\ascii{(Graph Partition)}。\footnote{在分布式运行时,图分裂经过两级分裂过程。在\ascii{Master}上按照任务分裂,而在\ascii{Worker}按照设备分裂。因此,得到结果都称为子图片段,它们仅存在范围,及其大小的差异。}

当\ascii{Worker}完成子图注册后,通过返回\code{RegisterGraphReponse}消息,并携带\code{graph\_handle}标识。这是因为\ascii{Worker}可以并发注册并运行多个子图,每个子图使用\code{graph\_handle}唯一标识。

\subsubsection{运行子图}

\ascii{Master}完成子图注册后,将广播所有\ascii{Worker}并发执行所有子图。这个过程是通过\ascii{Master}发送\code{RunGraphRequest}消息给\ascii{Worker}完成的。其中,消息中携带\code{(session\_handle, graph\_handle, step\_id)}三元组的标识信息,用于\ascii{Worker}索引相应的子图。

\ascii{Worker}收到消息\code{RunGraphRequest}消息后,\ascii{Worker}根据\code{graph\_handle}索引相应的子图。最终,\ascii{Worker}启动本地所有计算设备并发执行所有子图。其中,每个子图放置在单独的\code{Executor}中执行,\code{Executor}将按照拓扑排序算法完成子图片段的计算。上述算法可以形式化地描述为如下代码。


\begin{leftbar}
  \begin{python}
def run_partitions(rendezvous, executors_and_partitions, inputs, outputs):
  rendezvous.send(inputs)
  for (executor, partition) in executors_and_partitions: 
    executor.run(partition)
  rendezvous.recv(outputs)
  \end{python}
\end{leftbar}

\subsubsection{交换数据}

如果两个设备之间需要交换数据,则通过插入\ascii{Send/Recv}节点完成的。特殊地,如果两个\ascii{Worker}之间需要交换数据,则需要涉及跨进程间的通信。

此时,需要通过接收端主动发送\code{RecvTensorRequest}消息到发送方,再从发送方的信箱里取出对应的\ascii{Tensor},并通过\code{RecvTensorResponse}返回。如\refig{tf-recv-tensor-overview}所示。

\begin{figure}[!h]
\centering
\includegraphics[width=0.7\textwidth]{figures/tf-recv-tensor-overview.png}
\caption{Worker之间的数据交换}
 \label{fig:tf-recv-tensor-overview}
\end{figure}

\subsection{关闭会话}

当计算完成后,\ascii{Client}向\ascii{Master}发送\code{CloseSessionReq}消息。\ascii{Master}收到消息后,开始释放\code{MasterSession}所持有的所有资源。如\refig{tf-close-session-overview}所示。

\begin{figure}[!h]
\centering
\includegraphics[width=0.7\textwidth]{figures/tf-close-session-overview.png}
\caption{关闭会话}
 \label{fig:tf-close-session-overview}
\end{figure}

\end{content}

\input{contents/c-api}

\part{编程模型}
\input{contents/graph}
\input{contents/device}
\begin{savequote}[45mm]
\ascii{Any fool can write code that a computer can understand. Good programmers write code that humans can understand.}
\qauthor{\ascii{- Martin Flower}}
\end{savequote}

\chapter{会话} 
\label{ch:session}

\begin{content}

客户端以\code{Session}为桥梁,与后台计算引擎建立连接,并启动计算图的执行过程。其中,通过调用\code{Session.run}将触发\ascii{TensorFlow}的一次计算\ascii{(Step)}。

事实上,\code{Session}建立了执行计算图的闭包环境,它封装了\ascii{OP}计算,及其\ascii{Tensor}求值的计算环境。

\end{content}

\section{资源管理}

\begin{content}

在\code{Session}的生命周期中,将根据计算图的计算需求,按需分配系统资源,包括变量,队列,读取器等。

\subsection{关闭会话}

当计算完成后,需要确保\code{Session}被安全地关闭,以便安全释放所管理的系统资源。

\begin{leftbar}
\begin{python}
sess = tf.Session()
sess.run(targets)
sess.close()
\end{python}
\end{leftbar}

\subsection{上下文管理器}

一般地,常常使用上下文管理器创建\code{Session},使得\code{Session}在计算完成后,能够自动关闭,确保资源安全性地被释放。

\begin{leftbar}
\begin{python}
with tf.Session() as sess:
  sess.run(targets)
\end{python}
\end{leftbar}

\subsection{图实例}

一个\code{Session}实例,只能运行一个图实例;但是,一个图实例,可以运行在多个\code{Session}实例中。如果尝试在同一个\code{Session}运行另外一个图实例,必须先关闭\code{Session}(不必销毁),再启动新图的计算过程。

虽然一个\code{Session}实例,只能运行一个图实例。但是,可以\code{Session}是一个线程安全的类,可以并发地执行该图实例上的不同子图。例如,一个典型的机器学习训练模型中,可以使用同一个\code{Session}实例,并发地运行输入子图,训练子图,及其\ascii{Checkpoint}子图。

\subsubsection{引用计数器}

为了提高效率,避免计算图频繁地创建与销毁,存在一种实现上的优化技术。在图实例中维护一个\code{Session}的引用计数器,当且仅当\code{Session}的数目为零时,才真正地销毁图实例。

\begin{figure}[!htbp]
\centering
\includegraphics[width=0.7\textwidth]{figures/tf-graph-session-relation.png}
\caption{优化技术:会话实例的引用计数器}
 \label{fig:tf-graph-session-relation}
\end{figure}

\subsubsection{数据结构}

此处,摘取\code{TF\_Graph}部分关于\code{Session}引用计数器技术的关键字段;其中,\code{TF\_Graph}结构体定义于\ascii{C API}的头文件。

\begin{leftbar}
\begin{c++}
struct TF_Graph {
  TF_Graph();

  tensorflow::mutex mu;
  tensorflow::Graph graph GUARDED_BY(mu);

  // TF\_Graph may only and must be deleted when
  // num\_sessions == 0 and delete\_requested == true

  // num\_sessions incremented by TF\_NewSession, 
  // and decremented by TF\_DeleteSession.
  int num_sessions GUARDED_BY(mu);
  bool delete_requested GUARDED_BY(mu);
};
\end{c++}
\end{leftbar}

同理,\code{TF\_Session}持有一个二元组:\code{<tensorflow::Sesssion, TF\_Graph>},它们之间是一对一的关系。其中,\code{tensorflow::Sesssion}是\ascii{C++}客户端侧的会话实例。

\begin{leftbar}
\begin{c++}
struct TF_Session {
  TF_Session(tensorflow::Session* s, TF_Graph* g)
      : session(s), graph(g), last_num_graph_nodes(0) {}
  tensorflow::Session* session;
  TF_Graph* graph;
  tensorflow::mutex mu;
  int last_num_graph_nodes;
};
\end{c++}
\end{leftbar}

\subsubsection{创建会话}

\begin{leftbar}
\begin{c++}
TF_Session* TF_NewSession(TF_Graph* graph, const TF_SessionOptions* opt,
                          TF_Status* status) {
  Session* session;
  status->status = NewSession(opt->options, &session);
  if (status->status.ok()) {
    if (graph != nullptr) {
      mutex_lock l(graph->mu);
      graph->num_sessions += 1;
    }
    return new TF_Session(session, graph);
  } else {
    DCHECK_EQ(nullptr, session);
    return nullptr;
  }
}
\end{c++}
\end{leftbar}

\subsubsection{销毁会话}

\begin{leftbar}
\begin{c++}
void TF_DeleteSession(TF_Session* s, TF_Status* status) {
  status->status = Status::OK();
  TF_Graph* const graph = s->graph;
  if (graph != nullptr) {
    graph->mu.lock();
    graph->num_sessions -= 1;
    const bool del = graph->delete_requested && graph->num_sessions == 0;
    graph->mu.unlock();
    if (del) delete graph;
  }
  delete s->session;
  delete s;
}
\end{c++}
\end{leftbar}

\end{content}

\section{默认会话}

\begin{content}

通过调用\ascii{Session.as\_default()},将该\code{Session}置为默认\code{Session},同时它返回了一个上下文管理器。在默认\code{Session}的上文中,可以直接实施\ascii{OP}的运算,或者\ascii{Tensor}的求值。

\begin{leftbar}
\begin{python}
hello = tf.constant('hello, world')

sess = tf.Session()  
with sess.as_default():
  print(hello.eval())
sess.close()
\end{python}
\end{leftbar}

但是,\code{Session.as\_default()}并不会自动关闭\code{Session},需要用户显式地调用\code{Session.close}方法。

\subsection{张量求值}

如上例代码,\code{hello.eval()}等价于\code{tf.get\_default\_session().run(hello)}。其中,\code{Tensor.eval}如下代码实现。

\begin{leftbar}
\begin{python}
class Tensor(_TensorLike):
  def eval(self, feed_dict=None, session=None):
    if session is None:
      session = get_default_session()
    return session.run(tensors, feed_dict)
\end{python}
\end{leftbar}

\subsection{OP运算}

同理,当用户未显式提供\code{Session},\code{Operation.run}将自动获取默认的\code{Session}实例,并按照当前\ascii{OP}的依赖关系,以某个特定的拓扑排序执行该计算子图。

\begin{leftbar}
\begin{python}
class Operation(object):
  def run(self, feed_dict=None, session=None):
    if session is None:
      session = tf.get_default_session()
    session.run(self, feed_dict)
\end{python}
\end{leftbar}

\subsection{线程相关}

默认会话仅仅对当前线程有效,以便在当前线程追踪Session的调用栈。如果在新的线程中使用默认会话,需要在线程函数中通过调用\code{as\_default}将\code{Session}置为默认会话。

事实上,在\ascii{TensorFlow}运行时维护了一个\code{Session}的本地线程栈,实现默认\code{Session}的自动管理。

\begin{leftbar}
\begin{python}
_default_session_stack = _DefaultStack()

def get_default_session(session):
  return _default_session_stack.get_default(session)
\end{python}
\end{leftbar}

其中,\code{\_DefaultStack}表示栈的数据结构。

\begin{leftbar}
\begin{python}
class _DefaultStack(threading.local):
  def __init__(self):
    super(_DefaultStack, self).__init__()
    self.stack = []

  def get_default(self):
    return self.stack[-1] if len(self.stack) >= 1 else None

  @contextlib.contextmanager
  def get_controller(self, default):
    try:
      self.stack.append(default)
      yield default
    finally:
      self.stack.remove(default)
\end{python}
\end{leftbar}

\end{content}

\section{会话类型}

\begin{content}

一般地,存在两种基本的会话类型:\code{Session}与\code{InteractiveSession}。后者常常用于交互式环境,它在构造期间将其自身置为默认,简化默认会话的管理过程。

此外,两者在运行时的配置也存在差异。例如,\code{InteractiveSession}将\code{GPUOptions.allow\_growth}置为\code{True},避免在实验环境中独占整个GPU的存储资源。

\begin{figure}[!htbp]
\centering
\includegraphics[width=0.7\textwidth]{figures/py-session-hierarchy.png}
\caption{Session:类层次结构}
 \label{fig:py-session-hierarchy}
\end{figure}

\subsection{Session}

\code{Session}继承\code{BaseSession},并增加了默认图与默认会话的上下文管理器的功能,保证系统资源的安全释放。

一般地,使用\code{with}进入会话的上下文管理器,并自动切换默认图与默认会话的上下文;退出\code{with}语句时,将自动关闭默认图与默认会话的上下文,并自动关闭会话。

\begin{leftbar}
\begin{python}
class Session(BaseSession):
  def __init__(self, target='', graph=None, config=None):
    super(Session, self).__init__(target, graph, config=config)
    self._default_graph_context_manager = None
    self._default_session_context_manager = None

  def __enter__(self):
    self._default_graph_context_manager = self.graph.as_default()
    self._default_session_context_manager = self.as_default()

    self._default_graph_context_manager.__enter__()
    return self._default_session_context_manager.__enter__()

  def __exit__(self, exec_type, exec_value, exec_tb):
    self._default_session_context_manager.__exit__(
        exec_type, exec_value, exec_tb)
    self._default_graph_context_manager.__exit__(
        exec_type, exec_value, exec_tb)

    self._default_session_context_manager = None
    self._default_graph_context_manager = None

    self.close()
\end{python}
\end{leftbar}

\subsection{InteractiveSession}

与\code{Session}不同,\code{InteractiveSession}在构造期间将其自身置为默认,并实现默认图与默认会话的自动切换。与此相反,\code{Session}必须借助于\code{with}语句才能完成该功能。在交互式环境中,\code{InteractiveSession}简化了用户管理默认图和默认会话的过程。

同理,\code{InteractiveSession}在计算完成后需要显式地关闭,以便安全地释放其所占用的系统资源。

\begin{leftbar}
\begin{python}
class InteractiveSession(BaseSession):
  def __init__(self, target='', graph=None, config=None):
    super(InteractiveSession, self).__init__(target, graph, config)

    self._default_session_context_manager = self.as_default()
    self._default_session_context_manager.__enter__()

    self._default_graph_context_manager = graph.as_default()
    self._default_graph_context_manager.__enter__()

  def close(self):
    super(InteractiveSession, self).close()
    self._default_graph.__exit__(None, None, None)
    self._default_session.__exit__(None, None, None)
\end{python}
\end{leftbar}

\subsection{BaseSession}

\code{BaseSession}是两者的基类,它主要实现会话的创建,关闭,执行,销毁等管理生命周期的操作;它与后台计算引擎相连接,实现前后端计算的交互。

\subsubsection{创建会话}

通过调用\ascii{C API}的接口,\code{self.\_session}直接持有后台计算引擎的会话句柄,后期执行计算图,关闭会话等操作都以此句柄为标识。

\begin{leftbar}
\begin{python}
class BaseSession(SessionInterface):
  def __init__(self, target='', graph=None, config=None):
    # ignore implements...
    with errors.raise_exception_on_not_ok_status() as status:
      self._session = 
        tf_session.TF_NewDeprecatedSession(opts, status)
\end{python}
\end{leftbar}

\subsubsection{执行计算图}

通过调用\code{run}接口,实现计算图的一次计算。它首先通过\code{tf\_session.TF\_ExtendGraph}将图注册给后台计算引擎,然后再通过调用\code{tf\_session.TF\_Run}启动计算图的执行。

\begin{leftbar}
\begin{python}
class BaseSession(SessionInterface):
  def run(self, 
    fetches, feed_dict=None, options=None, run_metadata=None):
    self._extend_graph()
    with errors.raise_exception_on_not_ok_status() as status:
      return tf_session.TF_Run(session, 
        options, feed_dict, fetch_list, 
        target_list, status, run_metadata)
  
  def _extend_graph(self):
    with errors.raise_exception_on_not_ok_status() as status:
      tf_session.TF_ExtendGraph(self._session,
        graph_def.SerializeToString(), status)  
\end{python}
\end{leftbar}

\subsubsection{关闭会话}

\begin{leftbar}
\begin{python}
class BaseSession(SessionInterface):
  def close(self):
    with errors.raise_exception_on_not_ok_status() as status:
      tf_session.TF_CloseDeprecatedSession(self._session, status)
\end{python}
\end{leftbar}

\subsubsection{销毁会话}

\begin{leftbar}
\begin{python}
class BaseSession(SessionInterface):
  def __del__(self):
    try:
      status = tf_session.TF_NewStatus()
      tf_session.TF_DeleteDeprecatedSession(self._session, status)
    finally:
      tf_session.TF_DeleteStatus(status)
\end{python}
\end{leftbar}

\end{content}
\input{contents/variable}
\begin{savequote}[45mm]
\ascii{Any fool can write code that a computer can understand. Good programmers write code that humans can understand.}
\qauthor{\ascii{- Martin Flower}}
\end{savequote}

\chapter{队列} 
\label{ch:queue}

\begin{content}

\ascii{TensorFlow}的\code{Session}是线程安全的。也就是说,多个线程可以使用同一个\code{Session}实例,并发地执行同一个图实例的不同\ascii{OP};\ascii{TensorFlow}执行引擎会根据输入与输出对图实施剪枝,得到一个最小依赖的子图。

因此,通过多线程并使用同一个\code{Session}实例,并发地执行同一个图实例的不同\ascii{OP},最终实现的效果是子图之间的并发执行。

对于典型的模型训练,可以充分发挥\code{Session}多线程的并发能力,提升训练的性能。例如,输入子图运行在一个单独的线程中,用于准备样本数据;而训练子图则运行在另外一个单独的线程中,并按照\code{batch\_size}的大小一个批次取走训练样本,并启动迭代的训练过程。

本文将讲解上述并发模型中的基础设施,包括队列,多线程的协调器,及其控制\code{Enqueue OP}执行的\ascii{QueueRunner}。

\end{content}

\section{队列}

\begin{content}

在\ascii{TensorFlow}的执行引擎中,\ascii{Queue}是一种控制异步计算的强大工具。特殊地,\ascii{Queue}是一种特殊的\ascii{OP},与\ascii{Variable}类似,它是一类有状态的\ascii{OP}。

与之类似,\ascii{Variable}拥有关联的\ascii{Assign}等修改其状态的\ascii{OP},\ascii{Queue}也有与之关联的\ascii{OP},例如\code{Enqueue,Dequeue,EnqueueMany,DequeueMany}等\ascii{OP},它们都能直接修改\ascii{Queue}的状态。

\subsection{FIFOQueue}

举一个简单例子。首先,构造了一个\code{FIFOQueue}队列;然后,在计算图中添加了一个\code{EnqueueMany},该\ascii{OP}用于在队列头部追加\ascii{1}个或多个元素;其次,再添加一个出队的\code{Dequeue};最后,将出队元素的值增加\ascii{1},再将其结果入队。在启动计算图执行之前,计算图的构造如下图所示。

\begin{figure}[!h]
\centering
\includegraphics[width=0.9\textwidth]{figures/py-queue-example-1.png}
\caption{图构造期}
 \label{fig:py-queue-example-1}
\end{figure}

执行\code{EnqueueMany}操作后,计算图的状态如下图所示。

\begin{figure}[!h]
\centering
\includegraphics[width=0.9\textwidth]{figures/py-queue-example-2.png}
\caption{图执行期:执行一次EnqueueMany}
 \label{fig:py-queue-example-2}
\end{figure}


执行第一步\code{Enqueue}后,计算图的状态如下图所示。

\begin{figure}[!h]
\centering
\includegraphics[width=0.9\textwidth]{figures/py-queue-example-3.png}
\caption{图执行期:执行一次Enqueue}
 \label{fig:py-queue-example-3}
\end{figure}

\subsection{用途}

队列在模型训练中扮演重要角色,后文将讲述数据加载的\ascii{Pipeline},训练模型常常使用\code{RandomShuffleQueue}为其准备样本数据。为了提高\ascii{IO}的吞吐率,可以使用多线程,并发地将样本数据追加到样本队列中;与此同时,训练模型的线程迭代执行\code{train\_op}时,一次获取\code{batch\_size}大小的批次样本数据。

显而易见,队列在\ascii{Pipeline}过程中扮演了异步协调和数据交换的功能,这给\ascii{Pipeline}的设计和实现带来很大的弹性空间。

需要注意的是,为了使得队列在多线程最大化发挥作用,需要解决两个棘手的问题:

\begin{enum}
  \eitem{如何同时停止所有的线程,及其处理异常报告?}
  \eitem{如何并发地向队列中追加样本数据?} 
\end{enum}

因此,\ascii{TensorFlow}设计了\code{tf.train.Coordinator}和\code{tf.train.QueueRunner}两个类,分别解决上述两个问题。

这两个类相辅相成,\code{Coordinator}协调多个线程同时停止运行,并且向等待停止通知的主程序报告异常;而\code{QueueRunner}创建了一组线程,并协作多个入队\ascii{OP}(例如\code{Enqueue,EnqueueMany})的执行。

\end{content}

\section{协调器}

\begin{content}

\code{Coordinator}提供了一种同时停止一组线程执行的简单机制。它拥有\ascii{3}个重要的方法:

\begin{enum}
\eitem{\code{should\_stop}: 判断当前线程是否应该退出}
\eitem{\code{request\_stop}: 请求所有线程停止执行}
\eitem{\code{join}: 等待所有线程停止执行}
\end{enum}

\subsection{使用方法}

一般地,主程序常常使用如下模式使用\code{Coordinator}。

\begin{leftbar}
\begin{python}
# Create a coordinator.
coord = tf.train.Coordinator()

# Create 10 threads that run 'MyLoop()'
threads = [threading.Thread(target=MyLoop, args=(coord,)) 
          for i in xrange(10)]

# Start the threads.
for t in threads:
  t.start()
  
# wait for all of them to stop
coord.join(threads)
\end{python}
\end{leftbar}

任何子线程,都可以通过调用\code{coord.request\_stop},通知其他线程停止执行。因此,每个线程的迭代执行中,都要事先检查\code{coord.should\_stop()}。一旦\code{coord.request\_stop}被调用,其他线程的\code{coord.request\_stop()}将立即返回\code{True}。

一般地,一个子线程的迭代执行方法遵循如下实现模式。

\begin{leftbar}
\begin{python}
def MyLoop(coord):
  try
    while not coord.should_stop():
      # ...do something...
  except Exception as e:
    coord.request_stop(e)
\end{python}
\end{leftbar}

\subsection{异常处理}

当某个线程发生了异常,则可以通过\code{coord.request\_stop(e)}报告异常的发生。

\begin{leftbar}
\begin{python}
try:
  while not coord.should_stop():
    # ...do some work...
except Exception as e:
  coord.request_stop(e)
\end{python}
\end{leftbar}

为了消除异常代码处理的重复代码,可以使用\code{coord.stop\_on\_exception()}的上下文管理器。

\begin{leftbar}
\begin{python}
with coord.stop_on_exception():
  while not coord.should_stop():
    # ...do some work...
\end{python}
\end{leftbar}

其中,该异常也会在\code{coord.join}中被重新抛出。因此,在主程序也需要合理地处理异常。

\begin{leftbar}
\begin{python}
try:
  # Create a coordinator.
  coord = tf.train.Coordinator()

  # Create 10 threads that run 'MyLoop()'
  threads = [threading.Thread(target=MyLoop, args=(coord,)) 
            for i in xrange(10)]

  # Start the threads.
  for t in threads:
    t.start()

  # wait for all of them to stop
  coord.join(threads)
except Exception as e:
  # ...exception that was passed to coord.request\_stop(e)
\end{python}
\end{leftbar}

\subsection{实战:LoopThread}

\end{content}

\section{QueueRunner}

\begin{content}

一个\code{QueueRunner}实例持有一个或多个\code{Enqueue}的入队\ascii{OP},它为每个\code{Enqueue OP}启动一个线程。

\begin{figure}[!htbp]
\centering
\includegraphics[width=0.9\textwidth]{figures/tf-queue-runner-model.png}
\caption{TensorFlow系统架构}
 \label{fig:tf-queue-runner-model}
\end{figure}

\subsection{注册QueueRunner}

可以调用\code{tf.train.add\_queue\_runner}往计算图中注册\code{QueueRunner}实例,并且将其添加到\code{GraphKeys.QUEUE\_RUNNERS}集合中。

\begin{leftbar}
\begin{python}
def add_queue_runner(qr, collection=ops.GraphKeys.QUEUE_RUNNERS):
  ops.add_to_collection(collection, qr)
\end{python}
\end{leftbar}

\subsection{执行QueueRunner}

可以调用\code{tf.train.start\_queue\_runners}时,它会从计算图中找到所有\code{QueueRunner}实例,并从\code{QueueRunner}实例中取出所有\code{Enqueue OP},为每个\ascii{OP}启动一个线程。

\begin{leftbar}
\begin{python}
def start_queue_runners(sess, coord, daemon=True, start=True,
                        collection=ops.GraphKeys.QUEUE_RUNNERS):
  with sess.graph.as_default():
    threads = []
    for qr in ops.get_collection(collection):
      threads.extend(qr.create_threads(
          sess, coord=coord, daemon=daemon, start=start))
  return threads
\end{python}
\end{leftbar}

在\code{QueueRunner.create\_threads}方法中,为其包含的每个\code{Enqueue}类型的\ascii{OP}启动单独的线程。

\begin{leftbar}
\begin{python}
class QueueRunner(object):
  def create_threads(self, sess, coord, daemon, start):
    """Create threads to run the enqueue ops.
    """
    threads = [threading.Thread(
        target=self._run, args=(sess, op, coord))
        for op in self._enqueue_ops]
    if coord:
      threads.append(threading.Thread(
          target=self._close_on_stop, 
          args=(sess, self._cancel_op, coord)))
    for t in threads:
      if coord:
        coord.register_thread(t)
      if daemon:
        t.daemon = daemon
      if start:
        t.start()
    return threads
\end{python}
\end{leftbar}

\subsubsection{迭代执行Enqueue}

每个\code{Enqueue}子线程将迭代执行\code{Enqueue OP}。当发生\code{OutOfRangeError}异常时,将自动关闭队列,并退出子线程;但是,如果发生其他类型的异常,会主动通知\code{Coordinator}停止所有线程的运行,并退出子线程。

\begin{leftbar}
\begin{python}
class QueueRunner(object):
  def _run(self, sess, enqueue_op, coord):
    try:
      enqueue_callable = sess.make_callable(enqueue_op)
      while True:
        if coord.should_stop():
          break
        try:
          enqueue_callable()
        except errors.OutOfRangeError:  
          sess.run(self._close_op)
          return
    except Exception as e:
      coord.request_stop(e)
\end{python}
\end{leftbar}

\subsubsection{监听队列关闭}

另外,如果给定\code{Coordinator}实例,\code{QueueRunner}还会额外启动一个线程;当\code{Coordinator}实例被触发调用\code{request\_stop}方法后,该线程将会自动关闭队列。

\begin{leftbar}
\begin{python}
class QueueRunner(object):
  def _close_on_stop(self, sess, cancel_op, coord):
    """Close the queue, and cancel pending enqueue ops
       when the Coordinator requests stop.
    """
    coord.wait_for_stop()
    try:
      sess.run(cancel_op)
    except Exception:
      pass
\end{python}
\end{leftbar}

其中,\code{Queue}的\code{Cancel OP}与\code{Close OP}都会关闭队列,但是\code{Cancel OP}会撤销已缓存的\code{Enqueue OP}列表,但\code{Close OP}则保留已缓存的\code{Enqueue OP}列表。

\subsection{关闭队列}

当队列被关闭后,对于任何尝试\code{Enqueue}将会产生错误。但是,对于任何尝试\code{Dequeue}依然是成功的,只要队列中遗留元素;否则,\code{Dequeue}将立即失败,抛出\code{OutOfRangeError}异常,而不会阻塞等待更多元素被入队。
\end{content}
\input{contents/essential-op}

\part{运行模型}
\begin{savequote}[45mm]
\ascii{Any fool can write code that a computer can understand. Good programmers write code that humans can understand.}
\qauthor{\ascii{- Martin Flower}}
\end{savequote}

\chapter{本地执行} 
\label{ch:local}

\begin{content}

\tf{}可以独立地运行在一个进程内,完成计算图的执行过程。本章将重点介绍本地运行时的基本架构与运行机制;重点讨论计算图剪枝、分裂、优化、执行等实现技术细节;并且详细探究在本地模式下,跨设备间\ascii{OP}之间数据交互的工作机制,及其\ascii{OP}在设备集上的编排(\ascii{placement})算法。

\end{content}

\section{本地模式}
\label{sec:local-runtime}

\begin{content}

如\refig{local}所示,在本地模式下,\ascii{Client, Master, Worker}部署在同一台机器同一进程内,并由\code{DirectSession}同时扮演这三个角色。\code{DirectSession}运行在单独的进程内,各服务实体之间是函数调用关系。

\begin{figure}[H]
\centering
\includegraphics[width=0.7\textwidth]{figures/local.png}
\caption{本地模式}
 \label{fig:local}
\end{figure}

\ascii{Client}负责计算图的构造,通过调用\code{Session.run},启动计算图的执行过程。如\refig{local-runtime}所示,在\code{run\_step}执行过程之中,涉及计算图的剪枝、分裂、执行三个重要阶段。

\begin{figure}[H]
\centering
\includegraphics[width=0.8\textwidth]{figures/local-runtime.png}
\caption{本地模式:图操作}
 \label{fig:local-runtime}
\end{figure}

\subsection{部分执行}

\ascii{Master}收到计算图执行命令后,启动计算图的剪枝操作。它根据计算图的输入输出反向遍历图,寻找一个最小依赖的子图,常称为\code{ClientGraph}。

也就是说,每次执行\code{run\_step}时,并不会执行整个计算图(\code{FullGraph}),而是执行部分的子图。剪枝体现了\tf{}部分执行的设计理念。

\subsection{并发执行}

然后,运行时按照当前设备集完成图的分裂,生成了很多子图,每个子图称为\code{PartitionGraph};然后触发各个\ascii{Worker}并发地执行每个\code{PartitionGraph};对于每一个\ascii{PartitionGraph},运行时将启动一个\ascii{Executor},按照其拓扑排序完成\code{PartitionGraph}的执行。

也就是说,分裂和执行体现了\tf{}并发执行的设计理念。

\section{会话控制}

在本地模式下,其运行时由\code{DirectSession}控制。一般地,\code{DirectSession}执行计算图时,各组件之间都是函数调用关系。但是,\code{DirectSession}也存在清晰的生命周期管理机制,如\refig{local-direct-session-lifecycle}所示。

\begin{figure}[H]
\centering
\includegraphics[width=0.6\textwidth]{figures/local-direct-session-lifecycle.png}
\caption{DirectSession生命周期}
 \label{fig:local-direct-session-lifecycle}
\end{figure}

\subsection{领域模型}

如\refig{local-direct-session-model}所示,\code{DirectSession}持有\code{SimpleGraphExecutionState}实例,后者负责计算图的剪枝,生成\code{ClientGraph}实例。

\code{DirectSession}同时持有一组线程池,但是没次\code{DirectSession.run}时,根据外部配置的索引,从线程池组里选择其一为其提供服务。因为\code{DirectSession}是线程安全的,支持多个并发执行的\code{DirectSession.run},即可以同时运行多个线程池实例。

\begin{figure}[H]
\centering
\includegraphics[width=0.9\textwidth]{figures/local-direct-session-model.png}
\caption{DirectSession领域模型}
 \label{fig:local-direct-session-model}
\end{figure}

\subsection{创建会话}

如\refig{local-direct-session-factory}所示,\code{DirectSession}由\code{DirectSessionFactory}多态创建。其中,\code{DeviceFactory::AddDevices}将创建本地设备集。

其中,\code{DirectSession}中主要完成线程池组的创建。

\begin{figure}[H]
\centering
\includegraphics[width=0.6\textwidth]{figures/local-direct-session-factory.png}
\caption{多态创建DirectSession}
 \label{fig:local-direct-session-factory}
\end{figure}

\begin{leftbar}
\begin{c++}
struct DirectSessionFactory : SessionFactory {
  bool AcceptsOptions(const SessionOptions& options) override {
    return options.target.empty();
  }

  Session* NewSession(const SessionOptions& options) override {
    std::vector<Device*> devices;
    DeviceFactory::AddDevices(
        options, "/job:localhost/replica:0/task:0", &devices);
    return new DirectSession(options, new DeviceMgr(devices));
  }
};
\end{c++}
\end{leftbar}

其中,\code{DirectSessionFactory::NewSession}由\ascii{C API}调用。

\begin{leftbar}
\begin{c++}
Status NewSession(const SessionOptions& options, Session** out_session) {
  SessionFactory* factory;
  Status s = SessionFactory::GetFactory(options, &factory);
  if (!s.ok()) {
    *out_session = nullptr;
    return s;
  }
  *out_session = factory->NewSession(options);
  if (!*out_session) {
    return errors::Internal("Failed to create session.");
  }
  return Status::OK();
}

TF_DeprecatedSession* TF_NewDeprecatedSession(
  const TF_SessionOptions* opt, TF_Status* status) {
  Session* session;
  status->status = NewSession(opt->options, &session);
  if (status->status.ok()) {
    return new TF_DeprecatedSession({session});
  } else {
    return nullptr;
  }
}
\end{c++}
\end{leftbar}

在\code{DirectSession}的构造函数中,主要负责其领域模型的初始化,包括线程池的创建,构建\code{CancellationManager}实例。

\begin{leftbar}
\begin{c++}
DirectSession::DirectSession(
    const SessionOptions& options,
    const DeviceMgr* device_mgr)
    : options_(options),
      device_mgr_(device_mgr),
      cancellation_manager_(new CancellationManager()) {
  // thread\_pools\_ = ... 
}
\end{c++}
\end{leftbar}

\subsection{销毁会话}

由\ascii{SessionFactory}所\code{new}出来的\code{DirectSession},由\ascii{C API}负责\code{delete}掉。

\begin{leftbar}
\begin{c++}
void TF_DeleteDeprecatedSession(TF_DeprecatedSession* s, TF_Status* status) {
  status->status = Status::OK();
  delete s->session;  // delete DirectSession
  delete s;
}
\end{c++}
\end{leftbar}

随后,\code{DirectSession}的析构函数被调用,它负责清理其负责管理的系统资源。主要包括\code{Executor}列表,\code{ThreadPool}列表,\code{CancellationManager}实例。

\begin{leftbar}
\begin{c++}
DirectSession::~DirectSession() {
  for (auto& it : partial_runs_) {
    it.second.reset(nullptr);
  }
  
  for (auto& it : executors_) {
    it.second.reset();
  }
  
  for (auto d : device_mgr_->ListDevices()) {
    d->op_segment()->RemoveHold(session_handle_);
  }
  
  delete cancellation_manager_;
  
  for (const auto& p_and_owned : thread_pools_) {
    if (p_and_owned.second) delete p_and_owned.first;
  }

  execution_state_.reset(nullptr);
  flib_def_.reset(nullptr);
}
\end{c++}
\end{leftbar}

\subsection{创建/扩展图}

首次扩展图,等价于创建图。扩展图就是在原有计算图的基础上,追加新的子图。当然,追加的子图中所包含的节点,在原有的计算图中不应该存在。


\begin{leftbar}
\begin{c++}
Status DirectSession::Create(const GraphDef& graph) {
  if (graph.node_size() > 0) {
    mutex_lock l(graph_def_lock_);
    return ExtendLocked(graph);
  }
  return Status::OK();
}

Status DirectSession::Extend(const GraphDef& graph) {
  mutex_lock l(graph_def_lock_);
  return ExtendLocked(graph);
}
\end{c++}
\end{leftbar}

当创建计算图时,\code{DirectSession}主要完成\code{SimpleGraphExecutionState}实例的创建。如\refig{local-simple-graph-execution-state-model}所示,\code{SimpleGraphExecutionState}实例持有\code{FullGraph}两种格式的实例:\code{Graph}与\code{GraphDef},并由它负责管理和维护\code{FullGraph}的生命周期。

\begin{figure}[H]
\centering
\includegraphics[width=0.5\textwidth]{figures/local-simple-graph-execution-state-model.png}
\caption{创建SimpleGraphExecutionState实例}
 \label{fig:local-simple-graph-execution-state-model}
\end{figure}

其中,\code{SimpleGraphExecutionState}的主要职责包括:

\begin{enum}
  \eitem{构造\code{FullGraph}}:发生在\code{DirectSession.Create};
  \eitem{执行简单的\ascii{OP}编排算法}:发生在\code{DirectSession.Create};
  \eitem{执行图的剪枝操作}:发生在\code{DirectSession.Run}。
\end{enum}

当执行\code{DirectSession::Create}时,将创建\code{SimpleGraphExecutionState}实例,并完成\code{FullGraph}实例的构建和初始化。

\begin{leftbar}
\begin{c++}
Status SimpleGraphExecutionState::MakeForBaseGraph(
    GraphDef* graph_def, const SimpleGraphExecutionStateOptions& opts,
    std::unique_ptr<SimpleGraphExecutionState>* out_state) {
  auto ret = std::make_unique<SimpleGraphExecutionState>(graph_def, opts));

  AddDefaultAttrsToGraphDef(&ret->original_graph_def_, *ret->flib_def_, 0));
  if (!ret->session_options_->config.graph_options().place_pruned_graph()) {
    ret->InitBaseGraph();
  }
  *out_state = std::move(ret);
  return Status::OK();
}
\end{c++}
\end{leftbar}

其中,\code{SimpleGraphExecutionState::InitBaseGraph}完成\code{FullGraph}从\code{GraphDef}到\code{Graph}的格式转换,并启动\code{SimplePlacer}的\ascii{OP}编排算法。

\begin{leftbar}
\begin{c++}
Status SimpleGraphExecutionState::InitBaseGraph() {
  auto ng = std::make_unique<Graph>(OpRegistry::Global());

  GraphConstructorOptions opts;
  ConvertGraphDefToGraph(opts, *original_graph_def_, ng.get());

  SimplePlacer placer(ng.get(), device_set_, session_options_);
  placer.Run();

  this->graph_ = ng.release();
  return Status::OK();
}
\end{c++}
\end{leftbar}

\subsubsection{图构造:GraphDef -> Graph}

刚开始,\code{SimpleGraphExecutionState}得到的是\code{GraphDef},这是最原始的图结构。它由\ascii{Client}将序列化后传递到后端\ascii{C++},然后由后端反序列化得到的图结构。

如\refig{local-graph-def-to-graph}所示,通过调用\code{ConvertGraphDefToGraph}将\code{GraphDef}实例变换为等价的\code{Graph}实例;同理,可以调用\code{Graph.ToGraphDef}将\code{Graph}实例变换为等价的\code{GraphDef}实例。

其中,\code{GraphDef}是使用\ascii{protobuf}格式存在的图结构,它包含了图所有元数据;而\code{Graph}则是运行时系统中用于描述图结构的领域对象,它不仅仅持有\code{GraphDef}的元数据,并包含其它图结构的其它信息。

\begin{figure}[H]
\centering
\includegraphics[width=0.6\textwidth]{figures/local-graph-def-to-graph.png}
\caption{\code{GraphDef}与\code{Graph}之间的格式转换}
 \label{fig:local-graph-def-to-graph}
\end{figure}

\subsubsection{OP编排:SimplePlacer}

\ascii{OP}的编排(\ascii{placement})指的是,将计算图中包含的\ascii{OP}以最高效的方式置放在合适的计算设备上运算,以最大化计算资源的利用率,可以形式化地描述为\refig{local-cost-model}。

\begin{figure}[H]
\centering
\includegraphics[width=0.6\textwidth]{figures/local-cost-model.png}
\caption{费用模型}
 \label{fig:local-cost-model}
\end{figure}

求取最优的编排方案,我猜想这是一个\ascii{NP}问题。该问题取决于计算图的特征,网络拓扑与带宽,样本数目等多个复杂的因素,该问题也是社区中最活跃的问题之一。

\subsection{迭代执行}

\code{DirectSession.Run}是\tf{}运行时的关键路径,它负责完成一次迭代计算。首先,\code{DirectSession}根据输入/输出对\code{FullGraph}实施剪枝,生成\code{ClientGraph};然后,根据所持有本地设备集,将\code{ClientGraph}分裂为多个\code{PartitionGraph};运行时为其每个\code{PartitionGraph}启动一个\code{Executor}实例,后者执行\code{PartitionGraph}的拓扑排序算法,完成计算图的执行。

具体实现,请参照\refsec{graph-operation-prune},\refsec{graph-operation-split},\refsec{graph-operation-exec}。

\subsubsection{图操作}

如\refig{local-graph-transformation}所示,在本地模式下,计算图经历三个形态的变换,最终被分解至各个计算设备上,以便实现在各个计算设备上并发执行子图。

\begin{figure}[H]
\centering
\includegraphics[width=0.9\textwidth]{figures/local-graph-transformation.png}
\caption{图变换}
 \label{fig:local-graph-transformation}
\end{figure}

\begin{itemize}
  \item \code{FullGraph}: \ascii{Client}负责构造的完整的计算图,常称为\code{FullGraph};但是,一次\code{Session.run}并不会执行整个计算图;
  \item \code{ClientGraph}: \ascii{Master}根据\code{Session.run}传递\code{feeds, fetches}输入输出列表,对\ascii{FullGraph}实施剪枝操作,计算得到本地迭代执行的最小依赖子图,常称为\code{ClientGraph};
  \item \code{PartitionGraph}: \ascii{Master}根据当前计算设备集,及其\ascii{OP}的设备约束规范,将\code{ClientGraph}分裂为多个\code{PartitionGraph};其中,每个计算设备对应一个\code{PartitionGraph},计算设备负责\code{PartitionGraph}的执行。
\end{itemize}

但是,\code{FullGraph, ClientGraph, PartitionGraph}的数据结构相同,它们都是\code{Graph}三种不同表现形式,仅仅大小和范畴存在差异。

\subsubsection{形式化}

在真实的系统实现中,本地模式的运行时是使用\ascii{C++}实现。其中,\tf{}运行时的关键路径为\code{run\_step}。因为真实系统实现中涉及过多的细节,不易发现算法的主干和逻辑。为了简化问题的描述,将形式化地描述\code{run\_step}的实现过程。

\begin{leftbar}
\begin{python}
def do_run_partitions(executors_and_partitions):
  barrier = ExecutorBarrier(executors_and_partitions.size())
  for (executor, partition) in executors_and_partitions:
    executor.run(partition, barrier)  
  barrier.wait()

def run_partitions(executors_and_partitions, inputs, outputs):
  frame = FunctionCallFrame()
  frame.set_args(inputs)
  do_run_partitions(executors_and_partitions)
  frame.get_ret_vals(outputs)

def run_step(devices, full_graph, inputs, outputs):
  client_graph = prune(full_graph, inputs, outputs)
  executors_and_partitions = split(client_graph, devices)
  run_partitions(executors_and_partitions, inputs, outputs)
\end{python}
\end{leftbar}

其中,在每个计算设备上,启动一个\code{Executor}执行分配给它的\code{PartitionGraph}。当某一个计算设备执行完所分配的\code{PartitionGraph}之后,\code{ExecutorBarrier}的计数器加\ascii{1},直至所有设备完成\code{PartitionGraph}列表的执行,\code{barrier.wait()}阻塞操作退出。

跨设备的\code{PartitionGraph}之间可能存在数据依赖关系,它们之间通过插入\code{Send/Recv}节点完成交互。事实上,在本地模式中,\code{Send/Recv}通过\code{Rendezvous}完成数据交换的。\code{Send}将数据放在\code{Rendezvous}上,而\code{Recv}则根据标识从\code{Rendezvous}取走。其中,\code{Send}不阻塞,而\code{Recv}是阻塞的。

\subsection{关闭会话}

\begin{leftbar}
\begin{c++}
Status DirectSession::Close() {
  cancellation_manager_->StartCancel();
  {
    mutex_lock l(closed_lock_);
    if (closed_) return Status::OK();
    closed_ = true;
  }
  return Status::OK();
}
\end{c++}
\end{leftbar}

如\refig{local-cancellation-manager}所示,将\ascii{Step}注册给\code{DirectSession}的\code{CancellationManager}之中。当\code{DirectSession}被关闭时,\code{DirectSession}的\code{CancellationManager},将取消这次\ascii{step}的执行过程。

\begin{figure}[H]
\centering
\includegraphics[width=0.9\textwidth]{figures/local-cancellation-manager.png}
\caption{CancellationManager工作原理}
 \label{fig:local-cancellation-manager}
\end{figure}

\begin{leftbar}
\begin{c++}
Status DirectSession::Run(
   const NamedTensorList& inputs,
   const std::vector<string>& output_names,
   const std::vector<string>& target_nodes,
   std::vector<Tensor>* outputs) {
  // step\_cancellation\_manager is passed to `OpKernelContext`
  CancellationManager step_cancellation_manager;

  // Register this step with session's cancellation manager, so that
  // `Session::Close()` will cancel the step.
  CancellationToken cancellation_token =
      cancellation_manager_->get_cancellation_token();
  bool already_cancelled = !cancellation_manager_->RegisterCallback(
      cancellation_token, [&step_cancellation_manager]() {
        step_cancellation_manager.StartCancel();
      });
  // ignore others...
}
\end{c++}
\end{leftbar}

当前\ascii{Step}的\code{CancellationManager}最终会传递给\code{OpKernelContext}。\ascii{Kernel}实现计算时,如果保存了中间状态,可以向其注册相应的回调钩子。其中,每个回调钩子都有唯一的\code{token}标识。

当\ascii{Step}被取消时,回调钩子被调用,该\ascii{Kernel}可以取消该\ascii{OP}的计算。例如,\code{FIFOQueue}实现\code{TryEnqueue}时,便往本次\ascii{Step}的\code{CancellationManager}注册了回调钩子,用于取消该\ascii{Kernel}中间的状态信息。

\begin{leftbar}
\begin{c++}
void FIFOQueue::TryEnqueue(const Tuple& tuple, OpKernelContext* ctx,
                           DoneCallback callback) {
  CancellationManager* cm = ctx->cancellation_manager();
  CancellationToken token = cm->get_cancellation_token();
  bool already_cancelled;
  {
    mutex_lock l(mu_);
    already_cancelled = !cm->RegisterCallback(
        token, [this, cm, token]() { Cancel(kEnqueue, cm, token); });
  }
  // ignore others...
}
\end{c++}
\end{leftbar}


\section{剪枝}
\label{sec:graph-operation-prune}

\code{DirectSession::Run}执行时,首先完成\code{ClientGraph}的构造。事实上,\code{ClientGraph}的构造过程,主要完成\code{FullGraph}的剪枝算法,并生成\code{ClientGraph}。

\subsection{构建ClientGraph}

如\refig{local-simple-graph-execution-state}所示,\code{SimpleGraphExecutionState}实例持有\code{FullGraph}实例,并根据输入/输出列表,生成\code{ClientGraph}。

\begin{figure}[H]
\centering
\includegraphics[width=0.55\textwidth]{figures/local-simple-graph-execution-state.png}
\caption{生成\code{ClientGraph}}
 \label{fig:local-simple-graph-execution-state}
\end{figure}

其中,\code{BuildGraphOptions}包含了输入/输出列表,调用\code{SimpleGraphExecutionState::BuildGraph}生成\code{ClientGraph}实例。

\begin{leftbar}
\begin{c++}
namespace {
  BuildGraphOptions build_graph_options(
    const NamedTensorList& inputs,
    const std::vector<string>& outputs,
    const std::vector<string>& targets) {
    // sort inputs/outputs/targets
    std::vector<string> inputs_sorted(inputs.begin(), inputs.end());
    std::sort(inputs_sorted.begin(), inputs_sorted.end());

    std::vector<string> outputs_sorted(outputs.begin(), outputs.end());
    std::sort(outputs_sorted.begin(), outputs_sorted.end());

    std::vector<string> tn_sorted(targets.begin(), targets.end());
    std::sort(tn_sorted.begin(), tn_sorted.end());

    // build graph options
    BuildGraphOptions options;
    options.feed_endpoints = inputs_sorted;
    options.fetch_endpoints = outputs_sorted;
    options.target_nodes = tn_sorted;
    options.use_function_convention = !run_state_args->is_partial_run;
    return options;
  }
}

Status DirectSession::Run(
  const RunOptions& run_options,
  const NamedTensorList& inputs,
  const std::vector<string>& output_names,
  const std::vector<string>& target_nodes,
  std::vector<Tensor>* outputs,
  RunMetadata* run_metadata) {

  // 1. prune graph
  // client\_graph = prune(full\_graph, inputs, outputs)
  std::unique_ptr<SimpleClientGraph> client_graph;
  execution_state_->BuildGraph(
    build_graph_options(inputs, output_names, target_nodes), 
    &client_graph);
   
  // 2. split graph into partition by devices 
  // executors\_and\_partitions = split(client\_graph, devices)
  
  // 3. lauch executor per partition
  // def run\_partitions(executors\_and\_partitions, inputs, outputs):
  // \ \ frame = FunctionCallFrame()
  // \ \ frame.set\_args(inputs)
  // \ \ for (executor, partition) in executors\_and\_partitions: 
  // \ \ \ \ exec.run(part)
  // \ \ frame.get\_ret\_vals(outputs)

  return Status::OK();
}
\end{c++}
\end{leftbar}

\code{ClientGraph}初始来自原始的\code{FullGraph},调用\code{RewriteGraphForExecution}函数,将根据输入/输出,对\code{ClientGraph}实施改写操作,包括增加节点,或删除节点,最终生成\code{SimpleClientGraph}实例。

\begin{leftbar}
\begin{c++}
const DeviceAttributes& 
SimpleGraphExecutionState::local_device_attr() const {
  return device_set_->client_device()->attributes();
}

Status SimpleGraphExecutionState::BuildGraph(
  const BuildGraphOptions& options, 
  std::unique_ptr<SimpleClientGraph>* out) {
  // 1. create new\_graph from origin graph, 
  // which is client graph.
  std::unique_ptr<Graph> ng;
  ng.reset(new Graph(flib_def_.get()));
  CopyGraph(*graph_, ng.get());

  // 2. prune the client graph
  subgraph::RewriteGraphForExecution(
    ng.get(), options.feed_endpoints, options.fetch_endpoints,
    options.target_nodes, local_device_attr(),
    options.use_function_convention);
  }

  // 3. create SimpleClientGraph, and return it.
  std::unique_ptr<SimpleClientGraph> dense_copy(
      new SimpleClientGraph(std::move(flib)));
  CopyGraph(*ng, &dense_copy->graph);
  *out = std::move(dense_copy);

  return Status::OK();
}
\end{c++}
\end{leftbar}

因此,构建\code{ClientGraph}过程,其关键路径为\code{RewriteGraphForExecution},即剪枝算法。剪枝算法根据输入/输出列表,反向遍历\ascii{FullGraph},找到最小的依赖子图\code{ClientGraph}。

一般地,对于\code{ClientGraph}输入节点,扮演了起始节点;而输出节点,扮演了终止节点。因此,关于输入和输出,存在两个比较棘手的问题:

\begin{enum}
  \eitem{输入:当\code{ClientGraph}计算开始前,外部的运行时如何传递\code{Tensor}给输入节点};
  \eitem{输出:当\code{ClientGraph}计算完成后,外部的运行时又如何从输出节点获取\code{Tensor}}。
\end{enum}

存在两种媒介:\code{FunctionCallFrame}和\code{Rendezvous},外部运行时与输入/输出节点可以使用其中一种媒介交换数据。

\code{FunctionCallFrame}用于\code{Arg/RetVal}函数调用的\ascii{OP},用于函数调用时传递函数参数值,及其返回函数值。但是,它们仅适用于单进程的运行时环境。

\code{Rendezvous}用于\code{Send/Recv}消息发送的\ascii{OP},这是一种更为通用的通信方式,适用于分布式的运行时环境。

\subsection{基于Rendezvous}

如\refig{client-prune-graph}所示,根据\code{fetches}列表,反向搜索依赖的节点,直至\code{feeds},计算得到最小依赖的子图。

对于\code{Feed}的边实施剪枝,例如剪枝\code{ina:0})边,并在此处插入节点\code{Recv},并按照输入边的名字命名该节点,例如\code{\_recv\_ina\_0}。

同理,对于\code{Fetch}的边也实施剪枝,例如剪枝\code{f:0}边,并在此处插入节点\code{Send}节点,并按照输出边的名字命名该节点,例如\code{\_send\_f\_0}。

最终,通过插入\code{Source/Sink}节点,将剪枝后得到各个联通的子图进行汇总,形成一个完整的\ascii{DAG}图。

\begin{figure}[H]
\centering
\includegraphics[width=0.9\textwidth]{figures/client-prune-graph.png}
\caption{图剪枝:插入Send/Recv节点}
 \label{fig:client-prune-graph}
\end{figure}

\subsection{基于FunctionCallFrame}

但是,输入/输出通过\code{Rendezvous}交换数据可能存在性能上的瓶颈。因为待发送的\code{Tensor}需要携带发送设备,接收设备,\code{TensorId},共同组成了唯一的字符串标识,数据发送和接收需要花费很长的字符串解析的时间开销。

特殊地,对于本地模式,因为在同一进程内,使用\code{Rendezvous}交换数据存在不必要的性能损耗。可以使用基于\code{FunctionCallFrame}函数调用替代之。

因此,在本地模式下,可以使用\code{Arg/RetVal}分别替代\code{Send/Recv}节点,从而实现了函数调用交换数据的方式,替代原有基于\code{Rendezvous}交互数据的方式。

如\refig{client-prune-graph-function-ops}所示。对于\code{Feed}的边实施剪枝,例如剪枝\code{ina:0})边,并在此处插入节点\code{Arg},并按照输入边的名字命名该节点,例如\code{\_arg\_ina\_0}。

同理,对于\code{Fetch}的边也实施剪枝,例如剪枝\code{f:0}边,并在此处插入节点\code{RetVal}节点,并按照输出边的名字命名该节点,例如\code{\_retval\_f\_0}。

最终,通过插入\code{Source/Sink}节点,将剪枝后得到各个联通的子图进行汇总,形成一个完整的\ascii{DAG}图。

\begin{figure}[H]
\centering
\includegraphics[width=0.9\textwidth]{figures/client-prune-graph-function-ops.png}
\caption{图剪枝:插入Arg/RetVal节点}
 \label{fig:client-prune-graph-function-ops}
\end{figure}

\subsection{剪枝算法实现}

剪枝算法主要由\code{RewriteGraphForExecution}完成,主要包括\ascii{3}个子过程。

\begin{enum}
  \eitem{追加输入节点}
  \eitem{追加输出节点} 
  \eitem{反向剪枝}
\end{enum}

\begin{leftbar}
\begin{c++}
void RewriteGraphForExecution(Graph* g, bool use_function, 
    const ArraySlice<string>& fed_outputs,
    const ArraySlice<string>& fetch_outputs,
    const ArraySlice<string>& target_node_names,
    const DeviceAttributes& device_info) {
  FeedInputs(g, use_function, device_info, fed_outputs);

  std::vector<Node*> fetch_nodes;
  FetchOutputs(g, use_function, device_info, 
    fetch_outputs, &fetch_nodes);

  PruneForTargets(g, fetch_nodes, target_node_names);
}
\end{c++}
\end{leftbar}

\subsubsection{追加输入节点}

如\refig{local-prune-feed}所示,对于任意一条输入边实施剪枝时,插入相应的\code{Arg}或\code{Recv}节点,删除既有的边,并重新连接相应的边。

在计算图中,一条边唯一地由\code{TensorId}标识,它由\code{op:src\_output}二元组构成。前者表示边的上游节点,后者表示给边为上游节点的第几条边。

示例代码删除了部分不重要的逻辑,并搬迁了部分函数的职责,并在局部尝试部分函数提取,以便更好地还原算法的逻辑。其中,假设\code{Graph}可以按照\code{TensorId}索引节点和边。

\begin{figure}[H]
\centering
\includegraphics[width=0.6\textwidth]{figures/local-prune-feed.png}
\caption{剪枝:输入边}
 \label{fig:local-prune-feed}
\end{figure}

\begin{leftbar}
\begin{c++}
namespace {
  DataType data_type(Graph& g, const TensorId& tensor_id) {
    Node* upstream_node = g.upstream_node(tensor_id);
    return BaseType(upstream_node->output_type(tensor_id.src_output()));
  }

  Node* AppendRecvNode(Graph& g, 
    const TensorId& tensor_id, const DeviceAttributes& device_info) {
      Node* recv_node;
      NodeBuilder(strings::StrCat(
        "_recv_", tensor_id.op(), "_", tensor_id.src_output()), "_Recv")
        .Attr("tensor_type", data_type(g, tensor_id))
        .Attr("tensor_name", tensor_id.name())
        .Attr("send_device", device_info.name())
        .Attr("recv_device", device_info.name())
        .Attr("send_device_incarnation", device_info.incarnation())
        .Attr("client_terminated", true)
        .Finalize(g, &recv_node);
      return recv_node;
  }

  Node* AppendArgNode(Graph& g, size_t index, 
    const TensorId& tensor_id, const DeviceAttributes& device_info) {
    Node* arg_node;
    NodeBuilder(strings::StrCat(
      "_arg_", tensor_id.op(), "_", tensor_id.src_output()), "_Arg")
      .Attr("T", data_type(g, tensor_id))
      .Attr("index", index)
      .Finalize(g, &arg_node);
    return arg_node;
  }

  // 1. append arg/recv node
  Node* AppendNewNode(Graph& g, bool use_function, size_t index, 
    const TensorId& tensor_id,const DeviceAttributes& device_info) {
    if (use_function) {
      return AppendArgNode(g, index, tensor_id, device_info);
    } else {
      return AppendRecvNode(g, tensor_id, device_info);
    }
  }

  void AppendNewEdges(Graph& g, 
    Node* new_node, const TensorId& tensor_id) {
    // 2. add control edge between source node and new node.
    g.AddControlEdge(g.source_node(), new_node);

    Edge* old_edge = g.edge(tensor_id);
    
    // 3. add edge between new node and downstream node.
    g.AddEdge(new_node, 0, old_edge->dst(), old_edge->dst_input());
    
    // 4. remove old edge.
    g.RemoveEdge(old_edge);
  }
}

void FeedInputs(Graph& g, bool use_function,
  const DeviceAttributes& device_info,
  const ArraySlice<TensorId>& feeds) {
  for (size_t i = 0; i < feeds.size(); ++i) {
    Node* new_node = AppendNewNode(use_function, i, feeds[i]);
    AppendNewEdges(g, new_node, feeds[i]);
  }
}
\end{c++}
\end{leftbar}

\subsubsection{追加输出节点}

对于任意一条输出边实施剪枝时,插入相应的\code{RetVal}或\code{Send}节点,并将其与\code{Sink}节点通过控制依赖边连接。

如\refig{local-prune-fetch}所示,对输出边实施剪枝操作。新节点与上游节点的连接关系,在构造新节点时,通过\code{Input}已经指定。另外,函数直接返回了新节点(\code{RetVal/Send})为终止节点,因此没必要删除原来的边,其算法与输入边的处理存在微妙的差异。

\begin{figure}[H]
\centering
\includegraphics[width=0.6\textwidth]{figures/local-prune-feed.png}
\caption{剪枝:输出边}
 \label{fig:local-prune-fetch}
\end{figure}

\begin{leftbar}
\begin{c++}
namespace {
  Node* AppendSendNode(Graph& g, 
    const TensorId& tensor_id, const DeviceAttributes& device_info) {
    Node* send_node;
    NodeBuilder(strings::StrCat(
      "_send_", tensor_id.op(), "_", id.src_output()), "_Send")
      // 2. add edge between upstream node and send node.
      .Input(g.upstream_node(tensor_id), tensor_id.src_output())
      .Attr("tensor_name", tensor_id.name())
      .Attr("send_device", device_info.name())
      .Attr("recv_device", device_info.name())
      .Attr("send_device_incarnation",
            device_info.incarnation())
      .Attr("client_terminated", true)
      .Finalize(g, &send_node);
    return send_node;
  }

  Node* AppendRetvalNode(Graph& g, size_t index, 
    const TensorId& tensor_id, const DeviceAttributes& device_info) {
    Node* retval_node;
    NodeBuilder(strings::StrCat(
      "_retval_", tensor_id.op(), "_", tensor_id.src_output(), "_", index), 
      "_Retval")
      // 2. add edge between upstream node and retval node.
      .Input(g.upstream_node(tensor_id), tensor_id.src_output())
      .Attr("T", data_type(g, tensor_id))
      .Attr("index", index)
      .Finalize(g, &retval_node))
    return retval_node;
  }

  // 1. append retval/send node
  Node* AppendNewNode(Graph& g, bool use_function, size_t index, 
    const TensorId& tensor_id,const DeviceAttributes& device_info) {
    if (use_function) {
      return AppendRetvalNode(g, index, tensor_id, device_info);
    } else {
      return AppendSendNode(g, tensor_id, device_info);
    }
  }
}

void FetchOutputs(Graph& g, bool use_function,
  const DeviceAttributes& device_info,
  const ArraySlice<TensorId>& fetches,
  std::vector<Node*>& fetch_nodes) {
  for (size_t i = 0; i < fetches.size(); ++i) {
    Node* new_node = AppendNewNode(use_function, i, fetches[i]);
    
    // 3. add control edge between new node and sink node. 
    g->AddControlEdge(new_node, g->sink_node());

    fetch_nodes.push_back(new_node);
  }
}
\end{c++}
\end{leftbar}

\subsubsection{反向剪枝}

剪枝操作,其本质就是\ascii{DAG}反向的宽度优先遍历算法。首先,创建了一个队列,及其一个\code{visited}数组,后者用于记录已经遍历过的节点。初始化时,队列仅包含输出节点和输入节点(\code{targets})。当图遍历完毕后,不再\code{visited}里面的节点,表示本此执行不依赖于它,应从图中删除该节点,及其相关联的边。

经过剪枝后,将形成若干\ascii{DAG}子图。将入度为\code{0}的节点,与\code{Source}节点通过控制依赖边相连接;出度为\ascii{0}的节点,与\code{Sink}节点通过控制依赖边相连接,最终形成一个完整的\ascii{DAG}图。

\begin{leftbar}
\begin{c++}
namespace {
  void ReverseBFS(
    Graph* g, std::unordered_set<const Node*>& visited) {
    std::deque<const Node*> queue(visited.begin(), visited.end());
    while (!queue.empty()) {
      const Node* n = queue.front();
      queue.pop_front();
      for (const Node* in : n->in_nodes()) {
        if (visited.insert(in).second) {
          queue.push_back(in);
        }
      }
    }
  }

  void RemoveUnvisitedNodes(
    Graph* g, std::unordered_set<const Node*>& visited) {
    for (Node* n : g->nodes()) {
      if (visited.count(n) == 0 && !n->IsSource() && !n->IsSink()) {
        g->RemoveNode(n);
      }
    }
  }

  void PruneForReverseReachability(
    Graph* g, std::unordered_set<const Node*>& visited) {
    ReverseBFS(g, visited);
    RemoveUnvisitedNodes(g, visited);
  }

  void FixupSourceEdges(Graph* g, Node* n) {
    if (!n->IsSource() && n->in_edges().empty()) {
      g->AddControlEdge(g->source_node(), n);
    }  
  }

  void FixupSinkEdges(Graph* g, Node* n) {
    if (!n->IsSink() && n->out_edges().empty()) {
      g->AddControlEdge(n, g->sink_node());
    }  
  }

  void FixupSourceAndSinkEdges(Graph* g) {
    for (Node* n : g->nodes()) {
      FixupSourceEdges(g, n);
      FixupSinkEdges(g, n);
    }
  }

  void AppendTargetNodes(Graph& g, 
    const ArraySlice<string>& target_names,
    std::unordered_set<const Node*>& targets) {
    for (auto name : target_names) {
      Node* target = g.GetNodeBy(name);
      targets.insert(target);
    }
  }  
}

void PruneForTargets(Graph* g, 
  std::vector<Node*>& fetch_nodes,
  const ArraySlice<string>& target_names) {
  std::unordered_set<const Node*> targets(
    begin(fetch_nodes), end(fetch_nodes));

  AppendTargetNodes(g, target_names, targets);
  PruneForReverseReachability(g, targets);
  FixupSourceAndSinkEdges(g);
}
\end{c++}
\end{leftbar}

\section{分裂}
\label{sec:graph-operation-split}

如\refig{local-graph-split-by-device}所示,假如\code{d}节点放置在\ascii{GPU0}上执行,而其他节点放置在\ascii{CPU0}上执行。其中,节点\code{a}与\code{b}通过\code{Arg}输入数据;节点\code{f}将其结果输出到\code{RetVal}节点上。

\begin{figure}[H]
\centering
\includegraphics[width=0.9\textwidth]{figures/local-graph-split-by-device.png}
\caption{按本地设备集执行图分裂}
 \label{fig:local-graph-split-by-device}
\end{figure}

因此,计算图中存在若干条边跨越设备。对于跨越设备的边,运行时将其分裂,并就地插入\code{Send/Recv}边,分别用于原设备上发送数据,并在目标设备上接受数据,完成设备间的数据交换。如\refig{local-graph-split-insert-send-recv}所示。

其中,\code{Arg/RetVal}节点通过媒介\code{FunctionCallFrame}交换数据;\code{Send/Recv}节点通过媒介\code{Rendezvous}交换数据。

\begin{figure}[H]
\centering
\includegraphics[width=1.0\textwidth]{figures/local-graph-split-insert-send-recv.png}
\caption{跨设备OP之间插入Send/Recv节点}
 \label{fig:local-graph-split-insert-send-recv}
\end{figure}

\subsection{情况1}

最简单的情况下,\code{src}与\code{dst}在同一个\code{Partition}内。因此,直接将其划归在同一个\code{Partition}即可。

\begin{figure}[H]
\centering
\includegraphics[width=0.6\textwidth]{figures/split-graph-1.png}
\caption{情况1:src与dst在同一个Partition内}
 \label{fig:split-graph-1}
\end{figure}

\subsection{情况2}

如果\code{src}与\code{dst}不在同一个\code{Partition}内,但两者之间原来是通过普通边连接在一起的。因此,仅需要在它们中间增加\code{Send}与\code{Recv}节点,将其划归在两个不同的\code{Partition}内。

\begin{figure}[H]
\centering
\includegraphics[width=0.7\textwidth]{figures/split-graph-2.png}
\caption{情况2:src与dst不在同一个Partition内,但两者之间是普通边}
 \label{fig:split-graph-2}
\end{figure}

\subsection{情况3}

如果\code{src}与\code{dst}不在同一个\code{Partition}内,但两者之间原来是通过控制依赖边连接在一起的。

此时,需要在\code{src}侧增加一个\code{Const}的\code{DummyNode},并作为\code{src}的下游通过控制依赖边相连;最终,在通过\code{Send}将其值发送到对端。

在\code{dst}侧,\code{Recv}收到该值,使用\code{Identity}将其消费掉;最终,再将\code{Identity}与\code{dst}连接控制依赖边。

在这里,\code{Const}扮演生产者,\code{Identity}扮演消费者角色。既满足了跨设备间通信的需求,又满足原来\code{src}与\code{dst}之间的控制依赖的关系。但是,其缺点就是存在微妙的性能开销。

\begin{figure}[H]
\centering
\includegraphics[width=0.8\textwidth]{figures/split-graph-3.png}
\caption{情况3:src与dst不在同一个Partition内,但两者之间是控制依赖边}
 \label{fig:split-graph-3}
\end{figure}

\subsection{分裂算法实现}

分裂算法也是一个反向遍历图的算法。对于当前遍历的节点,将其标记为\code{dst};然后再寻找\code{dst}的所有输入边;遍历所有输入边,从而找到与改边相连的源节点,将其标记为\code{src}。

因此,更具上述讨论的三种情况,迭代实现\code{src}与\code{dst}两者之前的\code{Partition}划分算法。

\begin{leftbar}
\begin{c++}
namespace {
  
  using Edges = std::vector<const Edge*>;
  using Partitions = std::unordered_map<string, GraphDef>;

  void AddInput(NodeDef* dst, StringPiece src_name, int src_slot) {
    if (src_slot == Graph::kControlSlot) {
      dst->add_input(strings::StrCat("^", src_name));
    } else if (src_slot == 0) {
      dst->add_input(src_name.data(), src_name.size());
    } else {
      dst->add_input(strings::StrCat(src_name, ":", src_slot));
    }
  }

  Edges InputsOf(const Node* dst) {
    Edges inputs(dst->num_inputs(), nullptr);
    for (auto edge : dst.in_edges()) {
      if (edge->IsControlEdge()) {
        inputs.push_back(e);
      } else {
        inputs[edge->dst_input()] = edge;
      }
    }
    return inputs;
  }

  NodeDef* InitDstNodeDef(const Node& dst, NodeDef* dst_def) {
    dst_def = dst.def();
    dst_def->set_device(dst.assigned_device_name());
    dst_def->clear_input();
    return dst_def;  
  }

  NodeDef* AddDummyConst(const PartitionOptions& opts, GraphDef* gdef,
                         const Edge* edge, Status* status) {
    const Node* src = edge->src();
    Tensor tensor(DT_FLOAT, TensorShape({0}));
    NodeDef* result = gdef->add_node();
    *status = NodeDefBuilder(opts.new_name(src->name()), "Const")
                  .Device(src->assigned_device_name())
                  .Attr("dtype", DT_FLOAT)
                  .Attr("value", tensor)
                  .Finalize(result);
    return result;
  }

  NodeDefBuilder::NodeOut BuildSendFrom(
      const PartitionOptions& opts,
      GraphDef* src_graph,
      const Edge* edge,
      NodeDefBuilder::NodeOut& send_from) {
    if (edge->IsControlEdge()) {
      // Case 3: DummyNode(Const) -ctrl-> src -> send  
      NodeDef* dummy = AddDummyConst(opts, src_graph, edge);
      AddInput(dummy, edge->src()->name(), Graph::kControlSlot);
      send_from.Reset(dummy->name(), 0, DT_FLOAT);
    } else {
      // Case 2: src -> send  
      send_from.Reset(edge->src()->name(),
                      edge->src_output(), 
                      EdgeType(edge));
    }
  }

  void SetSendRecvAttrs(
      const PartitionOptions& opts, 
      const Edge* edge,
      NodeDefBuilder* builder) {
    builder->Attr("tensor_name",
                  strings::StrCat("edge_", edge->id(), "_", edge->src()->name()));
    builder->Attr("send_device", edge->src()->assigned_device_name());
    builder->Attr("send_device_incarnation",
                  static_cast<int64>(
                      opts.get_incarnation(edge->src()->assigned_device_name())));
    builder->Attr("recv_device", edge->dst()->assigned_device_name());
    builder->Attr("client_terminated", false);
  }

  NodeDef* AddSend(
      const PartitionOptions& opts, 
      GraphDef* gdef, 
      const Edge* edge,
      NodeDefBuilder::NodeOut send_from) {
    NodeDef* send = gdef->add_node();
    NodeDefBuilder builder(opts.new_name(edge->src()->name()), "_Send");
    SetSendRecvAttrs(opts, edge, &builder);
    builder.Device(edge->src()->assigned_device_name())
           .Input(send_from)
           .Finalize(send);
    return send;
  }

  NodeDef* AddRecv(const PartitionOptions& opts, const GraphInfo& g_info,
                   GraphDef* gdef, const Edge* edge, NodeDef** real_recv,
                   Status* status) {
    NodeDef* recv = gdef->add_node();
    NodeDefBuilder builder(opts.new_name(src->name()), "_Recv");
    SetSendRecvAttrs(opts, edge, &builder);
    builder.Device(dst->assigned_device_name())
           .Attr("tensor_type", EdgeType(edge))
           .Finalize(recv);
    return recv;

    if (edge->IsControlEdge()) {
      // Case 3: Recv -> Identity -contrl-> dst
      NodeDef* id = gdef->add_node();
      NodeDefBuilder(opts.new_name(src->name()), "Identity")
          .Device(dst->assigned_device_name())
          .Input(recv->name(), 0, cast_dtype)
          .Finalize(id);
      return id;
    } else {
      return recv;
    }
  }

  void InsertSendRecv(
      const PartitionOptions& opts,
      GraphDef* src_graph, 
      Edge* edge, 
      GraphDef* dst_graph, 
      NodeDef* dst_def) {
    NodeDefBuilder::NodeOut send_from;
    BuildSendFrom(opts, src_graph, edge, send_from);

    NodeDef* send = AddSend(opts, src_graph, edge, send_from);
    NodeDef* recv = AddRecv(opts, dst_graph, edge);

    if (edge->IsControlEdge()) {
      // Case 3: In fact, recv is identity.
      AddInput(dst_def, recv->name(), Graph::kControlSlot);
    } else {
      AddInput(dst_def, recv->name(), 0);
    }
  }
}

Status Partition(const PartitionOptions& opts, 
                 Partitions& partitions, Graph& client_graph) {
  for (const Node* dst : client_graph.op_nodes()) {
    // 1. find dst node
    GraphDef* dst_graph = &partitions[opts.node_to_loc(dst)];
    NodeDef* dst_def = InitDstNodeDef(*dst, dst_graph->add_node());
    
    // 2. search all input edges.
    for (const Edge* edge : InputsOf(dst)) {
      // 3. find src node: edge->src()
      GraphDef* src_graph = &partitions[opts.node_to_loc(src)];

      // skip sink/source nodes.
      if (!edge->src()->IsOp()) 
        continue;  

      // Case 1: same partition
      if (src_graph == dst_graph) {
        AddInput(dst_def, src->name(), edge->src_output());
        continue;
      }

      // Case 2-3: different partition
      InsertSendRecv(opts, src_graph, edge, dst_graph, dst_def);
    }
  }
}
\end{c++}
\end{leftbar}

\subsection{回调函数}

在\code{PartitionOptions}中,存在两个重要的回调函数。\code{NodeToLocFunc}用于图分裂;\code{NewNameFunc}用于给新增加的节点命名,例如\code{Send/Recv}。

\begin{leftbar}
\begin{c++}
struct PartitionOptions {
  typedef std::function<string(const Node*)> NodeToLocFunc;
  NodeToLocFunc node_to_loc = nullptr;

  typedef std::function<string(const string&)> NewNameFunc;
  NewNameFunc new_name = nullptr;

  // ignore others...
};
\end{c++}
\end{leftbar}

对于图分裂,存在两种最基本的分裂方法。

\begin{leftbar}
\begin{c++}
string SplitByDevice(const Node* node) {
  return node->assigned_device_name();
}

string SplitByWorker(const Node* node) {
  string task, device;
  DeviceNameUtils::SplitDeviceName(
      node->assigned_device_name(), &task, &device);
  return task;
}
\end{c++}
\end{leftbar}

在本地模式下,\code{NodeToLocFunc}被配置为\code{SplitByDevice}。如图\code{intraprocess-splity-by-device}所示。

\begin{figure}[H]
\centering
\includegraphics[width=0.7\textwidth]{figures/intraprocess-splity-by-device.png}
\caption{本地模式:SplitByDevice}
 \label{fig:intraprocess-splity-by-device}
\end{figure}


在分布式模式下,\code{Master}的\code{NodeToLocFunc}被配置为\code{SplitByWorker};而\code{Worker}
的\code{NodeToLocFunc}被配置为\code{SplitByDevice}。

因此,在分布式模式下,图分裂经历了两级分离。第一级按照\code{SplitByWorker}分裂,将图划分到各个\code{Worker}上去;第二级按照\code{SplitByDevice},再将图划分到各个计算设备上去。

\begin{figure}[H]
\centering
\includegraphics[width=0.7\textwidth]{figures/interprocess-splity-by-worker.png}
\caption{分布式模式:两级分裂}
 \label{fig:interprocess-splity-by-worker}
\end{figure}

\section{执行}
\label{sec:graph-operation-exec}

接下来,运行时将并发执行各个\code{PartitionGraph}。如\refig{local-graph-execution}所示,每个\code{PartitionGraph}启动一个\code{Executor},实现并发执行图的计算。

每个\code{Executor}将执行\code{PartitionGraph}的拓扑排序算法,将入度为\ascii{0}的\ascii{OP}追加到\code{ready\_queue}之中,并将其关联的\ascii{OP}的入度减\ascii{1}。调度器调度\code{ready\_queue}之中\ascii{OP
},并将其放入\code{ThreadPool}中执行对应的\ascii{Kernel}实现。

在所有\code{Partition}开始并发执行之前,需要外部将其输入传递给相应的\code{Arg}节点;当所有\code{Partition}完成计算后,外部再从\code{RetVal}节点中取走数据。其中,\code{Arg/RetVal}节点之间的数据时通过\code{FunctionCallFrame}完成交互的。

如果\code{PartitionGraph}之间需要跨设备交换数据,生产者将其放在\code{Send}节点,消费者通过\code{Recv}节点获取数据。其中,发送方不阻塞;接收方如果数据未到,则发生阻塞直至超时。此外,\code{Send/Recv}节点之间的数据是通过\code{Rendezvous}完成交互的。

\begin{figure}[H]
\centering
\includegraphics[width=1.0\textwidth]{figures/local-graph-execution.png}
\caption{执行图}
 \label{fig:local-graph-execution}
\end{figure}

因此,执行图计算需要解决如下\ascii{3}个核心问题:

\begin{enum}
  \eitem{输入/输出处理}
  \eitem{设备间数据交换} 
  \eitem{执行\code{PartitionGraph}}
\end{enum}

\subsection{输入}

在某个设备上,\code{PartitionGraph}的起始节点为\code{Arg}节点,结束节点为\code{RetVal}节点。整个过程可以看成函数调用过程,\code{Arg}用于传递函数参数,\code{RetVal}用于返回函数值。

更确切地说,\code{Arg}完成\code{PartitionGraph}的输入,\code{RetVal}完成
\code{PartitionGraph}的输出。对于\code{Arg}节点,其调用时序为:\code{set\_arg -> get\_arg}。其中,前者由\code{DirectSession}在启动\code{Executor}列表之前,通过调用\code{FunctionCallFrame.SetArgs(feeds)},传递输入参数列表的值;后者由\code{Arg}的\ascii{Kernel}实现调用。

\begin{leftbar}
\begin{c++}
Status DirectSession::Run(
  const RunOptions& run_options,
  const NamedTensorList& inputs,
  const std::vector<string>& output_names,
  const std::vector<string>& target_nodes,
  std::vector<Tensor>* outputs,
  RunMetadata* run_metadata) {

  // 1. prune graph
  // client\_graph = prune(full\_graph, inputs, outputs)
   
  // 2. split graph into partition by devices 
  // executors\_and\_partitions = split(client\_graph, devices)
  ExecutorsAndKeys* executors_and_keys = ... // ignore implements...
  
  // 3. lauch executor per partition
  // def run\_partitions(executors\_and\_partitions, inputs, outputs):
  // \ \ frame = FunctionCallFrame()
  // \ \ frame.set\_args(inputs)
  // \ \ for (executor, partition) in executors\_and\_partitions: 
  // \ \ \ \ exec.run(part)
  // \ \ frame.get\_ret\_vals(outputs)

  // 3.1 construct FunctionCallFrame
  FunctionCallFrame call_frame(
    executors_and_keys->input_types,
    executors_and_keys->output_types);
  
  // 3.2 frame.set\_args(inputs)
  // 3.2.1 construct feeds list
  gtl::InlinedVector<Tensor, 4> feed_args(inputs.size());
  for (const auto& it : inputs) {
    // (first, second) => (tensor\_name, tensor)
    feed_args[executors_and_keys->input_name_to_index[it.first]] = it.second;
  }

  // 3.2.2 frame.set\_args(feeds)
  call_frame.SetArgs(feed_args);
  
  // 3.3 concurent execution
  // for (executor, partition) in executors\_and\_partitions:
  // \ \ executor.run(partition) 

  // 3.4 fetch outputs.
}
\end{c++}
\end{leftbar}

而\code{frame.get\_arg}则有\code{Arg}来获取,并且\code{Arg}将其输出到\code{PartitionGraph}中的第一个计算节点。

\begin{leftbar}
\begin{c++}
struct ArgOp : OpKernel {
  explicit ArgOp(OpKernelConstruction* ctx) : OpKernel(ctx) {
    ctx->GetAttr("T", &dtype_);
    ctx->GetAttr("index", &index_);
  }

  void Compute(OpKernelContext* ctx) override {
    auto frame = ctx->call_frame();

    Tensor val;
    frame->GetArg(index_, &val);

    // put it into downsteram op's input.
    ctx->set_output(0, val); 
  }

 private:
  int index_;
  DataType dtype_;
};
\end{c++}
\end{leftbar}

\subsection{并发执行}

经过图分裂后,运行为每个\code{Partition}启动一个\code{Executor}。为了监听所有\code{Executor}是否全部完成,创建了一个\code{ExecutorBarrier}。并且在启动所有\code{Executor}之后,调用\code{executors\_done.Wait()}阻塞,等待所有\code{Executor}完成执行。

当完成一个\code{Executor}中完成,\code{ExecutorBarrier}的计算器减\ascii{1}(初始值为\code{num\_executors}),直至为\ascii{0},将调用其完成钩子,最终触发\code{executors\_done.Notify()}。

\begin{leftbar}
\begin{c++}
Status DirectSession::Run(
  const RunOptions& run_options,
  const NamedTensorList& inputs,
  const std::vector<string>& output_names,
  const std::vector<string>& target_nodes,
  std::vector<Tensor>* outputs,
  RunMetadata* run_metadata) {

  // 1. prune graph
  // client\_graph = prune(full\_graph, inputs, outputs)
   
  // 2. split graph into partition by devices 
  // executors\_and\_partitions = split(client\_graph, devices)
  ExecutorsAndKeys* executors_and_keys = ... // ignore implements...
  
  // 3. lauch executor per partition
  // def run\_partitions(executors\_and\_partitions, inputs, outputs):
  // \ \ frame = FunctionCallFrame()
  // \ \ frame.set\_args(inputs)
  // \ \ for (executor, partition) in executors\_and\_partitions: 
  // \ \ \ \ exec.run(part)
  // \ \ frame.get\_ret\_vals(outputs)

  // 3.1 construct FunctionCallFrame
  FunctionCallFrame call_frame(
    executors_and_keys->input_types,
    executors_and_keys->output_types);
  
  // 3.2 frame.set\_args(inputs)
  // 3.2.1 construct feeds list
  gtl::InlinedVector<Tensor, 4> feed_args(inputs.size());
  for (const auto& it : inputs) {
    // (first, second) => (tensor\_name, tensor)
    feed_args[executors_and_keys->input_name_to_index[it.first]] = it.second;
  }

  // 3.2.2 frame.set\_args(feeds)
  call_frame.SetArgs(feed_args);
  
  // 3.3 concurent execution
  // barrier = ExecutorBarrier(executors\_and\_partitions.size())
  // for (executor, partition) in executors\_and\_partitions:
  // \ \ executor.run(partition) 
  // barrier.wait()
  RunState run_state(&devices_);
  run_state.rendez = new IntraProcessRendezvous(device_mgr_.get());
  
  // 3.3.1 notify when finished.
  size_t num_executors = executors_and_keys->items.size();
  ExecutorBarrier* barrier = new ExecutorBarrier(
      num_executors, run_state.rendez, [&run_state](const Status& ret) {
        {
          mutex_lock l(run_state.mu_);
          run_state.status.Update(ret);
        }
        run_state.executors_done.Notify();
      });

  Executor::Args args;
  args.call_frame = &call_frame;
  args.rendezvous = run_state.rendez;
  args.runner = [this, pool](Executor::Args::Closure c) {
    SchedClosure(pool, std::move(c));
  };

  // 3.3.2 lauch all executors.
  for (const auto& item : executors_and_keys->items) {
    item.executor->RunAsync(args, barrier->Get());
  }

  // 3.3.3 wait until all executors finished.
  WaitForNotification(&run_state, 
      &step_cancellation_manager,
      GetTimeoutInMs(run_options));

  // 3.4 fetch outputs.
}
\end{c++}
\end{leftbar}

\subsection{输出}

同理,对于\code{RetVal}节点,其调用时序为:\code{set\_ret\_val -> get\_ret\_val}。前者由\code{RetVal}完成,后者由\code{DirectSession}完成。

\begin{leftbar}
\begin{c++}
struct RetvalOp : OpKernel {
  explicit RetvalOp(OpKernelConstruction* ctx) : OpKernel(ctx) {
    ctx->GetAttr("T", &dtype_);
    ctx->GetAttr("index", &index_);
  }

  void Compute(OpKernelContext* ctx) override {
    // get upstream op's output.
    const Tensor& val = ctx->input(0); 

    auto frame = ctx->call_frame();
    frame->SetRetval(index_, val);
  }

 private:
  int index_;
  DataType dtype_;
};
\end{c++}
\end{leftbar}

等所有\code{Executor}运行结束后,\code{DirectSession}便可以从\code{FunctionCallFrame}中取出所有输出值,并将其放置在\code{outputs},并返回\ascii{Client}。

\begin{leftbar}
\begin{c++}
Status DirectSession::Run(
  const RunOptions& run_options,
  const NamedTensorList& inputs,
  const std::vector<string>& output_names,
  const std::vector<string>& target_nodes,
  std::vector<Tensor>* outputs,
  RunMetadata* run_metadata) {
  
  // 1. prune graph
  // client\_graph = prune(full\_graph, inputs, outputs)
   
  // 2. split graph into partition by devices 
  // executors\_and\_partitions = split(client\_graph, devices)
  executors_and_keys = ... // ignore implements...
  
  // 3. lauch executor per partition
  // def run\_partitions(executors\_and\_partitions, inputs, outputs):
  // \ \ frame = FunctionCallFrame()
  // \ \ frame.set\_args(inputs)
  // \ \ for (executor, partition) in executors\_and\_partitions: 
  // \ \ \ \ exec.run(part)
  // \ \ frame.get\_ret\_vals(outputs)

  // 3.1 construct FunctionCallFrame
  FunctionCallFrame call_frame(
    executors_and_keys->input_types,
    executors_and_keys->output_types);
  
  // 3.2 frame.set\_args(inputs)
  // 3.2.1 construct feeds list
  gtl::InlinedVector<Tensor, 4> feed_args(inputs.size());
  for (const auto& it : inputs) {
    // (first, second) => (tensor\_name, tensor)
    feed_args[executors_and_keys->input_name_to_index[it.first]] = it.second;
  }

  // 3.2.2 frame.set\_args(feeds)
  call_frame.SetArgs(feed_args);
  
  // 3.3 concurent execution
  // barrier = ExecutorBarrier(executors\_and\_partitions.size())
  // for (executor, partition) in executors\_and\_partitions:
  // \ \ executor.run(partition) 
  // barrier.wait()
  RunState run_state(&devices_);
  run_state.rendez = new IntraProcessRendezvous(device_mgr_.get());
  
  // 3.3.1 notify when finished.
  size_t num_executors = executors_and_keys->items.size();
  ExecutorBarrier* barrier = new ExecutorBarrier(
      num_executors, run_state.rendez, [&run_state](const Status& ret) {
        {
          mutex_lock l(run_state.mu_);
          run_state.status.Update(ret);
        }
        run_state.executors_done.Notify();
      });

  Executor::Args args;
  args.call_frame = &call_frame;
  args.rendezvous = run_state.rendez;
  args.runner = [this, pool](Executor::Args::Closure c) {
    SchedClosure(pool, std::move(c));
  };

  // 3.3.2 lauch all executors.
  for (const auto& item : executors_and_keys->items) {
    item.executor->RunAsync(args, barrier->Get());
  }

  // 3.3.3 wait until all executors finished.
  WaitForNotification(&run_state, 
      &step_cancellation_manager,
      GetTimeoutInMs(run_options)); 

  // 3.4 fetch outputs. 
  // 3.4.1 frame.get\_get\_ret\_vals
  std::vector<Tensor> sorted_outputs;
  Status s = call_frame.ConsumeRetvals(&sorted_outputs);

  // 3.4.2 emplace to outputs, and return to client.
  outputs->reserve(sorted_outputs.size());
  for (int i = 0; i < output_names.size(); ++i) {
    const string& output_name = output_names[i];
    outputs->emplace_back(
      std::move(sorted_outputs[
        executors_and_keys->output_name_to_index[output_name]]));
  }
}
\end{c++}
\end{leftbar}

至此,整个\code{DirectSession.Run}解读完毕。但是,\code{Partition}中节点如何被调度执行的,\code{Partition}之间的\code{Send/Recv}是如何工作的呢?

因此,在最后一公里,还要探究三件事情。

\begin{enum}
  \eitem{\code{SendOp}与\code{RecvOp}的工作原理}
  \eitem{\code{IntraProcessRendezvous}的工作原理} 
  \eitem{\code{Executor}的调度算法}
\end{enum}

\section{设备间通信}

\code{SendOp/RecvOp}通过\code{Rendezvous}交换数据的;它实现了消息发送/接受,与具体消息传递相解耦。例如,在单进程内,\code{SendOp/RecvOp}基于\code{IntraProcessRendezvous}传递数据的;而在多进程环境中,\code{SendOp/RecvOp}则可以基于\code{GrpcRendezvous}传递数据。

首先,探究这两个\ascii{OP}的工作原理;然后,再探究本地模式下,\code{IntraProcessRendezvous}的工作原理。

\subsection{SendOp实现}

如\refig{local-send-recv-ops}所示,进程内的\code{Send/Recv}通过唯一的标识\code{ParsedKey}实现数据的交换。

\begin{figure}[H]
\centering
\includegraphics[width=0.8\textwidth]{figures/local-send-recv-ops.png}
\caption{进程内\code{SendOp}与\code{RecvOp}的数据交换}
 \label{fig:local-send-recv-ops}
\end{figure}

参考\code{SendOp}的\ascii{Kernel}实现,看起来非常复杂,但是它实际上就做了一件事情。首先,它构造设备间通信的关键字\code{ParsedKey},然后调用\code{Rendezvous.Send}操作,将上游\ascii{OP}输入到\code{SendOp}的\code{Tensor}发送到\code{Rendezvous}缓存之中,该操作是非阻塞的。

其中,\code{ParsedKey}包括:发送设备,接受设备,设备全局标识,及其待发送\code{Tensor}的标识(\code{src:output\_index})组成。

\begin{leftbar}
\begin{c++}
struct SendOp : OpKernel {
  explicit SendOp(OpKernelConstruction* ctx) : OpKernel(ctx) {
    string send_device;
    ctx->GetAttr("send_device", &send_device);

    string recv_device;
    ctx->GetAttr("recv_device", &recv_device);

    uint64 send_device_incarnation;
    ctx->GetAttr(
        "send_device_incarnation",
        reinterpret_cast<int64*>(&send_device_incarnation));

    string tensor_name;
    ctx->GetAttr("tensor_name", &tensor_name);

    key_prefix_ = GetRendezvousKeyPrefix(
        send_device, recv_device,
        send_device_incarnation, tensor_name);

    GetRendezvousKey(key_prefix_, {0, 0}, &parsed_key_.buf_);
    Rendezvous::ParseKey(parsed_key_.buf_, &parsed_key_);

    if (!ctx->GetAttr("_hostmem_sendrecv", &hostmem_sendrecv_).ok()) {
      hostmem_sendrecv_ = false;
    }
  }

  void Compute(OpKernelContext* ctx) override {
    Rendezvous::Args args;
    args.device_context = ctx->op_device_context();
    args.alloc_attrs = ctx->input_alloc_attr(0);
    
    // get it from upstream op's output, and as this op's input.
    ctx->rendezvous()->Send(
        CreateParsedkey(ctx), args, ctx->input(0),
        ctx->is_input_dead());
  }
 
 private:
  Rendezvous::ParsedKey CreateParsedkey(OpKernelContext* ctx) {
    FrameAndIter frame_iter = GetFrameAndIter(ctx, hostmem_sendrecv_);
    if (frame_iter == FrameAndIter(0, 0)) {
      return parsed_key_;
    } else {
      Rendezvous::ParsedKey in_loop_parsed;
      GetRendezvousKey(key_prefix_, frame_iter, &in_loop_parsed.buf_);
      Rendezvous::ParseKey(in_loop_parsed.buf_, &in_loop_parsed);
      return in_loop_parsed;
    }  
  }

 private:
  string key_prefix_;
  Rendezvous::ParsedKey parsed_key_;
  bool hostmem_sendrecv_;
};
\end{c++}
\end{leftbar}

\subsection{RecvOp实现}

同理地,可以猜测\code{Recv}的\ascii{Kernel}的实现了。它首先构造\code{Rendezvous}的\code{ParsedKey},然后调用\code{Rendezvous.RecvAsync}操作,从\code{Rendezvous}取出相应的\code{Tensor}。

这是一个异步操作,当\code{Rendezvous}中数据可获取,便开始执行回调函数\code{done\_cb},它将其得到的\code{Tensor}输出到下游\ascii{OP}。

\begin{leftbar}
\begin{c++}
struct RecvOp : AsyncOpKernel {
  explicit RecvOp(OpKernelConstruction* ctx) : AsyncOpKernel(ctx) {
    string send_device;
    ctx->GetAttr("send_device", &send_device);
  
    string recv_device;
    ctx->GetAttr("recv_device", &recv_device);

    uint64 send_device_incarnation;
    ctx->GetAttr(
        "send_device_incarnation",
        reinterpret_cast<int64*>(&send_device_incarnation));
  
    string tensor_name;
    ctx->GetAttr("tensor_name", &tensor_name);

    key_prefix_ = GetRendezvousKeyPrefix(
        send_device, recv_device,
        send_device_incarnation, tensor_name);
  
    GetRendezvousKey(key_prefix_, {0, 0}, &parsed_key_.buf_);
    Rendezvous::ParseKey(parsed_key_.buf_, &parsed_key_));
    if (!ctx->GetAttr("_hostmem_sendrecv", &hostmem_sendrecv_).ok()) {
      hostmem_sendrecv_ = false;
    }
  }

  void ComputeAsync(OpKernelContext* ctx, DoneCallback done) override {
    Rendezvous::Args args;
    args.device_context = ctx->op_device_context();
    args.alloc_attrs = ctx->output_alloc_attr(0);

    ctx->rendezvous()->RecvAsync(
      CreateParsedKey(ctx), args, CreateDoneCallback(ctx));
  }

 private:
  Rendezvous::ParsedKey CreateParsedKey(OpKernelContext* ctx) {
    FrameAndIter frame_iter = GetFrameAndIter(ctx, hostmem_sendrecv_);
    if (frame_iter == FrameAndIter(0, 0)) {
      return parsed_key_;
    } else {
      Rendezvous::ParsedKey in_loop_parsed;
      GetRendezvousKey(key_prefix_, frame_iter, &in_loop_parsed.buf_);
      Rendezvous::ParseKey(in_loop_parsed.buf_, &in_loop_parsed);
      return in_loop_parsed;
    }  
  }

  Rendezvous::DoneCallback CreateDoneCallback(OpKernelContext* ctx) {
    using namespace std::placeholders;
    return std::bind([ctx](DoneCallback done, const Status& s, 
        const Rendezvous::Args&, const Rendezvous::Args&, 
        const Tensor& val, bool is_dead) {
          ctx->SetStatus(s);
          if (s.ok()) {
            if (!is_dead) {
              // put it into downstream op's input.
              ctx->set_output(0, val);  
            }
            *ctx->is_output_dead() = is_dead;
          }
          done();
        },
        std::move(done), _1, _2, _3, _4, _5);  
  }

 private:
  string key_prefix_;
  Rendezvous::ParsedKey parsed_key_;
  bool hostmem_sendrecv_;
};
\end{c++}
\end{leftbar}
\end{content}
\input{contents/distributed}

\part{模型训练}
\input{contents/bp}
\begin{savequote}[45mm]
\ascii{Any fool can write code that a computer can understand. Good programmers write code that humans can understand.}
\qauthor{\ascii{- Martin Flower}}
\end{savequote}

\chapter{数据加载} 
\label{ch:input-pipeline}

\begin{content}

一般地,\ascii{TensorFlow}输入样本数据到训练/推理子图中执行运算,存在三种读取样本数据的方法:

\begin{enum}
  \eitem{数据注入:通过字典\code{feed\_dict}将数据传递给\code{Session.run},以替代\code{Placeholder}的输出\code{Tensor}的值;}
  \eitem{数据管道:通过构造输入子图,并发地从文件中读取样本数据;}
  \eitem{数据预加载:对于小数据集,使用\code{Const}或\code{Variable}直接持有数据。}
\end{enum}

基于大型数据集的训练或推理任务,样本数据的输入常常使用数据的管道模式,确保高的吞吐率,提高训练/推理的执行效率。该过程使用队列实现输入子图与训练/推理子图之间的数据交互与异步控制。

本章将重点论述数据加载的\ascii{Pipeline}的工作机制,并深入了解\ascii{TensorFlow}并发执行的协调机制,及其队列在并发执行中扮演的角色。

\end{content}

\section{数据注入}

\begin{content}

数据注入是最为常见的数据加载的方法,它通过字典\code{feed\_dict}将样本数据传递给\code{Session.run},或者\code{Tensor.eval}方法;其中,字典的关键字为\code{Tensor}的名字,值为样本数据。

\ascii{TensorFlow}将按照字典中\code{Tensor}的名字,将样本数据替换该\code{Tensor}的值。

\begin{leftbar}
\begin{python}
x = tf.placeholder(tf.float32, [None, 784])
y_ = tf.placeholder(tf.float32, [None, 10])

with tf.Session():
  batch_xs, batch_ys = mnist.train.next_batch(100)
  sess.run(train_step, feed_dict={x: batch_xs, y_: batch_ys})
\end{python}
\end{leftbar}

一般地,\code{feed\_dict}可以替代任何\code{Tensor}的值。但是,常常使用\code{Placeholder}表示其输出\code{Tensor}的值未确定,待使用\code{feed\_dict}替代。

\end{content}

\section{数据预加载}

\begin{content}

可以使用\code{Const}或\code{Variable}直接持有数据,将数据预加载至内存中,提升执行效率。该方法仅适用于小数据集,当样本数据集比较大时,内存资源消耗非常可观。这里以\ascii{mnist}数据集为例,讲解数据预加载的使用方法。

\begin{leftbar}
\begin{python}
from tensorflow.examples.tutorials.mnist import input_data

data_sets = input_data.read_data_sets('/tmp/mnist/data')
\end{python}
\end{leftbar}

\subsection{使用Const}

由于\code{Const OP}输出\code{Tensor}的值是直接内联在计算图中。如果该\code{Const OP}在图中被使用多次,可能造成重复的冗余数据,白白浪费了不必要的内存资源。

\begin{leftbar}
\begin{python}
with tf.name_scope('input'):
  input_images = tf.constant(data_sets.train.images)
  input_labels = tf.constant(data_sets.train.labels)
\end{python}
\end{leftbar}

\subsection{使用Variable}

可以使用不可变、非训练的\code{Variable}替代\code{Const}。一旦初始化了该类型的\code{Variable},便不能改变其值,从而具备\code{Const}的属性。

用于数据预加载的\code{Variable}与用于训练的\code{Variable}之间存在差异,它将置位\code{trainable=False},系统不会将其归类于\code{GraphKeys.TRAINABLE\_VARIABLES}集合中。在训练过程中,系统不会对其实施更新操作。

另外,在构造该类型的\code{Variable}时,还将设置\code{collections=[]},系统不会将其归类于\code{GraphKeys.GLOBAL\_VARIABLES}集合中。在训练过程中,系统不会对其实施\ascii{Checkpoint}操作。

为了创建不可变、非训练的\code{Variable},此处写了一个简单的工厂方法。

\begin{leftbar}
\begin{python}
def immutable_variable(initial_value):
  initializer = tf.placeholder(
    dtype=initial_value.dtype,
    shape=initial_value.shape)
  return tf.Variable(initializer, trainable=False, collections=[])
\end{python}
\end{leftbar}

\code{immutable\_variable}使用传递进来的\code{initial\_value}构造\code{Placeholder}的类型与形状信息,并以此作为\code{Variable}的初始值。可以使用\code{immutable\_variable}创建不可变的,用于数据预加载的\code{Variable}。

\begin{leftbar}
\begin{python}
with tf.name_scope('input'):
  input_images = immutable_variable(data_sets.train.images)
  input_labels = immutable_variable(data_sets.train.labels)
\end{python}
\end{leftbar}

\subsection{批次预加载}

可以构建\ascii{Pipeline},结合数据预加载机制,实现样本的批式加载。首先,使用\code{tf.train.slice\_input\_producer}在每个\ascii{epoch}开始时将整个样本空间随机化,每次从样本集合中随机采样获取一个训练样本。

\begin{leftbar}
\begin{python}
def one(input_xs, input_ys, num_epochs)
  return tf.train.slice_input_producer(
    [input_xs, input_ys], num_epochs=num_epochs)
\end{python}
\end{leftbar}

然后,使用\code{tf.train.batch}每次得到一个批次的样本数据。

\begin{leftbar}
\begin{python}
def batch(x, y, batch_size)
  return tf.train.batch(
    [x, y], batch_size=batch_size)
\end{python}
\end{leftbar}

对于使用\code{Variable}预加载数据,可以如下方式获取一个批次的样本数据。

\begin{leftbar}
\begin{python}
with tf.name_scope('input'):
  input_images = immutable_variable(data_sets.train.images)
  input_labels = immutable_variable(data_sets.train.labels)

  image, label = one(input_images, input_labels, epoch=1)
  batch_images, batch_labels = batch(image, label, batch_size=100)
\end{python}
\end{leftbar}

事实上,\code{tf.train.slice\_input\_producer}将构造样本队列,通过\code{QueueRunner}并发地通过执行\code{Enqueue}操作,将训练样本逐一加入到样本队列中去。在每次迭代训练启动时,通过调用\code{DequeueMany}一次性获取\code{batch\_size}个的批次样本数据到训练子图中去。

\end{content}

\section{数据管道}

\begin{content}

一个典型的数据加载的\ascii{Pipeline(Input Pipeline)},包括如下几个重要数据处理实体:

\begin{enum}
  \eitem{文件名称队列:将文件名称的列表加入到该队列中;}
  \eitem{读取器:从文件名称队列中读取文件名(出队);并根据数据格式选择相应的文件读取器,解析文件的记录;}
  \eitem{解码器:解码文件记录,并转换为数据样本;}
  \eitem{预处理器:对数据样本进行预处理,包括正则化,白化等;}
  \eitem{样本队列:将处理后的样本数据加入到样本队列中。}
\end{enum}

以\ascii{mnist}数据集为例,假如数据格式为\code{TFRecord}。首先,使用\code{tf.train.string\_input\_producer}构造了一个持有文件名列表的\code{FIFOQueue}队列(通过执行\code{EnqueueMany OP}),并且在每个\ascii{epoch}周期内实现文件名列表的随机化。

\subsection{构建文件名队列}

\begin{leftbar}
\begin{python}
def input_producer(num_epochs):
  return tf.train.string_input_producer(
    ['/tmp/mnist/train.tfrecords'], num_epochs=num_epochs)
\end{python}
\end{leftbar}

构造好了文件名队列之后,使用\code{tf.TFRecordReader}从文件名队列中获取文件名(出队,通过调用执行\code{Dequeue OP}),并从文件中读取样本记录(\ascii{Record})。然后,使用\code{tf.parse\_single\_example}解析出样本数据。

\subsection{读取器}

\begin{leftbar}
\begin{python}
def parse_record(filename_queue):
  reader = tf.TFRecordReader()
  _, serialized_example = reader.read(filename_queue)
  features = tf.parse_single_example(
      serialized_example,
      features={
          'image_raw': tf.FixedLenFeature([], tf.string),
          'label': tf.FixedLenFeature([], tf.int64),
      })
  return features
\end{python}
\end{leftbar}

\subsection{解码器}

接着对样本数据进行解码,及其可选的预处理过程,最终得到训练样本。

\begin{leftbar}
\begin{python}
def decode_image(features):
  image = tf.decode_raw(features['image_raw'], tf.uint8)
  image.set_shape([28*28])

  # Convert from [0, 255] -> [-0.5, 0.5] floats.
  image = tf.cast(image, tf.float32) * (1. / 255) - 0.5
  return image

def decode_label(features):
  label = tf.cast(features['label'], tf.int32)
  return label

def one_example(features):
  return decode_image(features), decode_label(features)
\end{python}
\end{leftbar}

\subsection{构建样本队列}

可以使用\code{tf.train.shuffle\_batch}构建一个\code{RandomShuffleQueue}队列,将解析后的训练样本追加在该队列中(通过执行\code{Enqueue OP});当迭代执行启动时,将批次获取\code{batch\_size}个样本数据(通过执行\code{DequeueMany OP})。

\begin{leftbar}
\begin{python}
def shuffle_batch(image, label, batch_size):
    # Shuffle the examples and collect them into batch\_size
    # batches.(Uses a RandomShuffleQueue)
    images, labels = tf.train.shuffle_batch(
      [image, label], batch_size=batch_size, num_threads=2,
      capacity=1000 + 3 * batch_size,
      # Ensures a minimum amount of shuffling of examples.
      min_after_dequeue=1000)
    return images, labels
\end{python}
\end{leftbar}

\subsection{输入子图}

最后,将整个程序传接起来便构造了一个输入子图。

\begin{leftbar}
\begin{python}
def inputs(num_epochs, batch_size):
  with tf.name_scope('input'):
    filename_queue = input_producer(num_epochs)
    features = parse_record(filename_queue)
    image, label = one_example(features)
    return shuffle_batch(image, label, batch_size)
\end{python}
\end{leftbar}

\end{content}

\section{数据协同}

\begin{content}

事实上,数据加载的\ascii{Pipeline}其本质是构造一个输入子图,实现并发\ascii{IO}操作,使得训练过程不会因操作\ascii{IO}而阻塞,从而实现\ascii{GPU}的利用率的提升。

对于输入子图,数据流的处理划分为若干阶段(\ascii{Stage}),每个阶段完成特定的数据处理功能;各阶段之间以队列为媒介,完成数据的协同和交互。

如下图所示,描述了一个典型的神经网络的训练模式。整个流水线由两个队列为媒介,将其划分为3个阶段。

\begin{figure}[!htbp]
\centering
\includegraphics[width=0.9\textwidth]{figures/tf-input-pipeline.png}
\caption{模型训练工作流}
 \label{fig:tf-input-pipeline}
\end{figure}

\subsection{阶段1}

\code{string\_input\_producer}构造了一个\code{FIFOQueue}的队列,它是一个有状态的\ascii{OP}。根据\code{shuffle}选项,在每个\ascii{epoch}开始时,随机生成文件列表,并将其一同追加至队列之中。

\begin{figure}[!htbp]
\centering
\includegraphics[width=0.7\textwidth]{figures/tf-input-pipeline-stage-1.png}
\caption{阶段1:模型训练工作流}
 \label{fig:tf-input-pipeline-stage-1}
\end{figure}

\subsubsection{随机化}

首先,执行名为\code{filenames}的\code{Const OP},再经过\code{RandomShuffle}将文件名称列表随机化。

\subsubsection{Epoch控制}

为了实现\ascii{epoch}的计数,实现巧妙地设计了一个名为\code{epochs}的本地变量。其中,本地变量仅对本进程的多轮Step之间共享数据,并且不会被训练子图实施更新。

在\code{Session.run}之前,系统会执行本地变量列表的初始化,将名为\code{epochs}的\code{Variable}实施零初始化。

\ascii{epoch}的计数功能由\code{CountUpTo}完成,它的工作原理类似于\ascii{C++}的\code{i++}。它持有\code{Variable}的引用,及其上限参数\code{limit}。每经过一轮\ascii{epoch},使其\code{Variable}自增1,直至达到\code{num\_epochs}数目。

其中,当\ascii{epoch}数到达\code{num\_epochs}时,\code{CountUpTo}将自动抛出\code{OutOfRangeError}异常。详细实现可以参考\code{CountUpToOp}的\ascii{Kernel}实现。

\begin{leftbar}
\begin{c++}
template <class T>
struct CountUpToOp : OpKernel {
  explicit CountUpToOp(OpKernelConstruction* ctxt)
    : OpKernel(ctxt) {
    OP_REQUIRES_OK(ctxt, ctxt->GetAttr("limit", &limit_));
  }

  void Compute(OpKernelContext* ctxt) override {
    T before_increment;
    {
      mutex_lock l(*ctxt->input_ref_mutex(0));
      
      // Fetch the old tensor
      Tensor tensor = ctxt->mutable_input(0, true);
      T* ptr = &tensor.scalar<T>()();      
      before_increment = *ptr;
      
      // throw OutOfRangeError if exceed limit
      if (*ptr >= limit_) {
        ctxt->SetStatus(errors::OutOfRange(
            "Reached limit of ", limit_));
        return;
      }
      // otherwise increase 1
      ++*ptr;
    }
    // Output if no error.
    Tensor* out_tensor;
    OP_REQUIRES_OK(ctxt, ctxt->allocate_output(
        "output", TensorShape({}), &out_tensor));
    out_tensor->scalar<T>()() = before_increment;
  }

private:
  T limit_;
};
\end{c++}
\end{leftbar}

\subsubsection{入队操作}

事实上,将文件名列表追加到队列中,执行的是\code{EnqueueMany},类似于\code{Assign}修改\code{Variable}的值,\code{EnqueueMany}也是一个有状态的\ascii{OP},它持有队列的句柄,直接完成队列的状态更新。

在此处,\code{EnqueueMany}将被\code{Session.run}执行,系统反向遍历,找到依赖的\code{Identity},发现控制依赖于\code{CountUpTo},此时会启动一次\ascii{epoch}计数,直至到达\code{num\_epoch}数目抛出\code{OutOfRangeError}异常。同时,\code{Identity}依赖于\code{RandomShuffle},以便得到随机化了的文件名列表。

\subsubsection{QueueRunner}

另外,在调用\code{tf.train.string\_input\_producer}时,将往计算图中注册一个特殊的\ascii{OP}:\code{QueueRunner},并且将其添加到\code{GraphKeys.QUEUE\_RUNNERS}集合中。并且,一个\code{QueueRunner}持有一个或多个\code{Enqueue, EnqueueMany}类型的\ascii{OP}。

\subsection{阶段2}

\code{Reader}从文件名队列中按照\ascii{FIFO}的顺序获取文件名,并按照文件名读取文件记录,成功后对该记录进行解码和预处理,将其转换为数据样本,最后将其追加至样本队列中。

\subsubsection{读取器}

事实上,实现构造了一个\code{ReaderRead}的\ascii{OP},它持有文件名队列的句柄,从队列中按照\ascii{FIFO}的顺序获取文件名。

因为文件的格式为\code{TFRecord},\code{ReaderRead}将委托调用\code{TFRecordReader}的\ascii{OP},执行文件的读取。最终,经过\code{ReaderRead}的运算,将得到一个序列化了的样本。

\subsubsection{解码器}

得到序列化了的样本后,将使用合适的解码器实施解码,从而得到一个期望的样本数据。可选地,可以对样本实施预处理,例如\code{reshape}等操作。

\subsubsection{入队操作}

得到样本数据后,将启动\code{QueueEnqueue}的运算,将样本追加至样本队列中去。其中,\code{QueueEnqueue}是一个有状态的\ascii{OP},它持有样本队列的句柄,直接完成队列的更新操作。

实施上,样本队列是一个\code{RandomShuffleQueue},使用出队操作实现随机采样。

\subsubsection{并发执行}

为了提高\ascii{IO}的吞吐率,可以启动多路并发的\code{Reader}与\code{Decoder}的工作流,并发地将样本追加至样本队列中去。其中,\code{RandomShuffleQueue}是线程安全的,支持并发的入队或出队操作。

\subsection{阶段3}

当数据样本累计至一个\code{batch\_size}时,训练/推理子图将取走该批次的样本数据,启动一次迭代计算(常称为一次\ascii{Step})。

\subsubsection{出队操作}

事实上,训练子图使用\code{DequeueMany}获取一个批次的样本数据。

\subsubsection{迭代执行}

一般地,一次迭代运行,包括两个基本过程:前向计算与反向梯度传递。\ascii{Worker}任务使用\ascii{PS}任务更新到本地的当前值,执行前向计算得到本次迭代的损失。

然后,根据本次迭代的损失,反向计算各个\ascii{Variable}的梯度,并更新到\ascii{PS}任务中;\ascii{PS}任务更新各个\code{Variable}的值,并将当前值广播到各个\ascii{Worker}任务上去。

\subsubsection{Checkpoint}

\ascii{PS}任务根据容错策略,周期性地实施\ascii{Checkpoint}。将当前所有\code{Variable}的数据,及其图的元数据,包括静态的图结构信息,持久化到外部存储设备上,以便后续恢复计算图,及其所有\code{Variable}的数据。

\subsection{Pipeline节拍}

例如,往\code{FIFOQueue}的队列中添加文件名称列表,此时调用\code{EnqueueMany}起始的子图计算,其中包括执行所依赖的\code{CountUpTo}。当\code{CountUpTo}达到\code{limit}上限时,将自动抛出\code{OutOfRangeError}异常。

扮演主程序的\code{QueueRunner},捕获\code{coord.join}重新抛出的\code{OutOfRangeError}异常,随后立即关闭相应的队列,并且退出该线程的执行。队列被关闭之后,入队操作将变为非法;而出队操作则依然合法,除非队列元素为空。

同样的道理,下游\ascii{OP}从队列(文件名队列)中出队元素,一旦该队列元素为空,则自动抛出\code{OutOfRangeError}异常。该阶段对应的\code{QueueRunner}将感知该异常的发生,然后捕获异常并关闭下游的队列(样本队列),退出线程的执行。

在\ascii{Pipeline}的最后阶段,\code{train\_op}从样本队列中出队批次训练样本时,队列为空,并且队列被关闭了,则抛出\code{OutOfRangeError}异常,最终停止整个训练任务。
\end{content}
\input{contents/saver}
\input{contents/monitored-session}

%%%%%%%%%%%%%%%%%%%%%
\appendix

\part{附录}
\chapter{代码阅读} 
\label{ch:code-reading}

\begin{content}

在程序员的日常工作之中,绝大多数时间都是在\emph{阅读代码},而不是在写代码。但是,阅读代码往往是一件很枯燥的事情,尤其当遇到了一个不漂亮的设计,反抗的心理往往更加强烈。事实上,变换一下习惯、思路和方法,代码阅读其实是一个很享受的过程。

阅读代码的模式,实践和习惯,集大成者莫过于希腊作者\ascii{Diomidis Spinellis}的经典之作:\ascii{Code Reading, The Open Source Perspective}。本文从另外一个视角出发,谈谈我阅读代码的一些习惯,期待找到更多知音的共鸣。

\end{content}

\section{工欲善其事,必先利其器}

\begin{content}

首先,阅读代码之前先准备好一个称心如意的工具箱,包括\ascii{IDE}, \ascii{UML},脑图等工具。我主要使用的编程语言包括\ascii{C++, Scala, Java, Ruby, Python};我更偏向使用\ascii{JetBrains}公司的产品,其很多习惯用法对程序员都很贴心。

其次,高效地使用快捷键,这是一个良好的代码阅读习惯,它极大地提高了代码阅读的效率和质量。例如,查看类层次关系,函数调用链,方法引用点等等。

\begin{remark}
拔掉鼠标,减低对鼠标的依赖。当发现没有鼠标而导致工作无法进行下去时,尝试寻找对应的快捷键。通过日常的点滴积累,工作效率必然能够得到成倍的提高。
\end{remark}

\end{content}

\section{力行而后知之真}

\begin{content}

阅读代码一种常见的反模式就是通过\ascii{Debug}的方式来阅读代码。作者不推荐这种代码阅读的方式,其一,因为运行时线程间的切换很容易导致方向的迷失;其二,了解代码调用栈对于理解系统行为并非见得有效,因为其包含太多实现细节,不易发现问题的本质。

但在阅读代码之前,有几件事情是必须做的。其一,手动地构建一次工程,并运行测试用例;其二,亲自动手写几个\ascii{Demo}感受一下。

先将工程跑起来,目的不是为了\ascii{Debug}代码,而是在于了解工程构建的方式,及其认识系统的基本结构,并体会系统的使用方式。

如果条件允许,可以尝试使用\ascii{ATDD}的方式,发现和挖掘系统的行为。通过这个过程,将自己当成一个客户,思考系统的行为,这是理解系统最重要的基石。

\end{content}

\section{发现领域模型}

\begin{content}

阅读代码,不是为了了解每个类,每个函数干什么,而是为了挖掘更本质,更不易变化的知识。事实上,发现\emph{领域模型}是阅读代码最重要的一个目标,因为领域模型是系统的灵魂所在。通过代码阅读,找到系统本质的知识,并通过自己的模式表达出来,才能真正地抓住系统的脉络,否则一切都是空谈。

例如,在阅读\tf{}的\ascii{Python}实现的客户端代码时,理顺计算图的领域模型,对于理解\ascii{TensorFlow}的编程模型,及其系统运行时的行为极其重要。

\begin{figure}[!htbp]
\centering
\includegraphics[width=0.9\textwidth]{figures/py-graph.png}
\caption{领域对象:Graph}
 \label{fig:py-graph}
\end{figure}

\end{content}

\section{挖掘系统架构}

\begin{content}

阅读代码犹如在大海中航行,系统架构图就是航海图。阅读代码不能没有整体的系统概念,否则收效不佳,阅读质量大大折扣。必须拥有系统思维,并明确目标,才不至于迷失方向。

首要的任务,就是找到系统的边界,并能够以抽象的思维思考外部系统的行为特征。其次,理清系统中各组件之间的交互,关联关系,及其职责,对于理解整个系统的行为极为重要。

例如,对于\ascii{TensorFlow},\ascii{C API}是衔接前后端系统的桥梁。理解\ascii{C API}的设计,基本能够猜测前后端系统的行为。

\begin{figure}[!h]
\centering
\includegraphics[width=0.9\textwidth]{figures/tf-architecture-simple.png}
\caption{TensorFlow系统架构}
 \label{fig:tf-architecture-simple}
\end{figure}

\end{content}

\section{细节是魔鬼}

\begin{content}

纠结于细节,将导致代码阅读代码的效率和质量大大折扣。例如,日志打印,解决某个\ascii{Bug}的补丁实现,某版本分支的兼容方案,某些变态需求的锤子代码实现等等。

阅读代码的一个常见的反模式就是「给代码做批注」。这是一个高耗低效,投入产出比极低的实践。一般地,越是优雅的系统,注释越少;越是复杂的系统,再多的注释也是于事无补。

我有一个代码阅读的习惯,为代码阅读建立一个单独的\ascii{code-reading}分支,一边阅读代码,一边删除这些无关的代码。

\begin{leftbar}
\begin{scala}
$ git checkout -b code-reading
\end{scala}
\end{leftbar}

删除这些噪声后,你会发现系统根本没有想象之中那么复杂。现实中,系统的复杂性,往往都是不成熟的设计和实现导致的额外复杂度。随着对系统的深入理解,很多细节都会自然地浮出水面,所有神秘的面纱都将被揭开而公示天下。

\end{content}

\section{适可而止}

\begin{content}

阅读代码的一个常见的反模式就是「一根筋走到底,不到黄河绝不死心」。程序员都拥有一颗好奇心,总是对不清楚的事情感兴趣。例如,消息是怎么发送出去的?任务调度工作原理是什么?数据存储怎么做到的?;虽然这种勇气值得赞扬,但在代码阅读时绝对不值得鼓励。

还有另外一个常见的反模式就是「追踪函数调用栈」。这是一个极度枯燥的过程,常常导致思维的僵化;因为你永远活在作者的阴影下,完全没有自我。

我个人阅读代码的时候,函数调用栈深度绝不超过\ascii{3},然后使用抽象的思维方式思考底层的调用。因为我发现,随着年龄的增长,曾今值得骄傲的记忆力,现在逐渐地变成自己的短板。当我尝试追踪过深的调用栈之后,之前的阅读信息完全地消失记忆了。

也就是说,我更习惯于「广度遍历」,而不习惯于「深度遍历」的阅读方式。这样,我才能找到系统隐晦存在的「分层概念」,并理顺系统的层次结构。

\end{content}

\section{发现她的美}

\begin{content}

三人行,必有我师焉。在代码阅读代码时,当发现好的设计,包括实现模式,习惯用法等,千万不要错过;否则过上一段时间,这次代码阅读对你来说就没有什么价值了。

当我发现一个好的设计时,我会尝试使用类图,状态机,序列图等方式来表达设计;如果发现潜在的不足,将自己的想法补充进去,将更加完美。

例如,当我阅读\ascii{Hamcrest}时,尝试画画类图,并体会它们之间关系,感受一下设计的美感,也是受益颇多的。

\begin{figure}[!h]
\centering
\includegraphics[width=0.9\textwidth]{figures/hamcrest.png}
\caption{组合式设计}
 \label{fig:hamcrest}
\end{figure}

\end{content}

\section{尝试重构}

\begin{content}

因为这是一次代码阅读的过程,不会因为重构带来潜在风险的问题。在一些复杂的逻辑,通过重构的等价变换可以将其变得更加明晰,直观。

对于一个巨函数,我常常会提取出一个抽象的代码层次,以便发现它潜在的本质逻辑。例如,这是一个使用\ascii{Scala}实现的\ascii{ArrayBuffer},当需要在尾部添加一个元素时,既有的设计是这样子的。

\begin{leftbar}
\begin{python}
def +=(elem: A): this.type = {
  if (size + 1 > array.length) {
    var newSize: Long = array.length
    while (n > newSize)
      newSize *= 2
    newSize = math.min(newSize, Int.MaxValue).toInt
  
    val newArray = new Array[AnyRef](newSize)
    System.arraycopy(array, 0, newArray, 0, size)
    array = newArray
  }
  array(size) = elem.asInstanceOf[AnyRef]
  size += 1
  this
}
\end{python}
\end{leftbar}

这段代码给阅读造成了极大的障碍,我会尝试通过快速的函数提取,发现逻辑的主干。

\begin{leftbar}
\begin{python}
def +=(elem: A): this.type = {
  if (atCapacity)
    grow()
  addElement(elem)
}
\end{python}
\end{leftbar}

至于\code{atCapacity, grow, addElement}是怎么实现的,压根不用关心,因为我已经达到阅读代码的效果了。

\end{content}

\section{形式化}

\begin{content}

当阅读代码时,有部分人习惯画程序的「流程图」。相反,我几乎从来不会画「流程图」,因为流程图反映了太多的实现细节,而不能深刻地反映算法的本质。

我更倾向于使用「形式化」的方式来描述问题。它拥有数学的美感,简洁的表达方式,及其高度抽象的思维,对挖掘问题本质极其关键。

例如,对于\ascii{FizzBuzzWhizz}的问题,相对于冗长的文字描述,或流程图,形式化的方式将更加简单,并富有表达力。以\ascii{3, 5, 7}为输入,形式化后描述后,可清晰地挖掘出问题的本质所在。

\begin{leftbar}
\begin{python}
r1: times(3) => Fizz || 
    times(5) => Buzz ||
    times(7) => Whizz

r2: times(3) && times(5) && times(7) => FizzBuzzWhizz ||
    times(3) && times(5) => FizzBuzz  ||
    times(3) && times(7) => FizzWhizz ||
    times(5) && times(7) => BuzzWhizz

r3: contains(3) => Fizz

rd: others => string of others

spec: r3 || r2 || r1 || rd
\end{python}
\end{leftbar}

\end{content}

\section{实例化}

\begin{content}

实例化是认识问题的一种重要方法,当逻辑非常复杂时,一个简单例子往往使自己豁然开朗。在理想的情况下,实例化可以做成自动化的测试用例,并以此描述系统的行为。

如果存在某个算法和实现都相当复杂时,也可以通过实例化探究算法的工作原理,这对于理解问题本身大有益处。

以\ascii{Spark}中划分\ascii{DAG}算法为例。以\ascii{G}为起始节点,从后往前按照\ascii{RDD}的依赖关系,依次识别出各个\ascii{Stage}的边界。

\begin{figure}[!htbp]
\centering
\includegraphics[width=0.9\textwidth]{figures/spark-stage-dag.png}
\caption{Spark:Stage划分算法}
 \label{fig:spark-stage-dag}
\end{figure}

\begin{itemize}
 \item \ascii{Stage 3}的划分
   \begin{enum}
     \eitem{\ascii{G}与\ascii{B}之间是窄依赖,规约为同一\ascii{Stage(3)};}
     \eitem{\ascii{B}与\ascii{A}之间是宽依赖,\ascii{A}为新的起始\ascii{RDD},递归调用此过程;}
     \eitem{\ascii{G}与\ascii{F}之间是宽依赖,\ascii{F}为新的起始\ascii{RDD},递归调用此过程。} 
   \end{enum}

 \item \ascii{Stage 1}的划分
   \begin{enum}
     \eitem{\ascii{A}没有父亲\ascii{RDD},\ascii{Stage(1)}划分结束。特殊地\ascii{Stage(1)}仅包含\ascii{RDD A}。}
   \end{enum}
 \item \ascii{Stage 2}的划分
   \begin{enum}
     \eitem{因\ascii{RDD}之间的关系都为窄依赖,规约为同一个\ascii{Stage(2)};}
     \eitem{直至\ascii{RDD C, E},因没有父亲\ascii{RDD},\ascii{Stage(2)}划分结束。}
   \end{enum} 
\end{itemize}

最终,形成了\ascii{Stage}的依赖关系,依次提交\ascii{TaskSet}至\ascii{TaskScheduler}进行调度执行。

\end{content}

\section{独乐乐,不如众乐乐}

\begin{content}

与他人分享你的经验,也许可以找到更多的启发;尤其对于熟知该领域的人沟通,如果是\ascii{Owner}就更好了,肯定能得到意外的惊喜和收获。

也可以通过各种渠道,收集他人的经验,并结合自己的思考,推敲出自己的理解,如此才能将知识放入自己的囊中。

阅读代码,不是一个人的世界;应该走出去,多参加一些社区活动,了解生态圈中主流的研究方向,技术动态,产业发展,对于理解业务是极其有帮助的。


\end{content}


\input{contents/learning}

%%%%%%%%%%%%%%%%%%%%
\backmatter
% \listoffigures
% \myclearpage

% \listoftables
% \myclearpage

\bibliographystyle{alpha}
\input{contents/biblio}

\end{document}
%%%% 正文部分结束
%%%%%%%%------------------------------------------------------------------------
